\documentclass[a4paper, 12pt]{article}
\title{}
\usepackage{geometry}
\usepackage{float}
\usepackage{subfigure}
\usepackage[justification=centering]{caption}
\usepackage{enumerate}
\usepackage{multirow}
\usepackage{listings}
\lstset{
    escapechar=`,
    language=C++,
    numbers=left,
    tabsize=2,
    prebreak=\raisebox{0ex}[0ex][0ex]{\ensuremath{\hookleftarrow}},
    frame=single,
    breaklines=true,
}
\usepackage{graphicx}
\graphicspath{ {./} }
\usepackage{nameref}
\usepackage{amsmath}
\usepackage{amssymb}
\usepackage{amsfonts}
\usepackage[linesnumbered,ruled]{algorithm2e}
\usepackage{tikz}
\usetikzlibrary{calc,patterns,decorations.pathmorphing,decorations.markings,positioning,automata}
\usepackage{pgfplots}
\usepackage{pgfplotstable}
\usepackage{makecell}
\usepackage{ulem}
\usepackage{verbatim}

\begin{document}

\section*{Part II: Implementation} \label{sec:intro}
The purpose of this exercise was to use the Finite Element Method (FEM) 
to approximate, and study, the solution to the problem presented in 
Eq. \ref{eq:StrongForm}.

\begin{align}
u_t &= u_{xx}  \label{eq:StrongForm} \\
u(0,t) = 0 &, u(\pi, t) = 0 
\end{align}

\noindent
This equation was solved on the domain $x\in[0,\pi]$ and $t\in(0, T]$.
It was subject to the initial conditions

\begin{equation}
u(x,0) = \sin(2x)
\end{equation}

\noindent
and

\[
  u(x, 0) = \begin{cases}
    x \quad x \in (0, \frac{\pi}{2}] \\
    x - \pi \quad x \in( \frac{\pi}{2}, \pi)
  \end{cases}
\]

\noindent
The problem domain was divided into a uniform and structured mesh.
The rule for determining element size is shown in the equation below.

\begin{equation}
  h =  \frac{ \pi}{ N+1}
\end{equation}

\noindent
Here, $N$ is the number of degrees of freedom; $N+1$ is the number of elements.
Tests were conducted for $N=10, 20, 40, 80$ for the first initial condition. 

A computer code was written to evaluate the posed problems. 
The results follow this section and the code itself 
is appended to this report.
Tabulated data, in the form of error norms and convergence order 
for the first initial condition
are presented in Tables \ref{tab:feT1} through \ref{tab:cnT10}.
Figure \ref{fig:jump} shows the approximated solution
to the second initial condition at various times.

\begin{figure}[H]
  \centering
  \includegraphics[width=14cm, height=11cm]{jump}
  \caption{ Approximated solution to initial condition 2.
            Backward Euler with a time step size of 0.01.
            Mesh has 10 elements.}
  \label{fig:jump}
\end{figure}

\begin{table}[H]
\caption{Errors and convergence orders for Forward Euler, final time of 1.0.}
\vspace{0.1in}
\centering
\begin{tabular}{|c|c|c|c| c| c| c| c|}
\hline
 $N+1$  & $\Delta t$ & $||e_h(\cdot)||_{L^2([0,1])}$ & Order  & $||e_h(\cdot)||_{L^\infty([0,1])}$ & Order& $||e_h(\cdot)||_h$& Order \\
 \hline            %   L_2         % O(L_2)% L_inf     % O(L_inf)%  H_1        %O(H_1) 
     10  & 7.4e-03 &  4.086e-03 & -     & 3.460e-03 & -     & 6.495e-02 & -     \\
     20  & 1.9e-03 &  1.076e-03 & 1.925 & 9.123e-04 & 1.923 & 3.317e-02 & 0.970 \\
     40  & 4.6e-04 &  2.960e-04 & 1.862 & 2.280e-04 & 2.001 & 1.651e-02 & 1.001 \\
     80  & 1.1e-04 &  6.647e-05 & 2.154 & 5.686e-05 & 2.004 & 8.234e-03 & 1.004 \\
\hline
\end{tabular}
\label{tab:feT1}
\end{table}

\begin{table}[H]
\caption{Errors and convergence orders for Backward Euler, final time of 1.0.}
\vspace{0.1in}
\centering
\begin{tabular}{|c|c|c|c| c| c| c| c|}
\hline
 $N+1$  & $\Delta t$ & $||e_h(\cdot)||_{L^2([0,1])}$ & Order  & $||e_h(\cdot)||_{L^\infty([0,1])}$ & Order& $||e_h(\cdot)||_h$& Order \\
 \hline            %   L_2         % O(L_2)% L_inf     % O(L_inf)%  H_1        %O(H_1) 
     10  & 0.1     & 1.661e-02 & -     & 1.348e-02 & -     & 1.338e-01 & -\\
     20  & 0.05    & 1.022e-02 & 0.700 & 8.375e-03 & 0.689 & 1.033e-01 & 0.373\\
     40  & 0.025   & 4.877e-03 & 1.067 & 3.935e-03 & 1.089 & 7.057e-02 & 0.549\\
     80  & 0.0125  & 2.285e-03 & 1.097 & 1.832e-03 & 1.103 & 4.803e-02 & 0.557\\
\hline
\end{tabular}
\label{tab:beT1}
\end{table}

\begin{table}[H]
\caption{Errors and convergence orders for Crank-Nicolson, final time of 1.0.}
\vspace{0.1in}
\centering
\begin{tabular}{|c|c|c|c|c| c| c| c| c|}
\hline
 $N+1$  & $\Delta t$ &  $||e_h(\cdot)||_{L^2([0,1])}$ & Order  & $||e_h(\cdot)||_{L^\infty([0,1])}$ & Order& $||e_h(\cdot)||_h$& Order \\
 \hline           %   L_2         % O(L_2)% L_inf     % O(L_inf)%  H_1        %O(H_1) 
     10  & 0.1    & 4.000e-03 & -     & 3.387e-03 & -     & 6.424e-02 & -     \\
     20  & 0.05   & 1.202e-03 & 1.737 & 1.025e-03 & 1.726 & 3.517e-02 & 0.869 \\
     40  & 0.025  & 2.812e-04 & 2.010 & 2.409e-04 & 2.089 & 1.697e-02 & 1.051 \\
     80  & 0.0125 & 6.539e-05 & 2.101 & 5.599e-05 & 2.105 & 8.169e-03 & 1.055 \\
\hline
\end{tabular}
\label{tab:cnT1}
\end{table}

\begin{table}[H]
\caption{Errors and convergence orders for Forward Euler, final time of 10.0.}
\vspace{0.1in}
\centering
\begin{tabular}{|c|c|c|c| c| c| c| c|}
\hline
 $N+1$  & $\Delta t$  & $||e_h(\cdot)||_{L^2([0,1])}$ & Order  & $||e_h(\cdot)||_{L^\infty([0,1])}$ & Order& $||e_h(\cdot)||_h$& Order \\
 \hline            %   L_2         % O(L_2)% L_inf     % O(L_inf)%  H_1        %O(H_1) 
     10  & 7.4e-03 &  4.632e-18 & - & 3.282e-18 & - & 6.494e-09 & -\\
     20  & 1.9e-03 &  1.794e-18 & - & 2.481e-18 & - & 1.339e-09 & -\\
     40  & 4.6e-04 &  5.797e-19 & - & 5.012e-19 & - & 7.614e-10 & -\\
     80  & 1.1e-04 &  4.741e-19 & - & 4.175e-19 & - & 6.885e-10 & -\\
\hline
\end{tabular}
\label{tab:feT10}
\end{table}

\begin{table}[H]
\caption{Errors and convergence orders for Backward Euler, final time of 10.0.}
\vspace{0.1in}
\centering
\begin{tabular}{|c|c|c|c| c| c| c| c|}
\hline
 $N+1$  & $\Delta t$ & $||e_h(\cdot)||_{L^2([0,1])}$ & Order  & $||e_h(\cdot)||_{L^\infty([0,1])}$ & Order& $||e_h(\cdot)||_h$& Order \\
 \hline            %   L_2         % O(L_2)% L_inf     % O(L_inf)%  H_1        %O(H_1) 
     10  & 0.1     & 1.183e-15 & - & 9.250e-16 & - & 3.441e-08 & -\\
     20  & 0.05    & 1.604e-16 & - & 1.266e-16 & - & 1.268e-08 & -\\
     40  & 0.025   & 2.967e-17 & - & 2.368e-17 & - & 5.441e-09 & -\\
     80  & 0.0125  & 8.485e-18 & - & 6.756e-18 & - & 2.901e-09 & -\\
\hline
\end{tabular}
\label{tab:beT10}
\end{table}

\begin{table}[H]
\caption{Errors and convergence orders for Crank-Nicolson, final time of 10.0.}
\vspace{0.1in}
\centering
\begin{tabular}{|c|c|c|c| c| c| c| c|}
\hline
 $N+1$  & $\Delta t$ &  $||e_h(\cdot)||_{L^2([0,1])}$ & Order  & $||e_h(\cdot)||_{L^\infty([0,1])}$ & Order& $||e_h(\cdot)||_h$& Order \\
 \hline           %   L_2         % O(L_2)% L_inf     % O(L_inf)%  H_1        %O(H_1) 
     10  & 0.1    & 4.558e-18 & - & 3.624e-18 & - & 2.134e-09 & -\\
     20  & 0.05   & 2.415e-18 & - & 1.932e-18 & - & 1.932e-09 & -\\
     40  & 0.025  & 6.413e-19 & - & 5.181e-19 & - & 8.008e-10 & -\\
     80  & 0.0125 & 3.650e-19 & - & 3.350e-19 & - & 6.041e-10 & -\\
\hline
\end{tabular}
\label{tab:cnT10}
\end{table}

The curves in Figure \ref{fig:jump} show that the 
numerical solution decays over time to the expected solution.
The time step size for the Forward Euler solves was chosen based
on the relation

\begin{equation}
  \Delta t \leq \frac{ (\Delta x)^2} {4 C_r^2}
\end{equation}

\noindent
where, the constant $C_r^2$ is bounded by

\begin{equation}
  C_r \leq \sqrt{3} r^2.
\end{equation}

\noindent
Since linear elements were used, $r = 1$.
Time step sizes for Backward Euler and Crank-Nicolson 
were refined on the same order as the mesh. 
In essence, when the number of elements 
doubled, the size of the time step halved.
For the shorter time period ($T=1.0$):
The Forward Euler scheme shows 
second order convergence in $L^\infty$ 
and $L^2$ norms but only
first order in $H^1$;
The Backward Euler scheme shows 
first order convergence in $L^\infty$ 
and $L^2$ norms but only
a convergence order of $0.5$ in $H^1$;
The Crank-Nicolson scheme shows 
second order convergence in $L^\infty$ 
and $L^2$ norms but only
first order in $H^1$;
Orders of convergence were not computed for
the longer time interval ($T=10.0$) 
as the solutions all converged to 
near machine precision.


The \texttt{Eigen} software package
was used as the linear solver for this project. 
The matrix inversion method used was LU with full pivoting.

\newpage
\appendix
\section{Source Code and Headers} \label{sec:code}

\subsection{fea\_hw6.cpp} \label{subsec:fea_hw6.cpp}
\lstinputlisting{/lore/clougj/Learning_Codes/FEA/HW6/src/fea_hw6.cpp}

\subsection{driver.cpp} \label{subsec:driver.cpp}
\lstinputlisting{/lore/clougj/Learning_Codes/FEA/HW6/src/driver.cpp}
\subsection{driver.hpp} \label{subsec:driver.hpp}
\lstinputlisting{/lore/clougj/Learning_Codes/FEA/HW6/src/driver.hpp}

\subsection{element.cpp} \label{subsec:element.cpp}
\lstinputlisting{/lore/clougj/Learning_Codes/FEA/HW6/src/element.cpp}
\subsection{element.hpp} \label{subsec:element.hpp}
\lstinputlisting{/lore/clougj/Learning_Codes/FEA/HW6/src/element.hpp}

\subsection{mesh.cpp} \label{subsec:mesh.cpp}
\lstinputlisting{/lore/clougj/Learning_Codes/FEA/HW6/src/mesh.cpp}
\subsection{mesh.hpp} \label{subsec:mesh.hpp}
\lstinputlisting{/lore/clougj/Learning_Codes/FEA/HW6/src/mesh.hpp}

\subsection{solution.cpp} \label{subsec:solution.cpp}
\lstinputlisting{/lore/clougj/Learning_Codes/FEA/HW6/src/solution.cpp}
\subsection{solution.hpp} \label{subsec:solution.hpp}
\lstinputlisting{/lore/clougj/Learning_Codes/FEA/HW6/src/solution.hpp}

\subsection{eig\_wrap.hpp} \label{subsec:eig_wrap.hpp}
\lstinputlisting{/lore/clougj/Learning_Codes/FEA/HW6/src/eig_wrap.hpp}

\subsection{CMakeLists.txt} \label{subsec:CMakeLists.txt}
\lstinputlisting{/lore/clougj/Learning_Codes/FEA/HW6/src/CMakeLists.txt}

\end{document}
