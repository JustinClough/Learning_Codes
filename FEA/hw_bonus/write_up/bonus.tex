\documentclass[a4paper, 12pt]{article}
\title{}
\usepackage[margin=0.75in]{geometry}
\usepackage{float}
\usepackage{subfigure}
\usepackage[justification=centering]{caption}
\usepackage{enumerate}
\usepackage{multirow}
\usepackage{listings}
\lstset{
    escapechar=`,
    language=C++,
    numbers=left,
    tabsize=2,
    prebreak=\raisebox{0ex}[0ex][0ex]{\ensuremath{\hookleftarrow}},
    frame=single,
    breaklines=true,
}
\usepackage{graphicx}
\graphicspath{ {./} }
\usepackage{nameref}
\usepackage{amsmath}
\usepackage{amssymb}
\usepackage{amsfonts}
\usepackage[linesnumbered,ruled]{algorithm2e}
\usepackage{tikz}
\usetikzlibrary{calc,patterns,decorations.pathmorphing,decorations.markings,positioning,automata}
\usepackage{pgfplots}
\usepackage{pgfplotstable}
\usepackage{makecell}
\usepackage{ulem}
\usepackage{verbatim}

\begin{document}

\section*{Bonus Problem: Pure Neumann Problem} \label{sec:intro}
The purpose of this exercise was to use the Finite Element Method (FEM) 
to approximate, and study, the solution to the problem presented in 
Eq. \ref{eq:StrongForm}.

\begin{align}
   -u_{xx} &= f               \label{eq:StrongForm} \\
u_x(0) = \alpha &, u_x(1) = \beta
\end{align}

\noindent
This equation was solved on the domain $x\in[0,1]$.
The given forcing function $f$ and boundary conditions 
$\alpha$, and $\beta$ meet the condition
shown in Equation \ref{f_cond}.

\begin{equation} \label{f_cond}
  \int_0^1 f dx = \alpha - \beta
\end{equation}

\noindent
As posed, this problem has the solution $U(x) + C$ where $C$ is 
any constant. 
Two methods are considered in this report to remove the ambiguity
of this constant:
(1) Prescribe a Dirichlet condition at a node;
(2) Prescribe a zero-average condition on the solution.

The problem domain was divided into a uniform and structured mesh.
The rule for determining element size is shown in the equation below.

\begin{equation}
  h =  \frac{ 1}{ N+1}
\end{equation}

\noindent
Here, $N$ is the number of degrees of freedom; $N+1$ is the number of elements.
Tests were conducted for $N= 5, 10, 20, ..., 640, 1280$.
All tests used the manufactured solution shown
in Equation \ref{manSolution}.

\begin{equation} \label{manSolution}
  u(x) = sin( \pi x) e^x - \frac{(1+e)\pi}{1+\pi^2}
\end{equation}

\noindent
This manufactured solution was chosen as it is smooth and 
has a zero average over the interval $x\in[0,1]$.
A computer code was written to evaluate the posed problem
with the two methods.

The Dirichlet condition used was the 
value of the manufactured solution evaluated at $x=0$.
Written out explicitly, $u(0) = b$ where

\begin{equation}
  b = - \frac{(1+e)\pi}{1+\pi^2}
\end{equation}

\noindent
It was implemented with the method discussed in lecture
and shown in previous homework assignments.
In practice as the solution is not known beforehand,
assigning the Dirichlet boundary condition as any known number
(or more conveniently, as zero) would lead to similar 
convergence order results.
E.g., choosing $b = 0$ instead would still 
lead to similar convergence rates and solve times.

The zero-average was imposed by enforcing

\begin{equation}
  \int_0^1 u dx = 0
\end{equation}

\noindent
To implement this constraint, first the base FE system 
was constructed:

\begin{equation} \label{original_system}
  K \underline{U} = \underline{F}
\end{equation}

\noindent
Next, the zero-average condition was discretized 
by approximating $u$ as $u^h$, the FE solution,
and using the shape functions in the elements:

\begin{align}
  \int_0^1 u dx &\approx \sum_{i=1}^N \int_{e_i} (u^h_{i,1} \phi_{i,1} + u^h_{i,2} \phi_{i,2}) dx \\
  \int_{e_i} (u^h_{i,1} \phi_{i,1} + u^h_{i,2} \phi_{i,2}) dx &= (u^h_{i,1} + u^h_{i,2}) \frac{h_i}{2} 
\end{align}

\noindent
where $e_i$ is the $i$th element, 
$u^h_{i,k}$ is the FE solution at the $i$th element's $k$th node, 
and
$\phi_{i,k}$ is the basis function of the $i$th element's $k$th node.
A vector $H$ was then defined as:

\begin{equation}
  \underline{H} = \frac{1}{2}
    \begin{bmatrix}
      h_1\\
      h_1 + h_2\\
      \vdots\\
      h_{i-1} + h_i\\
      \vdots\\
      h_{N-1} + h_N\\
      h_N
    \end{bmatrix}
\end{equation}

\noindent
This then allowed the zero average condition to be written as
$\underline{H} \cdot \underline{U} = 0$.
Adding this condition to Equation \ref{original_system}
then gives the block structure:

\begin{equation}
  \begin{bmatrix}
    K   \\
    H^T 
  \end{bmatrix}
  \underline{U} =
  \begin{bmatrix}
    \underline{F} \\
    0
  \end{bmatrix}
\end{equation}

\noindent
which has a new stiffness matrix that is $N+2 \times N+1$. 
To instead force this back to a square and symmetric system, 
the system below is used instead:

\begin{equation}
  \begin{bmatrix}
    K   & H \\
    H^T & \gamma
  \end{bmatrix}
  \begin{bmatrix}
    \underline{U}  \\
    c
  \end{bmatrix}
  =
  \begin{bmatrix}
    \underline{F} \\
    0
  \end{bmatrix}
\end{equation}

\noindent
with $\gamma = \sum_{i=1}^N H_i$ to also preserve
diagonal dominance. 
Note that since $K \underline{U} = \underline{F}$
is a valid solution, then $c=0$.

The results follow this section and the code itself 
is appended to this report.
Tabulated data, in the form of error norms, convergence order, 
assembly, and solve times are shown in
Tables \ref{tab:zdbc} and \ref{tab:zAve}.

\begin{table}[H]
\caption{ Error measures, convergence orders, and timing data for the assigned Dirichlet condition.}
\vspace{0.1in}
\centering
\begin{tabular}{|c|c|c|c|c|c|c|c|c|}
\hline
 $N+1$&  $||e_h(\cdot)||_{L^2(\Omega_c)}$ & Order & $||e_h(\cdot)||_{L^\infty([0,1])}$ & Order & $||e_h(\cdot)||_{H^1}$ & Order & Assembly & Solve \\
 \hline   % L_2         O(L_2)   L_inf       O(L_inf)   H_1       O(H_1)    assembly    solve
5    & 5.575e-02 & 0.000 & 9.803e-02 & 0.000 & 9.888e-01 & 0.000 & 3.17e-05 & 1.56e-05\\
10   & 1.390e-02 & 2.003 & 2.524e-02 & 1.957 & 4.918e-01 & 1.007 & 1.06e-05 & 1.55e-05\\
20   & 3.473e-03 & 2.000 & 6.355e-03 & 1.989 & 2.451e-01 & 1.004 & 1.89e-05 & 4.13e-05\\
40   & 8.681e-04 & 2.000 & 1.591e-03 & 1.997 & 1.223e-01 & 1.002 & 3.67e-05 & 1.75e-04\\
80   & 2.170e-04 & 2.000 & 3.979e-04 & 1.999 & 6.113e-02 & 1.001 & 7.16e-05 & 1.00e-03\\
160  & 5.425e-05 & 2.000 & 9.947e-05 & 2.000 & 3.055e-02 & 1.000 & 1.41e-04 & 6.60e-03\\
320  & 1.356e-05 & 2.000 & 2.486e-05 & 1.999 & 1.527e-02 & 1.000 & 2.85e-04 & 1.90e-02\\
640  & 3.391e-06 & 2.000 & 6.217e-06 & 1.999 & 7.635e-03 & 1.000 & 1.87e-04 & 1.22e-01\\
1280 & 8.477e-07 & 2.000 & 1.554e-06 & 1.999 & 3.817e-03 & 1.000 & 4.73e-04 & 1.08e+00\\
\hline
\end{tabular}
\label{tab:zdbc}
\end{table}

\begin{table}[H]
\caption{ Error measures, convergence orders, and timing data for the zero-average condition.}
\vspace{0.1in}
\centering
\begin{tabular}{|c|c|c|c|c|c|c|c|c|}
\hline
 $N+1$&  $||e_h(\cdot)||_{L^2(\Omega_c)}$ & Order & $||e_h(\cdot)||_{L^\infty([0,1])}$ & Order & $||e_h(\cdot)||_{H^1}$ & Order & Assembly & Solve \\
 \hline   % L_2         O(L_2)   L_inf       O(L_inf)   H_1       O(H_1)    assembly    solve
5    & 4.350e-02 & 0.000 & 5.861e-02 & 0.000 & 9.882e-01 & 0.000 & 2.94e-05 & 1.83e-05\\
10   & 1.090e-02 & 1.995 & 1.553e-02 & 1.916 & 4.917e-01 & 1.006 & 1.06e-05 & 1.75e-05\\
20   & 2.729e-03 & 1.998 & 4.038e-03 & 1.943 & 2.451e-01 & 1.004 & 1.94e-05 & 4.78e-05\\
40   & 6.824e-04 & 1.999 & 1.027e-03 & 1.974 & 1.223e-01 & 1.002 & 3.74e-05 & 1.93e-04\\
80   & 1.706e-04 & 1.999 & 2.590e-04 & 1.987 & 6.113e-02 & 1.001 & 7.32e-05 & 1.06e-03\\
160  & 4.265e-05 & 1.999 & 6.503e-05 & 1.994 & 3.055e-02 & 1.000 & 1.42e-04 & 6.87e-03\\
320  & 1.066e-05 & 1.999 & 1.629e-05 & 1.997 & 1.527e-02 & 1.000 & 2.88e-04 & 2.03e-02\\
640  & 2.665e-06 & 1.999 & 4.076e-06 & 1.998 & 7.635e-03 & 1.000 & 1.89e-04 & 1.27e-01\\
1280 & 6.664e-07 & 2.000 & 1.019e-06 & 1.999 & 3.817e-03 & 1.000 & 4.77e-04 & 1.11e+00\\
\hline
\end{tabular}
\label{tab:zAve}
\end{table}

From the calculated data in Tables \ref{tab:zdbc} and \ref{tab:zAve},
it appears that more nodes are needed for the zero-average method
to achieve a similar level of accuracy to the assigned Dirichlet condition.
Assembly and solve times for both methods are on par with one another.
Adjusting the $\gamma$ value to be in the range of zero to ten times 
the original appeared to have no noticeable effect on the values reported on.
The \texttt{Eigen} software package
was used as the linear solver for this project. 
The matrix inversion method used was LU with full pivoting.

\newpage
\appendix
\section{Source Code and Headers} \label{sec:code}

\subsection{fea\_bonus.cpp} \label{subsec:fea_bonus.cpp}
\lstinputlisting{/lore/clougj/Learning_Codes/FEA/hw_bonus/src/fea_bonus.cpp}

\subsection{driver.cpp} \label{subsec:driver.cpp}
\lstinputlisting{/lore/clougj/Learning_Codes/FEA/hw_bonus/src/driver.cpp}
\subsection{driver.hpp} \label{subsec:driver.hpp}
\lstinputlisting{/lore/clougj/Learning_Codes/FEA/hw_bonus/src/driver.hpp}

\subsection{element.cpp} \label{subsec:element.cpp}
\lstinputlisting{/lore/clougj/Learning_Codes/FEA/hw_bonus/src/element.cpp}
\subsection{element.hpp} \label{subsec:element.hpp}
\lstinputlisting{/lore/clougj/Learning_Codes/FEA/hw_bonus/src/element.hpp}

\subsection{mesh.cpp} \label{subsec:mesh.cpp}
\lstinputlisting{/lore/clougj/Learning_Codes/FEA/hw_bonus/src/mesh.cpp}
\subsection{mesh.hpp} \label{subsec:mesh.hpp}
\lstinputlisting{/lore/clougj/Learning_Codes/FEA/hw_bonus/src/mesh.hpp}

\subsection{solution.cpp} \label{subsec:solution.cpp}
\lstinputlisting{/lore/clougj/Learning_Codes/FEA/hw_bonus/src/solution.cpp}
\subsection{solution.hpp} \label{subsec:solution.hpp}
\lstinputlisting{/lore/clougj/Learning_Codes/FEA/hw_bonus/src/solution.hpp}

\subsection{printer.cpp} \label{subsec:printer.cpp}
\lstinputlisting{/lore/clougj/Learning_Codes/FEA/hw_bonus/src/printer.cpp}
\subsection{printer.hpp} \label{subsec:printer.hpp}
\lstinputlisting{/lore/clougj/Learning_Codes/FEA/hw_bonus/src/printer.hpp}

\subsection{eig\_wrap.hpp} \label{subsec:eig_wrap.hpp}
\lstinputlisting{/lore/clougj/Learning_Codes/FEA/hw_bonus/src/eig_wrap.hpp}

\subsection{CMakeLists.txt} \label{subsec:CMakeLists.txt}
\lstinputlisting{/lore/clougj/Learning_Codes/FEA/hw_bonus/src/CMakeLists.txt}

\end{document}
