\documentclass[a4paper, 12pt]{article}
\title{}
\usepackage{geometry}
\usepackage{float}
\usepackage{subfigure}
\usepackage[justification=centering]{caption}
\usepackage{enumerate}
\usepackage{multirow}
\usepackage{listings}
\lstset{
    escapechar=`,
    language=C++,
    numbers=left,
    tabsize=2,
    prebreak=\raisebox{0ex}[0ex][0ex]{\ensuremath{\hookleftarrow}},
    frame=single,
    breaklines=true,
}
\usepackage{graphicx}
\graphicspath{ {./} }
\usepackage{nameref}
\usepackage{amsmath}
\usepackage{amssymb}
\usepackage{amsfonts}
\usepackage[linesnumbered,ruled]{algorithm2e}
\usepackage{tikz}
\usetikzlibrary{calc,patterns,decorations.pathmorphing,decorations.markings,positioning,automata}
\usepackage{pgfplots}
\usepackage{pgfplotstable}
\usepackage{makecell}
\usepackage{ulem}
\usepackage{verbatim}

\begin{document}

\section*{Part II++: Implementation of Bonus 2}
\subsection{Introduction}\label{sec:intro}
The latter bonus question, meshing a domain with a curved 
boundary, was completed. 
The top boundary of domain $\Omega = [0,1] \times [0,1]$
was replaced with Equation \eqref{e:y_beta}.

\begin{equation} \label{e:y_beta}
Y_{top}(x) = \beta x ( 1 - x) +1
\end{equation}

In Equation \eqref{e:y_beta}, $\beta = \frac{1}{4}$. 
In essence, $\Omega_c = [0, 1] \times [0, Y_{top}]$ where the 
$c$ subscript represents the curved domain.
A mesh was made to approximate this geometry.
An example of this mesh is shown in Figure \ref{fig:mesh}.
Additionally, a perturbed mesh is shown 
in Figure \ref{fig:pMesh} and is overlaid with
the original mesh in Figure \ref{fig:overlap}.

\begin{figure}[H]
  \centering
  \includegraphics[width=11cm, height=10cm]{curved_mesh}
  \caption{ Mesh for domain with curved top boundary.}
  \label{fig:mesh}
\end{figure}

\begin{figure}[H]
  \centering
  \includegraphics[width=11cm, height=10cm]{perturbed_mesh}
  \caption{ Mesh for domain with curved top boundary 
            and perturbed node locations.}
  \label{fig:pMesh}
\end{figure}

\begin{figure}[H]
  \centering
  \includegraphics[width=11cm, height=10cm]{overlay_mesh}
  \caption{ Mesh for domain with curved top boundary 
            and perturbed node locations overlaid
            with original.}
  \label{fig:overlap}
\end{figure}

The $y$ node positions were determined by linearly interpolating the 
location of the top boundary with the location of the bottom boundary
for predetermined $x$ value. 
This is is shown in Algorithm \ref{al:create_nodes}.
The algorithm that perturbs the mesh of $\Omega_c$ is the same
as the that which perturbs the mesh of $\Omega$ except
for nodes on the top. 
For the nodes on the top boundary, first the $x$ value is 
perturbed, then the new $y$ value is calculated using 
Equation \eqref{e:y_beta}.

\begin{algorithm}[H]
  \underline{create\_nodes} $()$
  \BlankLine
  $x := 0.0$ \;
  $y := 0.0$ \;
  $index := 0$ \;
  $dx := 1.0 / (N+1.0)$\;
  \For{for i = 0 to (M-1)}
  {
    \For{for j = 0 to (M-1)}
    {
      $Y_{top} := \beta x (1-x) + 1$\;
      $dy := Y_{top} / (N+1.0)$\;
      $y := i * dy$\;

      $node\_matrix( index, 0) := x$\;
      $node\_matrix( index, 1) := y$\;

      $check\_boudary( index)$\;

      $x += dx$\;
      $index++$\;
    }
    $x := 0.0$\;
  }
  return;
  \caption{Creates the nodal locations for the mesh with 
           a curved top domain defined by $Y_{top}(x)$. 
           It assumes $M$ total nodes along each axis 
           and $N$ interior nodes along each axis.
           The function $check\_boundary(\cdot)$ checks
           which boundary the node is on based on its index
           and stores it in a separate data structure.}
  \label{al:create_nodes}
\end{algorithm}
\vspace{\baselineskip}



A convergence study was done on these meshes.

% TODO

\end{document}
