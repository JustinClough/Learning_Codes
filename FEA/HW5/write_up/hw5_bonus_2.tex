\documentclass[a4paper, 12pt]{article}
\title{}
\usepackage[margin=1.0in]{geometry}
\usepackage{float}
\usepackage{subfigure}
\usepackage[justification=centering]{caption}
\usepackage{enumerate}
\usepackage{multirow}
\usepackage{listings}
\lstset{
    escapechar=`,
    language=C++,
    numbers=left,
    tabsize=2,
    prebreak=\raisebox{0ex}[0ex][0ex]{\ensuremath{\hookleftarrow}},
    frame=single,
    breaklines=true,
}
\usepackage{graphicx}
\graphicspath{ {./} }
\usepackage{nameref}
\usepackage{amsmath}
\usepackage{amssymb}
\usepackage{amsfonts}
\usepackage[linesnumbered,ruled]{algorithm2e}
\usepackage{tikz}
\usetikzlibrary{calc,patterns,decorations.pathmorphing,decorations.markings,positioning,automata}
\usepackage{pgfplots}
\usepackage{pgfplotstable}
\usepackage{makecell}
\usepackage{ulem}
\usepackage{verbatim}

\begin{document}

\section*{Part II++: Implementation of Bonus 2}
\subsection{Introduction}\label{sec:intro}
The latter bonus question, meshing a domain with a curved 
boundary, was completed. 
The top boundary of domain $\Omega = [0,1] \times [0,1]$
was replaced with Equation \eqref{e:y_beta}.

\begin{equation} \label{e:y_beta}
Y_{top}(x) = \beta x ( 1 - x) +1
\end{equation}

In Equation \eqref{e:y_beta}, $\beta = \frac{1}{4}$. 
In essence, $\Omega_c = [0, 1] \times [0, Y_{top}]$ where the 
$c$ subscript represents the curved domain.
A mesh was made to approximate this geometry.
An example of this mesh is shown in Figure \ref{fig:mesh}.
Additionally, a perturbed mesh is shown 
in Figure \ref{fig:pMesh} and is overlaid with
the original mesh in Figure \ref{fig:overlap}.

\begin{figure}[H]
  \centering
  \includegraphics[width=11cm, height=10cm]{curved_mesh}
  \caption{ Mesh for domain with curved top boundary.}
  \label{fig:mesh}
\end{figure}

\begin{figure}[H]
  \centering
  \includegraphics[width=11cm, height=10cm]{perturbed_mesh}
  \caption{ Mesh for domain with curved top boundary 
            and perturbed node locations.}
  \label{fig:pMesh}
\end{figure}

\begin{figure}[H]
  \centering
  \includegraphics[width=11cm, height=10cm]{overlay_mesh}
  \caption{ Mesh for domain with curved top boundary 
            and perturbed node locations overlaid
            with original.}
  \label{fig:overlap}
\end{figure}

The $y$ node positions were determined by linearly interpolating the 
location of the top boundary with the location of the bottom boundary
for predetermined $x$ value. 
This is is shown in Algorithm \ref{al:create_nodes}.
The algorithm that perturbs the mesh of $\Omega_c$ is the same
as the that which perturbs the mesh of $\Omega$ except
for nodes on the top. 
For the nodes on the top boundary, first the $x$ value is 
perturbed, then the new $y$ value is calculated using 
Equation \eqref{e:y_beta}.

\begin{algorithm}[H]
  \underline{create\_nodes} $()$
  \BlankLine
  $x := 0.0$ \;
  $y := 0.0$ \;
  $index := 0$ \;
  $dx := 1.0 / (N+1.0)$\;
  \For{for i = 0 to (M-1)}
  {
    \For{for j = 0 to (M-1)}
    {
      $Y_{top} := \beta x (1-x) + 1$\;
      $dy := Y_{top} / (N+1.0)$\;
      $y := i * dy$\;

      $node\_matrix( index, 0) := x$\;
      $node\_matrix( index, 1) := y$\;

      $check\_boundary( index)$\;

      $x += dx$\;
      $index++$\;
    }
    $x := 0.0$\;
  }
  return;
  \caption{Creates the nodal locations for the mesh with 
           a curved top domain defined by $Y_{top}(x)$. 
           It assumes $M$ total nodes along each axis 
           and $N$ interior nodes along each axis.
           The function $check\_boundary(\cdot)$ checks
           which boundary the node is on based on its index
           and stores it in a separate data structure.}
  \label{al:create_nodes}
\end{algorithm}
\vspace{\baselineskip}


A convergence study was done on these meshes. 
Meshes of sizes $N+1=10, 20, 40, 80, 160$ were evaluated.
The exact solution was taken to be 
$u = y^3 + sin( 5 (x + y)) + 2e^x$.
The results for the unperturbed mesh are 
shown in Table \ref{tab:conv}. 
The results for the perturbed mesh are 
shown in Table \ref{tab:pert}. 


\begin{table}[!ht]
\caption{Errors for regular mesh for $\Omega_c$. Assembly and Solve times in seconds.}
\vspace{0.1in}
\centering
\begin{tabular}{|c|c|c|c|c|c|c|}
\hline
 $N+1$&  $||e_h(\cdot)||_{L^2(\Omega_c)}$ & Order & $||e_h(\cdot)||_{H^1}$ & Order & Assembly & Solve \\
 \hline
     10   & 7.774E-02 &  -      & 1.708E+00 &  -     & 2.621E-03 & 5.924E-05 \\
     20   & 2.003E-02 &  1.957  & 8.677E-01 &  0.977 & 1.858E-02 & 5.200E-05 \\
     40   & 5.048E-03 &  1.988  & 4.356E-01 &  0.994 & 9.830E-02 & 2.871E-04 \\
     80   & 1.264E-03 &  1.998  & 2.180E-01 &  0.998 & 1.978E+00 & 1.919E-03 \\
     160  & 3.163E-04 &  1.999  & 1.091E-01 &  0.999 & 3.937E+01 & 1.219E-02 \\
\hline
\end{tabular}
\label{tab:conv}
\end{table}

\begin{table}[!ht]
\caption{Errors for perturbed mesh for $\Omega_c$. Assembly and Solve times in seconds.}
\vspace{0.1in}
\centering
\begin{tabular}{|c|c|c|c|c|c|c|}
\hline
 $N+1$&  $||e_h(\cdot)||_{L^2(\Omega_c)}$ & Order & $||e_h(\cdot)||_{H^1}$ & Order & Assembly & Solve \\
 \hline
     10   & 7.852E-02 & -     & 1.714E+00 &  -     & 2.607E-03 & 5.517E-05 \\
     20   & 2.024E-02 & 1.956 & 8.718E-01 &  0.975 & 4.620E-03 & 5.095E-05 \\
     40   & 5.075E-03 & 1.996 & 4.363E-01 &  0.998 & 6.542E-02 & 2.180E-04 \\
     80   & 1.273E-03 & 1.995 & 2.184E-01 &  0.998 & 1.866E+00 & 1.898E-03 \\
     160  & 3.184E-04 & 1.999 & 1.093E-01 &  0.999 & 2.929E+01 & 1.220E-02 \\
\hline
\end{tabular}
\label{tab:pert}
\end{table}


As shown in Tables \ref{tab:conv} and \ref{tab:pert}, the method 
still converges at approximately the expected order: 
2 for the $L^2$ and 
1 for the $H^1$ norms.
A ten times larger $\beta$ value, $\beta=\frac{10}{4}$ was tried as an 
experiment.
The results for the unperturbed mesh are shown in Table \ref{t:b2}.

% If a large $\beta$ value was used, then this meshing method 
% would create more needle-like elements near the center. 
% These elements with higher aspect ratios would then
% cause issues with solution accuracy and convergence.

\begin{table}[!ht]
\caption{Errors for regular mesh for $\Omega_c$ with $\beta = \frac{10}{4}$. Assembly and Solve times in seconds.}
\vspace{0.1in}
\centering
\begin{tabular}{|c|c|c|c|c|c|c|}
\hline
 $N+1$&  $||e_h(\cdot)||_{L^2(\Omega_c)}$ & Order & $||e_h(\cdot)||_{H^1}$ & Order & Assembly & Solve \\
 \hline
     10   & 2.231E-01 &  -      & 3.424E+00 &  -     & 2.544E-03 & 5.839E-05 \\
     20   & 6.901E-02 &  1.693  & 1.869E+00 &  0.873 & 1.805E-02 & 5.259E-05 \\
     40   & 1.880E-02 &  1.876  & 9.628E-01 &  0.957 & 9.471E-02 & 2.880E-04 \\
     80   & 4.835E-03 &  1.959  & 4.855E-01 &  0.988 & 1.971E+00 & 1.920E-03 \\
     160  & 1.219E-03 &  1.988  & 2.433E-01 &  0.997 & 2.961E+01 & 1.221E-02 \\
\hline
\end{tabular}
\label{t:b2}
\end{table}

As shown in Table \ref{t:b2}, a larger $\beta$ value 
requires more elements to both converge on the solution
and achieve the theoretical convergence rate.
If an even larger $\beta$ value was used, then this meshing method 
would create more needle-like elements.
These elements with higher aspect ratios would then 
continue to cause issues with solution accuracy and convergence.

The code used is not appended to this report in 
an effort to save paper.
Instead, it can be viewed online at
\texttt{www.github.com/JustinClough/Learning\_Codes/tree/master/FEA}.


\end{document}
