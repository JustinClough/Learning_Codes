\documentclass[a4paper, 12pt]{article}
\title{}
\usepackage{geometry}
\usepackage{float}
\usepackage{subfigure}
\usepackage[justification=centering]{caption}
\usepackage{enumerate}
\usepackage{multirow}
\usepackage{listings}
\lstset{
    escapechar=`,
    language=C++,
    numbers=left,
    tabsize=2,
    prebreak=\raisebox{0ex}[0ex][0ex]{\ensuremath{\hookleftarrow}},
    frame=single,
    breaklines=true,
}
\usepackage{graphicx}
\graphicspath{ {./} }
\usepackage{nameref}
\usepackage{amsmath}
\usepackage{amssymb}
\usepackage{amsfonts}
\usepackage[linesnumbered,ruled]{algorithm2e}
\usepackage{tikz}
\usetikzlibrary{calc,patterns,decorations.pathmorphing,decorations.markings,positioning,automata}
\usepackage{pgfplots}
\usepackage{pgfplotstable}
\usepackage{makecell}
\usepackage{ulem}
\usepackage{verbatim}

\begin{document}

\section*{Part II++: Implementation of Bonus 2}
\subsection{Introduction}\label{sec:intro}
The latter bonus question, meshing a domain with a curved 
boundary, was completed. 
The top boundary of domain $\Omega = [0,1] \times [0,1]$
was replaced with Equation \eqref{e:y_beta}.

\begin{equation} \label{e:y_beta}
y(x) = \beta x ( 1 - x) +1
\end{equation}

In Equation \eqref{e:y_beta}, $\beta = \frac{1}{4}$. 
A mesh was made to approximate this geometry.
An example of this mesh is shown in Figure \ref{fig:mesh}.
Additionally, a perturbed mesh is shown 
in Figure \ref{fig:pMesh} and is overlaid with
the original mesh in Figure \ref{fig:overlap}.

\begin{figure}[H]
  \centering
  \includegraphics[width=11cm, height=10cm]{curved_mesh}
  \caption{ Mesh for domain with curved top boundary.}
  \label{fig:mesh}
\end{figure}

\begin{figure}[H]
  \centering
  \includegraphics[width=11cm, height=10cm]{perturbed_mesh}
  \caption{ Mesh for domain with curved top boundary 
            and perturbed node locations.}
  \label{fig:pMesh}
\end{figure}

\begin{figure}[H]
  \centering
  \includegraphics[width=11cm, height=10cm]{overlay_mesh}
  \caption{ Mesh for domain with curved top boundary 
            and perturbed node locations overlaid
            with original.}
  \label{fig:overlap}
\end{figure}


% TODO

\end{document}
