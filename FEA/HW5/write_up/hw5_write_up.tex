\documentclass[a4paper, 12pt]{article}
\title{}
\usepackage{geometry}
\usepackage{float}
\usepackage{subfigure}
\usepackage[justification=centering]{caption}
\usepackage{enumerate}
\usepackage{multirow}
\usepackage{listings}
\lstset{
    escapechar=`,
    language=C++,
    numbers=left,
    tabsize=2,
    prebreak=\raisebox{0ex}[0ex][0ex]{\ensuremath{\hookleftarrow}},
    frame=single,
    breaklines=true,
}
\usepackage{graphicx}
\graphicspath{ {./} }
\usepackage{nameref}
\usepackage{amsmath}
\usepackage{amssymb}
\usepackage{amsfonts}
\usepackage[linesnumbered,ruled]{algorithm2e}
\usepackage{tikz}
\usetikzlibrary{calc,patterns,decorations.pathmorphing,decorations.markings,positioning,automata}
\usepackage{pgfplots}
\usepackage{pgfplotstable}
\usepackage{makecell}
\usepackage{ulem}
\usepackage{verbatim}

\begin{document}


\section*{Part II: Implementation} \label{sec:intro}
The purpose of this exercise was to use the Finite Element Method (FEM)
to solve the Boundary Value Problem (BVP) shown in Equations \ref{eq:wfStart}
to \ref{eq:wfEnd}:


\begin{align}
a( u, v) &= F(v) \quad \forall v \in H_0^1(\Omega) \label{eq:wfStart} \\
a( u, v) &= \int_{\Omega} p( x, y) \nabla u( x, y)  \cdot \nabla v( x, y) + q( x, y) u( x, y) v( x, y) dx dy \\
F(v)     &= \int_{\Omega} f( x, y) v( x, y) dxdy \label{eq:wfEnd}
\end{align}

\noindent
where $p(x,y)$, $u(x,y)$, and $q(x,y)$ were given functions on $\Omega$.
The boundary of $\Omega$ is $\partial \Omega$ and the boundary
information, $\alpha( x, y)$ was also given.
The solution is $u \in V = \{ v\in H^1(\Omega):v\rvert_{\partial \Omega} = \alpha( x, y)\}$
which satisfies Eq. \ref{eq:wfStart}. 
The domain was tessellated with triangular elements;
a typical element is denoted by $K$ and the estimated 
domain and boundary as $\Omega_h$ and $\partial \Omega_h$, respectively.
The FEM was then used to recover $u_h$ from Eq. \ref{eq:fem}.

\begin{equation}
a( u_h, v) = F( v) \quad \forall v \in V_{h,0}^1  \label{eq:fem}
\end{equation}

\noindent
The test space was defined as
$V_{h,0}^1 = \{v \in H_0^1(\Omega): v\rvert_{K} \in P^1(K), \forall K \in \Omega_h\}$.
The domain considered was $\Omega = [0, 1] \times [0,1]$ with 
parameters $p(x,y) = 3$ and $q(x,y) = 2$.  
Four test cases, outlined in \ref{t:cases}, were considered.


\begin{table}[!ht]
\caption{ Case numbers and exact solutions.}
\vspace{0.1in}
\centering
\begin{tabular}{ |c|c|}
  \hline
  Case Number  & $u$ \\
  \hline
  1            &  $1$ \\
  \hline
  2            &  $x$ \\
  \hline
  3            &  $y$ \\
  \hline
  4            &  $y^3 + sin( 5(x+y)) + 2e^x$ \\
  \hline
\end{tabular}
\label{t:cases}
\end{table}

\noindent
For each case, the $L^2$ and $H^1$ norms of the error were measured. 
Additionally, the mesh was refined to determine the order of 
error convergence.


The \texttt{Eigen} software package
was used as the linear solver for this project. 
The \texttt{Sparse LU} solver was used to 
solve the linear system.

\newpage
\appendix
\section{Source Code and Headers} \label{sec:code}

\subsection{fea\_hw5.cpp} \label{subsec:fea_hw5.cpp}
\lstinputlisting{/lore/clougj/Learning_Codes/FEA/HW5/src/fea_hw5.cpp}

\subsection{driver.cpp} \label{subsec:driver.cpp}
\lstinputlisting{/lore/clougj/Learning_Codes/FEA/HW5/src/driver.cpp}
\subsection{driver.hpp} \label{subsec:driver.hpp}
\lstinputlisting{/lore/clougj/Learning_Codes/FEA/HW5/src/driver.hpp}

\subsection{mesh.cpp} \label{subsec:mesh.cpp}
\lstinputlisting{/lore/clougj/Learning_Codes/FEA/HW5/src/mesh.cpp}
\subsection{mesh.hpp} \label{subsec:mesh.hpp}
\lstinputlisting{/lore/clougj/Learning_Codes/FEA/HW5/src/mesh.hpp}

\subsection{solution.cpp} \label{subsec:solution.cpp}
\lstinputlisting{/lore/clougj/Learning_Codes/FEA/HW5/src/solution.cpp}
\subsection{solution.hpp} \label{subsec:solution.hpp}
\lstinputlisting{/lore/clougj/Learning_Codes/FEA/HW5/src/solution.hpp}

\subsection{CMakeLists.txt} \label{subsec:CMakeLists.txt}
\lstinputlisting{/lore/clougj/Learning_Codes/FEA/HW5/src/CMakeLists.txt}

\subsection{eig\_wrap.hpp} \label{subsec:eig_wrap.hpp}
\lstinputlisting{/lore/clougj/Learning_Codes/FEA/HW5/src/eig_wrap.hpp}

\end{document}
