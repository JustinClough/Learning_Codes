\documentclass[a4paper, 12pt]{article}
\title{}
\usepackage{geometry}
\usepackage{float}
\usepackage{subfigure}
\usepackage[justification=centering]{caption}
\usepackage{enumerate}
\usepackage{multirow}
\usepackage{listings}
\lstset{
    escapechar=`,
    language=C++,
    numbers=left,
    tabsize=2,
    prebreak=\raisebox{0ex}[0ex][0ex]{\ensuremath{\hookleftarrow}},
    frame=single,
    breaklines=true,
}
\usepackage{graphicx}
\graphicspath{ {./} }
\usepackage{nameref}
\usepackage{amsmath}
\usepackage{amssymb}
\usepackage{amsfonts}
\usepackage[linesnumbered,ruled]{algorithm2e}
\usepackage{tikz}
\usetikzlibrary{calc,patterns,decorations.pathmorphing,decorations.markings,positioning,automata}
\usepackage{pgfplots}
\usepackage{pgfplotstable}
\usepackage{makecell}
\usepackage{ulem}
\usepackage{verbatim}

\begin{document}


\section*{Part II: Implementation} \label{sec:intro}
The purpose of this exercise was to use the Finite Element Method (FEM)
to solve the Boundary Value Problem (BVP) shown in Equations \ref{eq:wfStart}
to \ref{eq:wfEnd}.


\begin{align}
a( u, v) &= F(v) \forall v \in H_0^1(\Omega) \label{eq:wfStart} \\
a( u, v) = \int_{\Omega} p \nabla u \cdot \nabla v + q u v dx dy                    
\end{align}


The \texttt{Eigen} software package
was used as the linear solver for this project. 
The \texttt{Sparse LU} solver was used to 
solve the linear system.

\newpage
\appendix
\section{Source Code and Headers} \label{sec:code}

\subsection{fea\_hw5.cpp} \label{subsec:fea_hw5.cpp}
\lstinputlisting{/lore/clougj/Learning_Codes/FEA/HW5/src/fea_hw5.cpp}

\subsection{driver.cpp} \label{subsec:driver.cpp}
\lstinputlisting{/lore/clougj/Learning_Codes/FEA/HW5/src/driver.cpp}
\subsection{driver.hpp} \label{subsec:driver.hpp}
\lstinputlisting{/lore/clougj/Learning_Codes/FEA/HW5/src/driver.hpp}

\subsection{mesh.cpp} \label{subsec:mesh.cpp}
\lstinputlisting{/lore/clougj/Learning_Codes/FEA/HW5/src/mesh.cpp}
\subsection{mesh.hpp} \label{subsec:mesh.hpp}
\lstinputlisting{/lore/clougj/Learning_Codes/FEA/HW5/src/mesh.hpp}

\subsection{solution.cpp} \label{subsec:solution.cpp}
\lstinputlisting{/lore/clougj/Learning_Codes/FEA/HW5/src/solution.cpp}
\subsection{solution.hpp} \label{subsec:solution.hpp}
\lstinputlisting{/lore/clougj/Learning_Codes/FEA/HW5/src/solution.hpp}

\subsection{CMakeLists.txt} \label{subsec:CMakeLists.txt}
\lstinputlisting{/lore/clougj/Learning_Codes/FEA/HW5/src/CMakeLists.txt}

\subsection{eig\_wrap.hpp} \label{subsec:eig_wrap.hpp}
\lstinputlisting{/lore/clougj/Learning_Codes/FEA/HW5/src/eig_wrap.hpp}

\end{document}
