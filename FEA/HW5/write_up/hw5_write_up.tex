\documentclass[a4paper, 12pt]{article}
\title{}
\usepackage{geometry}
\usepackage{float}
\usepackage{subfigure}
\usepackage[justification=centering]{caption}
\usepackage{enumerate}
\usepackage{multirow}
\usepackage{listings}
\lstset{
    escapechar=`,
    language=C++,
    numbers=left,
    tabsize=2,
    prebreak=\raisebox{0ex}[0ex][0ex]{\ensuremath{\hookleftarrow}},
    frame=single,
    breaklines=true,
}
\usepackage{graphicx}
\graphicspath{ {./} }
\usepackage{nameref}
\usepackage{amsmath}
\usepackage{amssymb}
\usepackage{amsfonts}
\usepackage[linesnumbered,ruled]{algorithm2e}
\usepackage{tikz}
\usetikzlibrary{calc,patterns,decorations.pathmorphing,decorations.markings,positioning,automata}
\usepackage{pgfplots}
\usepackage{pgfplotstable}
\usepackage{makecell}
\usepackage{ulem}
\usepackage{verbatim}

\begin{document}


\section*{Part II: Implementation (reattempted)} \label{sec:intro}



The \texttt{Eigen} software package
was used as the linear solver for this project. 
The matrix inversion method used was Householder-QR with pivoting.

\begin{comment}
\newpage
\appendix
\section{Source Code and Headers} \label{sec:code}

\subsection{fea\_hw5.cpp} \label{subsec:fea_hw5.cpp}
\lstinputlisting{/lore/clougj/Learning_Codes/FEA/HW5/src/fea_hw5.cpp}

\subsection{driver.cpp} \label{subsec:driver.cpp}
\lstinputlisting{/lore/clougj/Learning_Codes/FEA/HW5/src/driver.cpp}
\subsection{driver.hpp} \label{subsec:driver.hpp}
\lstinputlisting{/lore/clougj/Learning_Codes/FEA/HW5/src/driver.hpp}

\end{document}
