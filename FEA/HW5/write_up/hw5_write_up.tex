\documentclass[a4paper, 12pt]{article}
\title{}
\usepackage{geometry}
\usepackage{float}
\usepackage{subfigure}
\usepackage[justification=centering]{caption}
\usepackage{enumerate}
\usepackage{multirow}
\usepackage{listings}
\lstset{
    escapechar=`,
    language=C++,
    numbers=left,
    tabsize=2,
    prebreak=\raisebox{0ex}[0ex][0ex]{\ensuremath{\hookleftarrow}},
    frame=single,
    breaklines=true,
}
\usepackage{graphicx}
\graphicspath{ {./} }
\usepackage{nameref}
\usepackage{amsmath}
\usepackage{amssymb}
\usepackage{amsfonts}
\usepackage[linesnumbered,ruled]{algorithm2e}
\usepackage{tikz}
\usetikzlibrary{calc,patterns,decorations.pathmorphing,decorations.markings,positioning,automata}
\usepackage{pgfplots}
\usepackage{pgfplotstable}
\usepackage{makecell}
\usepackage{ulem}
\usepackage{verbatim}

\begin{document}


\section*{Part II: Implementation} 
\subsection{Introduction}\label{sec:intro}
The purpose of this exercise was to use the Finite Element Method (FEM)
to solve the Boundary Value Problem (BVP) shown in Equations \ref{eq:wfStart}
to \ref{eq:wfEnd}:


\begin{align}
a( u, v) &= F(v) \quad \forall v \in H_0^1(\Omega) \label{eq:wfStart} \\
a( u, v) &= \int_{\Omega} p( x, y) \nabla u( x, y)  \cdot \nabla v( x, y) + q( x, y) u( x, y) v( x, y) dx dy \\
F(v)     &= \int_{\Omega} f( x, y) v( x, y) dxdy \label{eq:wfEnd}
\end{align}

\noindent
where $p(x,y)$, $u(x,y)$, and $q(x,y)$ were given functions on $\Omega$.
The boundary of $\Omega$ is $\partial \Omega$ and the boundary
information, $\alpha( x, y)$ was also given.
The solution is $u \in V = \{ v\in H^1(\Omega):v\rvert_{\partial \Omega} = \alpha( x, y)\}$
which satisfies Eq. \ref{eq:wfStart}. 
The domain was tessellated with triangular elements;
a typical element is denoted by $K$ and the estimated 
domain and boundary as $\Omega_h$ and $\partial \Omega_h$, respectively.
The FEM was then used to recover $u_h$ from Eq. \ref{eq:fem}.

\begin{equation}
a( u_h, v) = F( v) \quad \forall v \in V_{h,0}^1  \label{eq:fem}
\end{equation}

\noindent
The test space was defined as
$V_{h,0}^1 = \{v \in H_0^1(\Omega): v\rvert_{K} \in P^1(K), \forall K \in \Omega_h\}$.
The domain considered was $\Omega = [0, 1] \times [0,1]$ with 
parameters $p(x,y) = 3$ and $q(x,y) = 2$.
It was approximated by $\Omega_h$ which had $N+2$ nodes
on each domain boundary for a total of $(N+2)^2$ nodes.

\subsection{Code Overview} \label{ss:codeOverview}
The code followed the following major steps to 
construct and solve the FEM problem and solution:

\begin{enumerate}
\item Construct a mesh. This first involved calculating node location
      and then assembling the nodes into elements. Additional data,
      such as element areas, was constructed at this steps.
\item Assemble the linear algebra system. This was done by populating a
      sparse matrix (the stiffness matrix) and dense vector (the forcing
      vector). To populate these objects, the elemental contributions
      were summed into the global system. The elemental contributions
      were approximated using 3\textsuperscript{rd} order numerical
      quadrature. \label{i:assemble}
\item Apply boundary conditions. This was done by removing
      all non-diagonal entries in the stiffness matrix for each row 
      corresponding to each node on the boundary. The given boundary
      information, $\alpha$, was then multiplied by the diagonal value
      which replaced that row's forcing vector term. For example, if 
      node $n$ at $(X, Y)$ was on the boundary, then the $n$th row
      of the stiffness matrix was cleared except for the 
      diagonal value, $k_{n,n}$. The $n$th term in the forcing vector,
      $F_n$, was then assigned $k_{n,n} \alpha( X, Y)$.
\item Solve the system. Sparse LU factorization was then used 
      to solve the linear system.
\item Compute errors. The $L^2$ and $H^1$ errors were estimated 
      by summing the elemental error contributions. The elemental
      error contributions were approximated using 3\textsuperscript{rd}
      order numerical quadrature. \label{i:errors}
\item Perturb mesh and repeat. The mesh was perturbed and the above steps
      from \ref{i:assemble} to \ref{i:errors} was repeated. 
      The mesh was perturbed by adjusting each node a random amount in 
      both the $x$ and $y$ directions so long as the boundary was 
      left unperturbed. The random amount had a maximum amount of 
      $\frac{1}{20}$ \textsuperscript{th} of the original node spacing.
\end{enumerate}

\subsection{Testing and Results} \label{ss:testAndResults}
Four test cases, outlined in \ref{t:cases}, were considered.

\begin{table}[!ht]
\caption{ Case numbers and exact solutions.}
\vspace{0.1in}
\centering
\begin{tabular}{ |c|c|}
  \hline
  Case Number  & $u$ \\
  \hline
  1            &  $1$ \\
  \hline
  2            &  $x$ \\
  \hline
  3            &  $y$ \\
  \hline
  4            &  $y^3 + sin( 5(x+y)) + 2e^x$ \\
  \hline
\end{tabular}
\label{t:cases}
\end{table}

\noindent
For each case, the $L^2$ and $H^1$ norms of the error were measured. 
Additionally, the mesh was refined to determine the order of 
error convergence. For cases 1, 2, and 3 meshes of size $N+1 = 20, 40$
were considered. For case 4, meshes of size $N+1 = 10, 20, 40, 80, 160$.
The results for case 1 are shown in Tables \ref{tab:C1o}
and \ref{tab:C1p}.

\begin{table}[!ht]
\caption{Errors for regular mesh for Case 1.}
\vspace{0.1in}
\centering
\begin{tabular}{|c|c|c|}
\hline
 $N+1$&  $||e_h(\cdot)||_{L^2([0,1])}$ & $||e_h(\cdot)||_H^1$ \\
 \hline
     20  & 4.063E-15 & 2.217E-14 \\
     40  & 2.684E-14 & 1.313E-13 \\
\hline
\end{tabular}
\label{tab:C1o}
\end{table}

\begin{table}[!ht]
\caption{Errors for perturbed mesh for Case 1.}
\vspace{0.1in}
\centering
\begin{tabular}{|c|c|c|}
\hline
 $N+1$  & $||e_h(\cdot)||_{L^2([0,1])}$ & $||e_h(\cdot)||_H^1$ \\
 \hline
     20  & 7.780E-16 & 1.057E-14 \\
     40  & 3.401E-15 & 2.695E-14 \\
\hline
\end{tabular}
\label{tab:C1p}
\end{table}

The results for case 2 are shown in Tables \ref{tab:C2o}
and \ref{tab:C2p}.

\begin{table}[!ht]
\caption{Errors for regular mesh for Case 2.}
\vspace{0.1in}
\centering
\begin{tabular}{|c|c|c|}
\hline
 $N+1$&  $||e_h(\cdot)||_{L^2([0,1])}$ & $||e_h(\cdot)||_H^1$ \\
 \hline
     20  & 2.196E-15 & 1.240E-14 \\
     40  & 1.197E-14 & 5.959E-14 \\
\hline
\end{tabular}
\label{tab:C2o}
\end{table}

\begin{table}[!ht]
\caption{Errors for perturbed mesh for Case 2.}
\vspace{0.1in}
\centering
\begin{tabular}{|c|c|c|}
\hline
 $N+1$  & $||e_h(\cdot)||_{L^2([0,1])}$ & $||e_h(\cdot)||_H^1$ \\
 \hline
     20  & 9.345E-16 & 6.721E-14 \\
     40  & 9.639E-16 & 1.346E-14 \\
\hline
\end{tabular}
\label{tab:C2p}
\end{table}

The results for case 3 are shown in Tables \ref{tab:C3o}
and \ref{tab:C3p}.

\begin{table}[!ht]
\caption{Errors for regular mesh for Case 3.}
\vspace{0.1in}
\centering
\begin{tabular}{|c|c|c|}
\hline
 $N+1$&  $||e_h(\cdot)||_{L^2([0,1])}$ & $||e_h(\cdot)||_H^1$ \\
 \hline
     20  & 2.404E-15 & 1.345E-14 \\
     40  & 1.272E-14 & 6.425E-14 \\
\hline
\end{tabular}
\label{tab:C3o}
\end{table}

\begin{table}[!ht]
\caption{Errors for perturbed mesh for Case 3.}
\vspace{0.1in}
\centering
\begin{tabular}{|c|c|c|}
\hline
 $N+1$  & $||e_h(\cdot)||_{L^2([0,1])}$ & $||e_h(\cdot)||_H^1$ \\
 \hline
     20  & 3.927E-16 & 6.095E-15 \\
     40  & 1.415E-16 & 1.487E-14 \\
\hline
\end{tabular}
\label{tab:C3p}
\end{table}

The results for case 4 are shown in Tables \ref{tab:C4o}
and \ref{tab:C4p}. 

\begin{table}[!ht]
\caption{Errors for regular mesh for Case 4. Assembly and Solve times in seconds.}
\vspace{0.1in}
\centering
\begin{tabular}{|c|c|c|c|c|c|c|}
\hline
 $N+1$&  $||e_h(\cdot)||_{L^2([0,1])}$ & Order & $||e_h(\cdot)||_H^1$ & Order & Assembly & Solve \\
 \hline
     10   & 7.136E-02 &  -      & 1.619E+00 &  -     & 2.512E-03 & 5.481E-05 \\
     20   & 1.831E-02 &  1.962  & 8.211E-01 &  0.979 & 1.658E-02 & 5.187E-05 \\
     40   & 4.607E-03 &  1.991  & 4.120E-01 &  0.995 & 9.366E-02 & 2.842E-04 \\
     80   & 1.154E-03 &  1.997  & 2.062E-01 &  1.000 & 1.957E+00 & 1.890E-03 \\
     160  & 2.885E-04 &  2.000  & 1.031E-01 &  1.000 & 3.907E+01 & 1.211E-02 \\
\hline
\end{tabular}
\label{tab:C4o}
\end{table}

\begin{table}[!ht]
\caption{Errors for perturbed mesh for Case 4. Assembly and Solve times in seconds.}
\vspace{0.1in}
\centering
\begin{tabular}{|c|c|c|c|c|c|c|}
\hline
 $N+1$&  $||e_h(\cdot)||_{L^2([0,1])}$ & Order & $||e_h(\cdot)||_H^1$ & Order & Assembly & Solve \\
 \hline
     10   & 7.153E-02 & -     & 1.619E+00 &  -     & 2.561E-03 & 6.031E-05 \\
     20   & 1.841E-02 & 1.958 & 8.225E-01 &  0.977 & 4.797E-03 & 5.071E-05 \\
     40   & 4.638E-03 & 1.989 & 4.127E-01 &  0.995 & 6.699E-02 & 2.833E-04 \\
     80   & 1.162E-03 & 1.997 & 2.067E-01 &  0.998 & 1.842E+00 & 1.881E-03 \\
     160  & 2.908E-04 & 1.999 & 1.033E-01 &  1.000 & 2.904E+01 & 1.205E-02 \\
\hline
\end{tabular}
\label{tab:C4p}
\end{table}


The \texttt{Eigen} software package
was used as the linear solver for this project. 
The \texttt{Sparse LU} solver was used to 
solve the linear system.

\newpage
\appendix
\section{Source Code and Headers} \label{sec:code}

\subsection{fea\_hw5.cpp} \label{subsec:fea_hw5.cpp}
\lstinputlisting{/lore/clougj/Learning_Codes/FEA/HW5/src/fea_hw5.cpp}

\subsection{driver.cpp} \label{subsec:driver.cpp}
\lstinputlisting{/lore/clougj/Learning_Codes/FEA/HW5/src/driver.cpp}
\subsection{driver.hpp} \label{subsec:driver.hpp}
\lstinputlisting{/lore/clougj/Learning_Codes/FEA/HW5/src/driver.hpp}

\subsection{mesh.cpp} \label{subsec:mesh.cpp}
\lstinputlisting{/lore/clougj/Learning_Codes/FEA/HW5/src/mesh.cpp}
\subsection{mesh.hpp} \label{subsec:mesh.hpp}
\lstinputlisting{/lore/clougj/Learning_Codes/FEA/HW5/src/mesh.hpp}

\subsection{solution.cpp} \label{subsec:solution.cpp}
\lstinputlisting{/lore/clougj/Learning_Codes/FEA/HW5/src/solution.cpp}
\subsection{solution.hpp} \label{subsec:solution.hpp}
\lstinputlisting{/lore/clougj/Learning_Codes/FEA/HW5/src/solution.hpp}

\subsection{CMakeLists.txt} \label{subsec:CMakeLists.txt}
\lstinputlisting{/lore/clougj/Learning_Codes/FEA/HW5/src/CMakeLists.txt}

\subsection{eig\_wrap.hpp} \label{subsec:eig_wrap.hpp}
\lstinputlisting{/lore/clougj/Learning_Codes/FEA/HW5/src/eig_wrap.hpp}

\end{document}
