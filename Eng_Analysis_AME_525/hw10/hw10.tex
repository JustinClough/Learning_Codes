\documentclass[11pt]{homework}

\newcommand{\hwname}{Justin L. Clough}
\newcommand{\hwemail}{jlclough@usc.edu}
\newcommand{\hwtype}{Homework}
\newcommand{\hwnum}{10}
\newcommand{\hwclass}{AME: 525}

\usepackage{amsmath}
 \usepackage{delarray}
\usepackage{physics}

\begin{document}
\maketitle

\question*{\S 19.2 page 666: 2, 6, 10, 12}
\emph{Find:}
\newline
In each of Problems 2, 6, 10, and 12, 
find $u$ and $v$ so that 
$f=u+iv$, determine all points $(x,y)$
at which the Cauchy-Riemann 
equations hold and determine all 
$z$ at which $f$ is differentiable.
Familiar facts about continuity of 
real-valued functions of two real variables
can be assumed. 

2. 
\begin{equation*}
  f(z) = z^2 - iz
\end{equation*}

6. 
\begin{equation*}
  f(z) = z + \text{ Im}(z)
\end{equation*}

10.
\begin{equation*}
  f(z) = iz + |z|
\end{equation*}

12.
\begin{equation*}
  f(z) = \frac{z-i}{z+i}
\end{equation*}

\emph{Solution:}
\newline
2.
Let $z = x+yi$ and $x,y \in \mathcal{R}$.
\begin{align*}
  f(z) &= z^2 - iz \\
       &= (x+yi)^2 - i(x+iy) \\
       &= (x^2 - y^2 +y) + i ( 2xy - x) \\
      \\
\Rightarrow
  u(x,y) &=x^2 - y^2 +y  \\
\Rightarrow
  v(x,y) &=2xy - x 
\end{align*}

The partial derivatives are then:
\begin{align*}
  \pdv{u}{x} &= 2x    & \pdv{u}{y} &= -2y +1 \\
  \pdv{v}{x} &= 2y-1  & \pdv{v}{y} &= 2x
\end{align*}

Since $\pdv{u}{x}=\pdv{v}{y}$ and $\pdv{u}{y}=-\pdv{v}{x}$ 
$\forall z \in \mathcal{C}$,
then the function is differentiable $ \forall z \in \mathcal{C}$.

6. 
Let $z = x+yi$ and $x,y \in \mathcal{R}$.
\begin{align*}
  f(z) &= z + \text{ Im}(z) \\
   &= (x+iy) + (y) \\
   &= (x+y) + i(y)
      \\
\Rightarrow
  u(x,y) &= x+y \\
\Rightarrow
  v(x,y) &=y
\end{align*}

The partial derivatives are then:
\begin{align*}
  \pdv{u}{x} &= 1    & \pdv{u}{y} &= 1 \\
  \pdv{v}{x} &= 0  & \pdv{v}{y} &= 1 
\end{align*}

Since the Cauchy-Riemann equations do not 
hold for any $z$, then the function
is not differentiable.

10.
Let $z = x+yi$ and $x,y \in \mathcal{R}$.
\begin{align*}
  f(z) &= iz + |z| \\
  &= i(x+iy) + \sqrt{x^2 + y^2} \\
  &= (\sqrt{x^2 + y^2} -y) + i(x) \\
      \\
\Rightarrow
  u(x,y) &=\sqrt{x^2 + y^2} -y  \\
\Rightarrow
  v(x,y) &=x
\end{align*}

The partial derivatives are then:
\begin{align*}
  \pdv{u}{x} &= \frac{x}{\sqrt{x^2 +y^2}}    & \pdv{u}{y} &= \frac{y}{\sqrt{x^2+y^2}} - 1 \\
  \pdv{v}{x} &= 1  & \pdv{v}{y} &= 0
\end{align*}

Let $S_i$ be the solution spaces for equations
$i=1,2$.
For the Cauchy-Riemann equations to hold:
\begin{align*}
  \pdv{u}{x} &= \pdv{v}{y} \\
  \frac{x}{\sqrt{x^2+y^2}} &= 0 \quad \forall (x,y)\in S_1=\{(x,y)|x=0,y\neq0\} \\
  \pdv{u}{y} &= -\pdv{v}{x} \\
  \frac{y}{\sqrt{x^2+y^2}} -1 &= -1  \quad \forall (x,y)\in S_2=\{(x,y)| x\neq 0,y=0\}
\end{align*}

Since $S_1 \cup S_2 = \varnothing$, then there are no points
where the Cauchy-Riemann equations hold and the 
function is not differentiable.

12.
Let $z = x+yi$ and $x,y \in \mathcal{R}$.
\begin{align*}
  f(z) &= \frac{z-i}{z+i} \\
    &= \frac{x+iy-i}{x+iy+i} \\
    &= \frac{(x+iy-i)(x-i(y+1)}{(x+iy+i)(x-i(y+1)} \\
    &= \frac{x^2 + y^2 -1}{x^2+y^2+2y+1} + i \frac{-2x}{x^2+y^2+2y+1}
      \\
\Rightarrow
  u(x,y) &=\frac{x^2 + y^2 -1}{x^2+y^2+2y+1}  \\
\Rightarrow
  v(x,y) &=\frac{-2x}{x^2+y^2+2y+1}
\end{align*}

The partial derivatives are then:
\begin{align*}
  \pdv{u}{x} &= \frac{4x(y+1)}{(x^2+y^2+2y+1)^2}           & \pdv{u}{y} &= \frac{2(-x^2 + y^2 +2y+1)}{(x^2+y^2+2y+1)^2} \\
  \pdv{v}{x} &=  \frac{2(-x^2+y^2+2y+1)}{(x^2+y^2+2y+1)^2} & \pdv{v}{y} &= \frac{4x(y+1)}{(x^2+y^2+2y+1)^2}
\end{align*}

Since $\pdv{u}{x}=\pdv{v}{y}$ and $\pdv{u}{y}=-\pdv{v}{x}$ 
$\forall z \in \mathcal{C} \setminus z = -i$,
then the function is differentiable $ \forall z \in \mathcal{C} \setminus z=-i$.

\newpage
\question*{\S 19.3 page 671: 2, 8}
\emph{Find:}
\newline
In each of Problems 2 and 8,
write the function value in the form
$a+bi$.

2.
\begin{equation*}
  \sin (1-4i)
\end{equation*}

8.
\begin{equation*}
  \cos (2-i) - \sin ( 2-i)
\end{equation*}

\emph{Solution:}
\newline

2.
\begin{align*}
\sin (1-4i) &= \frac{1}{i2} \left( e^{i(1-4i)} - e^{-i(1-4i)}\right) \\
  &= \sin(1) \cosh(-4) + i \cos(1) \sinh(-4) \\
  \\
\Rightarrow
  a &= \sin(1)\cosh(-4) \\
\Rightarrow
  b &= \cos(1)\sinh(-4)
\end{align*}

4. 
\begin{align*}
\cos(2-i) - \sin(2-i) 
  &= 
    \left( 
      \frac{1}{2} \left( e^{i(2-i)} + e^{-i(2- i)}\right)
    \right)
    -
    \left( 
      \frac{1}{i2} \left( e^{i(2-i)} - e^{-i(2- i)}\right)
    \right) \\
  &= 
    \cos(2)\cosh(-1) - i \sin(2)\sinh(-1)
    -
    \sin(2)\cosh(-1) - i\cos(2)\sinh(-1)  \\
  &=
    \left(
      \cos(2) - \sin(2)
    \right) \cosh(-1)
    -i
    \sinh(-1)
    \left( 
      \cos(2) + \sin(2)
    \right) \\
  \\
\Rightarrow 
  a &= \cosh(-1) \left( \cos(2) - \sin(2) \right)  \\
\Rightarrow 
  b &= \sinh(-1) \left( \cos(2) + \sin(2) \right) 
\end{align*}

\newpage
\question*{\S 19.4 page 672: 2,4}
\emph{Find:}
In each of Problems 2 and 4,
determine all values of the 
complex logarithm of $z$.

2. 
\begin{equation*}
  2 - 2i
\end{equation*}

4. 
\begin{equation*}
  1 + 5i
\end{equation*}

\emph{Solution:}
\newline
2.
\begin{align*}
  z&= 2-2i  \\
   &= \sqrt{2^2 + 2^2} e^{i \atan(\frac{-2}{2})} \\
   &= \sqrt{8} e^{i \frac{-\pi}{4}}
  \\
 \ln(z) &= \ln(\sqrt{8}) + i (\frac{-\pi}{4} + 2\pi n) \\
    &= \frac{\ln(8)}{2} + i (\frac{-\pi}{4} + 2\pi n) 
\end{align*}
where $n=1,2,...$.

4.
\begin{align*}
  z&= 1+5i \\
   &= \sqrt{1^2 + 5^2} e^{i\atan(\frac{5}{1})} \\
   &\approx \sqrt{26} e^{i 1.37} \\
  \\
\ln(z) &\approx \ln( \sqrt{26}) + i( 1.37 +2\pi n) \\
       &\approx \frac{\ln( 26)}{2} + i( 1.37 +2\pi n)
\end{align*}
where $n=1,2,...$.


\question*{ \S 19.5 page 676: 2, 6, 8}
\emph{Find:}
In each of Problems 2, 6 and 8,
determine all values of $z^w$.

2.
\begin{equation*}
  (1+i)^{2i}
\end{equation*}

6.
\begin{equation*}
  (1-i)^{\frac{1}{3}}
\end{equation*}

8.
\begin{equation*}
  (16)^{\frac{1}{4}}
\end{equation*}


\end{document}
