\documentclass[11pt]{homework}

\newcommand{\hwname}{Justin L. Clough}
\newcommand{\hwemail}{jlclough@usc.edu}
\newcommand{\hwtype}{Homework}
\newcommand{\hwnum}{8}
\newcommand{\hwclass}{AME: 525}

\usepackage{amsmath}
 \usepackage{delarray}

\begin{document}
\maketitle

\question
\emph{Find:}
\newline
Find the eigenpairs.
Check that the eigenvectors of distinct eigenvalues are orthogonal.
If the eigenvalues are distinct, find an orthogonal matrix that diagonalizes the matrix.
\begin{equation*}
A = 
  \begin{bmatrix}
  -3 & 5 \\
  5 & 4
  \end{bmatrix}
\end{equation*}

\emph{Solution:}
\newline
First find eigenvalues of $A$:
\begin{align*}
\det (\lambda I - A) = 
  \begin{vmatrix}
  \lambda + 3 & -5 \\
  -5          & \lambda - 4
  \end{vmatrix}
  &= 0 \\
  &= (\lambda + 3) ( \lambda -4) - (-5)(-5) \\
  &= \lambda^2 - \lambda - 37 
\end{align*}
which has solutions $\lambda = \frac{1}{2} \pm \frac{\sqrt{149}}{2}$.
The corresponding eigenvectors are:
\begin{align*}
  \text{When $\lambda = \frac{1}{2} + \frac{\sqrt{149}}{2}$:}
  \begin{bmatrix}
    \frac{7}{2} + \frac{\sqrt{149}}{2} & -5 \\
    -5                                & -\frac{7}{2} + \frac{\sqrt{149}}{2} \\
  \end{bmatrix}
  \begin{bmatrix}
    \vec e_1
  \end{bmatrix}
  &=
  \begin{bmatrix}
    \vec 0
  \end{bmatrix} \\
  \Rightarrow
  \begin{bmatrix}
    \vec e_1
  \end{bmatrix}
  &=
  \begin{bmatrix}
  10 \\
  7 + \sqrt{149}
  \end{bmatrix} \\
  \text{When $\lambda = \frac{1}{2} - \frac{\sqrt{149}}{2}$:}
  \begin{bmatrix}
    \frac{7}{2} - \frac{\sqrt{149}}{2} & -5 \\
    -5                                & -\frac{7}{2} - \frac{\sqrt{149}}{2} \\
  \end{bmatrix}
  \begin{bmatrix}
    \vec e_2
  \end{bmatrix}
  &=
  \begin{bmatrix}
    \vec 0
  \end{bmatrix} \\
  \Rightarrow
  \begin{bmatrix}
    \vec e_2
  \end{bmatrix}
  &=
  \begin{bmatrix}
  10 \\
  7 - \sqrt{149}
  \end{bmatrix}
\end{align*}
Construct $D$ such that $D_{ij} = (\vec e_i, \vec e_j)$. 
If the eigenvectors are orthogonal, 
then $D$ will be a diagonal matrix.
\begin{align*}
D &= 
  \begin{bmatrix}
  (\vec e_i, \vec e_j)
  \end{bmatrix} \\
  &=
  \begin{bmatrix}
  468.89 & 0 \\
  0      & 127.11
  \end{bmatrix}
\end{align*}
So the eigenvectors are orthogonal.
Construct $Q$ such that it has
columns of eigenvectors:
\begin{align*}
Q &= 
  \begin{bmatrix}
  | & | \\
  \vec e_1 & \vec e_2 \\
  | & | 
  \end{bmatrix}\\
  &=
  \begin{bmatrix}
  10             & 10 \\
  7 - \sqrt{149} & 7 + \sqrt{149}
  \end{bmatrix}
\end{align*}
Check that $Q$ diagonalizes $A$:
\begin{align*}
Q^T A Q &=
  \begin{bmatrix}
  10 & 7 - \sqrt{149}\\
  10 & 7 + \sqrt{149}
  \end{bmatrix}
  \begin{bmatrix}
  -3 & 5 \\
  5 & 4
  \end{bmatrix}
  \begin{bmatrix}
  10             & 10 \\
  7 - \sqrt{149} & 7 + \sqrt{149}
  \end{bmatrix} \\
  &=
  \begin{bmatrix}
  7 + \sqrt{149} & 0 \\
  0 & 7 - \sqrt{149}
  \end{bmatrix}
\end{align*}

\question
\emph{Find:}
\newline
Find the eigenpairs.
Check that the eigenvectors of distinct eigenvalues are orthogonal.
If the eigenvalues are distinct, find an orthogonal matrix that diagonalizes the matrix.
\begin{equation*}
A = 
  \begin{bmatrix}
  0 & 1 & 1 \\
  1 & 2 & 0 \\
  0 & 0 & 3
  \end{bmatrix}
\end{equation*}

\emph{Solution:}
\newline
First find eigenvalues of $A$:
\begin{align*}
\det (\lambda I - A) = 
  \begin{vmatrix}
  \lambda & -1 & -1 \\
  -1 & \lambda -2 & 0 \\
  0 & 0 & \lambda-3
  \end{vmatrix} 
  &= 0 \\
  &= (\lambda-3) ( \lambda (\lambda-2) - (-1)(-1)) \\
  &= (\lambda-3) ( \lambda^2 - 2 \lambda -1 )
\end{align*}
which has solutions $\lambda = 3, 1\pm \sqrt{2}$.
The corresponding eigenvectors are:
\begin{align*}
  \text{When $\lambda = 3$:}
  \begin{bmatrix}
  3  & -1 & -1 \\
  -1 & 1 & 0 \\
  0 & 0 & 0
  \end{bmatrix}
  \begin{bmatrix}
    \vec e_1
  \end{bmatrix}
  &= 
  \begin{bmatrix}
    \vec 0
  \end{bmatrix} \\
  \Rightarrow
  \begin{bmatrix}
    \vec e_1
  \end{bmatrix}
  &= 
  \begin{bmatrix}
    1 \\
    1 \\  
    2
  \end{bmatrix}  \\
\text{When $\lambda = 1 + \sqrt{2}$:}
  \begin{bmatrix}
  1+ \sqrt{1}  & -1 & -1 \\
  -1 & -1 + \sqrt{1} & 0 \\
  0 & 0 & -2 + \sqrt{1}
  \end{bmatrix}
  \begin{bmatrix}
    \vec e_2
  \end{bmatrix}
  &= 
  \begin{bmatrix}
    \vec 0
  \end{bmatrix} \\
  \Rightarrow
  \begin{bmatrix}
    \vec e_2
  \end{bmatrix}
  &= 
  \begin{bmatrix}
    1 \\
    1 + \sqrt{2} \\
    0
  \end{bmatrix}  \\
\text{When $\lambda = 1 - \sqrt{2}$:}
  \begin{bmatrix}
  1- \sqrt{1}  & -1 & -1 \\
  -1 & -1 - \sqrt{1} & 0 \\
  0 & 0 & -2 - \sqrt{1}
  \end{bmatrix}
  \begin{bmatrix}
    \vec e_3
  \end{bmatrix}
  &= 
  \begin{bmatrix}
    \vec 0
  \end{bmatrix} \\
  \Rightarrow
  \begin{bmatrix}
    \vec e_3
  \end{bmatrix}
  &= 
  \begin{bmatrix}
    1 \\
    1 - \sqrt{2} \\
    0
  \end{bmatrix}  \\
\end{align*}
Construct $D$ such that $D_{ij} = (\vec e_i, \vec e_j)$. 
If the eigenvectors are orthogonal, 
then $D$ will be a diagonal matrix.
\begin{align*}
D &= 
  \begin{bmatrix}
  (\vec e_i, \vec e_j)
  \end{bmatrix} \\
  &=
  \begin{bmatrix}
    4 & 2 +\sqrt{2} & 0 \\
    2+\sqrt{2} & 4 + 2\sqrt{2} & 0 \\
    0 & 0 & 4- 2\sqrt{2}
  \end{bmatrix}
\end{align*}
Since $( \vec e_2, \vec e_1) \neq 0$, 
then the set of eigenvectors are not orthogonal.

\newpage
\question
\emph{Find:}
\newline
Find the eigenpairs.
Check that the eigenvectors of distinct eigenvalues are orthogonal.
If the eigenvalues are distinct, find an orthogonal matrix that diagonalizes the matrix.
\begin{equation*}
A = 
  \begin{bmatrix}
  2 & -4 & 0 \\
  -4 & 0 & 0 \\
  0 &  0 & 0
  \end{bmatrix}
\end{equation*}

\emph{Solution:}
\newline
First find eigenvalues of $A$:
\begin{align*}
\det (\lambda I - A) = 
  \begin{vmatrix}
    \lambda -2 & 4 & 0 \\
    4 & \lambda & 0 \\
    0 & 0 & \lambda 
  \end{vmatrix}
  &= 0 \\
  &= \lambda ( (\lambda - 2)\lambda - (4)(4)) \\
  &= \lambda ( \lambda^2 - 2 \lambda - 16) 
\end{align*}
which has solutions $\lambda = 0, 1\pm \sqrt{17}$.
The corresponding eigenvectors are:
\begin{align*}
  \text{When $\lambda = 0$:}
    \begin{bmatrix}
      -2 & 4 & 0 \\
      4 & 0  & 0 \\
      0 & 0 & 0  
    \end{bmatrix}
    \begin{bmatrix}
      \vec e_1
    \end{bmatrix}
    &=
    \begin{bmatrix}
      \vec 0
    \end{bmatrix} \\
    \Rightarrow
    \begin{bmatrix}
      \vec e_1
    \end{bmatrix}
    &=
    \begin{bmatrix}
    0 \\
    0 \\
    1
    \end{bmatrix} \\
  \text{When $\lambda = 1+\sqrt{17}$:}
    \begin{bmatrix}
      -1 + \sqrt{17} & 4 & 0 \\
      4 & 1+\sqrt{17}  & 0 \\
      0 & 0 & 1+\sqrt{17}
    \end{bmatrix}
    \begin{bmatrix}
      \vec e_2
    \end{bmatrix}
    &=
    \begin{bmatrix}
      \vec 0
    \end{bmatrix} \\
    \Rightarrow
    \begin{bmatrix}
      \vec e_2
    \end{bmatrix}
    &=
    \begin{bmatrix}
    4 \\
    1 - \sqrt{17}\\
    0
    \end{bmatrix} \\
  \text{When $\lambda = 1 - \sqrt{17}$:}
    \begin{bmatrix}
      -1 - \sqrt{17} & 4 & 0 \\
      4 & 1-\sqrt{17}  & 0 \\
      0 & 0 & 1-\sqrt{17}
    \end{bmatrix}
    \begin{bmatrix}
      \vec e_3
    \end{bmatrix}
    &=
    \begin{bmatrix}
      \vec 0
    \end{bmatrix} \\
    \Rightarrow
    \begin{bmatrix}
      \vec e_3
    \end{bmatrix}
    &=
    \begin{bmatrix}
    4 \\
    1 + \sqrt{17}\\
    0
    \end{bmatrix}
\end{align*}
Construct $D$ such that $D_{ij} = (\vec e_i, \vec e_j)$. 
If the eigenvectors are orthogonal, 
then $D$ will be a diagonal matrix.
\begin{align*}
D &= 
  \begin{bmatrix}
  (\vec e_i, \vec e_j)
  \end{bmatrix} \\
  &=
  \begin{bmatrix}
  1 & 0 & 0 \\
  0 & 34 - 2 \sqrt{17} & 0 \\
  0 & 0 & 34 - 2 \sqrt{17} 
  \end{bmatrix}
\end{align*}
So the eigenvectors are orthogonal.
Construct $Q$ such that it has
columns of eigenvectors:
\begin{align*}
Q &= 
  \begin{bmatrix}
  | & | & |  \\
  \vec e_1 & \vec e_2 & \vec e_3 \\
  | & | & |
  \end{bmatrix}\\
  &=
  \begin{bmatrix}
  0 & 4 & 4 \\ 
  0 & 1- \sqrt{17} & 1+\sqrt{17} \\
  1 & 0 & 0
  \end{bmatrix}
\end{align*}
Check that $Q$ diagonalizes $A$:
\begin{align*}
Q^T A Q &=
  \begin{bmatrix}
  0 & 0 & 1 \\
  4 & 1-\sqrt{17} & 0 \\
  4 & 1+\sqrt{17} & 0 
  \end{bmatrix}
  \begin{bmatrix}
  2 & -4 & 0 \\
  -4 & 0 & 0 \\
  0 &  0 & 0
  \end{bmatrix}
  \begin{bmatrix}
  0 & 4 & 4 \\ 
  0 & 1- \sqrt{17} & 1+\sqrt{17} \\
  1 & 0 & 0
  \end{bmatrix} \\
  &=
  \begin{bmatrix}
  0 & 0 & 0 \\
  0 & 1+\sqrt{17} & 0 \\
  0 & 0 & 1-\sqrt{17}
  \end{bmatrix}
\end{align*}

\newpage
\question
\emph{Find:}
\newline
Find the eigenpairs.
Check that the eigenvectors of distinct eigenvalues are orthogonal.
If the eigenvalues are distinct, find an orthogonal matrix that diagonalizes the matrix.
\begin{equation*}
A = 
  \begin{bmatrix}
  1 & 3 & 0 \\
  3 & 0 & 1 \\
  0 & 1 & 1
  \end{bmatrix}
\end{equation*}

\emph{Solution:}
\newline

First find eigenvalues of $A$:
\begin{align*}
\det (\lambda I - A) = 
  \begin{vmatrix}
  \lambda - 1 & -3 & 0 \\
  -3 & \lambda & -1 \\
  0 & -1 & \lambda-1
  \end{vmatrix}
  &= 0 \\
  &= (\lambda-1) (\lambda( \lambda -1) - (-1)(-1)) -(-3) ((-3)(\lambda - 1)) \\
  &= (\lambda-1) (\lambda^2 - \lambda - 10) 
\end{align*}
which has solutions $\lambda = 1, \frac{1}{2} \pm \frac{\sqrt{41}}{2}$.
The corresponding eigenvectors are:
\begin{align*}
  \text{When $\lambda = 1$:}
  \begin{bmatrix}
  0 & -3 & 0 \\
  -3 & 1 & -1 \\
  0 & -1 & 0
  \end{bmatrix}
  \begin{bmatrix}
    \vec e_1
  \end{bmatrix}
  &= 
  \begin{bmatrix}
    \vec 0
  \end{bmatrix} \\
  \begin{bmatrix}
    \vec e_1
  \end{bmatrix}
  &= 
  \begin{bmatrix}
  1 \\
  0 \\
  -3
  \end{bmatrix} \\
  \text{When $\lambda = \frac{1}{2} + \frac{\sqrt{41}}{2}$:}
  \begin{bmatrix}
  \frac{-1}{2} + \frac{\sqrt{41}}{2} & -3 & 0 \\
  -3 & \frac{1}{2} + \frac{\sqrt{41}}{2} & -1 \\
  0 & -1 & \frac{1}{2} + \frac{\sqrt{41}}{2}
  \end{bmatrix}
  \begin{bmatrix}
    \vec e_2
  \end{bmatrix}
  &= 
  \begin{bmatrix}
    \vec 0
  \end{bmatrix} \\
  \begin{bmatrix}
    \vec e_2
  \end{bmatrix}
  &= 
  \begin{bmatrix}
  6 \\  
  -1 + \sqrt{41} \\
  2
  \end{bmatrix} \\
  \text{When $\lambda = \frac{1}{2} - \frac{\sqrt{41}}{2}$:}
  \begin{bmatrix}
  \frac{-1}{2} - \frac{\sqrt{41}}{2} & -3 & 0 \\
  -3 & \frac{1}{2} - \frac{\sqrt{41}}{2} & -1 \\
  0 & -1 & \frac{1}{2} - \frac{\sqrt{41}}{2}
  \end{bmatrix}
  \begin{bmatrix}
    \vec e_3
  \end{bmatrix}
  &= 
  \begin{bmatrix}
    \vec 0
  \end{bmatrix} \\
  \begin{bmatrix}
    \vec e_3
  \end{bmatrix}
  &= 
  \begin{bmatrix}
  6 \\  
  -1 - \sqrt{41} \\
  2
  \end{bmatrix} 
\end{align*}
Construct $D$ such that $D_{ij} = (\vec e_i, \vec e_j)$. 
If the eigenvectors are orthogonal, 
then $D$ will be a diagonal matrix.
\begin{align*}
D &= 
  \begin{bmatrix}
  (\vec e_i, \vec e_j)
  \end{bmatrix} \\
  &=
  \begin{bmatrix}
  10 & 0 & 0 \\ 
  0 & 205-2\sqrt{41} & 0 \\
  0 & 0 & 205 - 2\sqrt{41}
  \end{bmatrix}
\end{align*}
So the eigenvectors are orthogonal.
Construct $Q$ such that it has
columns of eigenvectors:
\begin{align*}
Q &= 
  \begin{bmatrix}
  | & | & |  \\
  \vec e_1 & \vec e_2 & \vec e_3 \\
  | & | & |
  \end{bmatrix}\\
  &=
  \begin{bmatrix}
  1 & 6 & 6 \\
  0 & -1+\sqrt{41} & -1-\sqrt{41} \\
  -3 & 2 & 2
  \end{bmatrix}
\end{align*}
Check that $Q$ diagonalizes $A$:
\begin{align*}
Q^T A Q &=
  \begin{bmatrix}
  1 & 0 & -3 \\
  6 & -1+\sqrt{41} & 2 \\
  6 & -1-\sqrt{41} & 2 
  \end{bmatrix}
  \begin{bmatrix}
  1 & 3 & 0 \\
  3 & 0 & 1 \\
  0 & 1 & 1
  \end{bmatrix}
  \begin{bmatrix}
  1 & 6 & 6 \\
  0 & -1+\sqrt{41} & -1-\sqrt{41} \\
  -3 & 2 & 2
  \end{bmatrix} \\
  &=
  \begin{bmatrix}
  1 & 0 & 0 \\
  0 & \frac{1}{2}+\frac{\sqrt{41}}{2} & 0 \\
  0 & 0 & \frac{1}{2}-\frac{\sqrt{41}}{2}
  \end{bmatrix}
\end{align*}

\question*{\S 12.3 page 421: 12 and 14}
\emph{Find:}
\newline
In each of Problems 12 and 14, 
determine whether the matrix unitary, hermitian
skew-hermitian, or none of these.
Find the eigenvalues.
\newline
12.
\begin{equation*}
  A = 
  \begin{bmatrix}
  \frac{1}{\sqrt{2}} & \frac{i}{\sqrt{2}} & 0 \\
  -\frac{1}{\sqrt{2}} & \frac{i}{\sqrt{2}} & 0 \\
  0 & 0 & 1
  \end{bmatrix}
\end{equation*}
14. 
\begin{equation*}
A = 
  \begin{bmatrix}
  -1 & 0 & 3-i \\
  0 & 1 & 0 \\
  3+i & 0 & 0
  \end{bmatrix}
\end{equation*}

\emph{Solution:}
\newline

12. First check to see if $A$ is unitary:
\begin{align*}
\bar{A}A &=
  \begin{bmatrix}
  a
  \end{bmatrix}
\end{align*}

\end{document}
