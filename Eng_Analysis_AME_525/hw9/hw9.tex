\documentclass[11pt]{homework}

\newcommand{\hwname}{Justin L. Clough}
\newcommand{\hwemail}{jlclough@usc.edu}
\newcommand{\hwtype}{Homework}
\newcommand{\hwnum}{9}
\newcommand{\hwclass}{AME: 525}

\usepackage{amsmath}
 \usepackage{delarray}

\begin{document}
\maketitle

\question
\emph{Find:}
\newline
Consider a new basis given by the orthogonal unit vectors:
\begin{align*}
e_1 &= 
  \left( \frac{1}{\sqrt{2}}, \frac{1}{\sqrt{2}}, 0 \right) \\
e_2 &= 
  \left( \frac{1}{\sqrt{2}}, \frac{-1}{\sqrt{2}}, 0 \right) \\
e_3 &= 
  \left( 0,0,1 \right)
\end{align*}
\noindent
Find the representation in the new basis of the matrix:
\begin{equation*}
A = 
  \begin{bmatrix}
    1 & 2 & 3 \\
    4 & 5 & 6 \\
    7 & 8 & 9 
  \end{bmatrix}
\end{equation*}
\noindent
Use the principal axis theorem to analyze the conics
(i.e., find the conic sections, principal axes, and orientations).

\emph{Solution:}
\newline
Define $Q$ such that it has columns of the orthogonal unit vectors:
\begin{align*}
Q &=
    \begin{bmatrix}
    | & | & | \\
    e_1 & e_2 & e_3 \\
    | & | & | 
    \end{bmatrix} \\
  &=
    \begin{bmatrix}
    \frac{1}{\sqrt{2}} & \frac{1}{\sqrt{2}}  & 0 \\
    \frac{1}{\sqrt{2}} & \frac{-1}{\sqrt{2}} & 0 \\
    0                  & 0                   & 1
    \end{bmatrix}
\end{align*}

Transform $A$ by $Q$:

\begin{align*}
Q^T A Q &= 
  \begin{bmatrix}
    \frac{1}{\sqrt{2}} & \frac{1}{\sqrt{2}} & 0 \\
    \frac{1}{\sqrt{2}}   & \frac{-1}{\sqrt{2}} & 0 \\
    0 & 0 & 1
  \end{bmatrix}
  \begin{bmatrix}
    1 & 2 & 3 \\
    4 & 5 & 6 \\
    7 & 8 & 9 
  \end{bmatrix}
  \begin{bmatrix}
    \frac{1}{\sqrt{2}} & \frac{1}{\sqrt{2}}  & 0 \\
    \frac{1}{\sqrt{2}} & \frac{-1}{\sqrt{2}} & 0 \\
    0                  & 0                   & 1
  \end{bmatrix} \\
  &=
  \begin{bmatrix}
  6  & -1 & \frac{9}{\sqrt{2}} \\
  -3 & 0 & \frac{-3}{\sqrt{2}} \\
  \frac{15}{\sqrt{2}} & \frac{-1}{\sqrt{2}} & 9
  \end{bmatrix}
\end{align*}

Choose a vector $x$ such that:
\begin{equation*}
x^T Q^T A Q x = 
  6 x_1^2 
  -4 x_1 x_2
  +\frac{14}{\sqrt{2}} x_1 x_3 
  -\frac{4}{\sqrt{2}} x_2 x_3 
  +9 x_3^2
\end{equation*}

Try a symmetric matrix $S$ such that:
\begin{equation*}
S =
  \begin{bmatrix}
    6 & -2 & \frac{7}{\sqrt{2}} \\
    -2 & 0 & \frac{-2}{\sqrt{2}} \\
    \frac{7}{\sqrt{2}} & \frac{-2}{\sqrt{2}} & 9
  \end{bmatrix}
\end{equation*}
which has eigenvalues:
\begin{align*}
\det (\lambda I - S ) =
  \begin{bmatrix}
    \lambda - 6 & 2 & \frac{-7}{\sqrt{2}} \\
    2 & \lambda & \frac{2}{\sqrt{2}} \\
    \frac{-7}{\sqrt{2}} & \frac{2}{\sqrt{2}} & \lambda-9
  \end{bmatrix}
  &= 0 \\
  &= \lambda^3 - 15 \lambda^2 + \frac{109}{2} \lambda + 20
\end{align*}
\noindent
which has approximate solutions of $\lambda = 0.34, 7.67 \pm 0.92i$.
So the standard form is:

\begin{equation*}
0.34 y_1^2 + (7.67 - 0.92i) y_2^2 + (7.67 + 0.92) y_3^2
\end{equation*}

\question
\emph{Find:}
\newline
By direct calculation find the three invariants of:
\begin{equation*}
A = 
  \begin{bmatrix}
  2 & 4 & 6 \\
  0 & 3 & 0 \\
  4 & 0 & 5
  \end{bmatrix}
\end{equation*}

Then, calculate the eigenvalues and use the result to check your answer.

\emph{Solution:}
\newline
The invariants are:
\begin{align*}
\beta_1 &= \text{tr}(A) \\
        &= 2 + 3 + 5 \\
        &= 10 \\
\beta_2 &=
  \begin{vmatrix}
  2 & 4 \\
  0 & 3
  \end{vmatrix}
  +
  \begin{vmatrix}
  3 & 0 \\
  0 & 5
  \end{vmatrix} \\
  &= 6 + 15 \\
  &= 21 \\
\beta_3 &= \det(A) \\
        &= -42
\end{align*}

The eigenvalues of $A$ are then:
\begin{align*}
\det(\lambda I - A) = 
  \begin{bmatrix}
  \lambda - 2 & -4 & -6 \\
  0 & \lambda-3 & 0 \\
  -4 & 0 & \lambda-5
  \end{bmatrix}
  &= 0 \\
  &= (\lambda -3) ( \lambda^2 - 7 \lambda - 14)
\end{align*}
\noindent
which has solutions $\lambda = 3, \frac{7}{2} \pm \frac{\sqrt{105}}{2}$.
The invariants are then:

\begin{align*}
\beta_1 &= 3 
        + \left( \frac{7}{2} + \frac{\sqrt{105}}{2} \right)
        + \left( \frac{7}{2} - \frac{\sqrt{105}}{2} \right) \\
        &= 10 \\
\beta_2 &= 3 \left( \frac{7}{2} + \frac{\sqrt{105}}{2} \right)
        + \left( \frac{7}{2} + \frac{\sqrt{105}}{2} \right) \left( \frac{7}{2} - \frac{\sqrt{105}}{2} \right) \\
        &= 21 \\
\beta_3 &= 3 
         \left( \frac{7}{2} + \frac{\sqrt{105}}{2} \right)
         \left( \frac{7}{2} - \frac{\sqrt{105}}{2} \right) \\
        &= -42
\end{align*}

\newpage
\question
\emph{Find:}
\newline
Find the square root of:
\begin{equation*}
A = 
  \begin{bmatrix}
  3 & 4 \\
  4 & -3 
  \end{bmatrix}
\end{equation*}

Use spectral theory. 

\emph{Solution:}
\newline
The eigenvalues of $A$ are:
\begin{align*}
\det(\lambda I -A) =
  \begin{bmatrix}
  \lambda - 3 & -4 \\
  -4 & \lambda+3 
  \end{bmatrix}
  &= 0 \\
  &= (\lambda-3)(\lambda+3) - (-4)(-4) \\
  &= (\lambda-5)(\lambda+5) \\
\end{align*}
which has solutions of $\lambda = \pm 5$.
The corresponding eigenvectors
\begin{align*}
\text{When $\lambda=5$:}
  \begin{bmatrix}
  2 & -4 \\
  -4 & 8 
  \end{bmatrix}
  \begin{bmatrix}
    e_1
  \end{bmatrix}
  &= 
  \begin{bmatrix}
    0
  \end{bmatrix} \\
  \Rightarrow
  \begin{bmatrix}
    e_1
  \end{bmatrix}
  &= 
  k_1
  \begin{bmatrix}
    2 \\
    1
  \end{bmatrix} \\
\text{When $\lambda = -5$:}
  \begin{bmatrix}
  -8 & -4 \\
  -4 & 2
  \end{bmatrix}
  \begin{bmatrix}
    e_2
  \end{bmatrix}
  &= 
  \begin{bmatrix}
    0
  \end{bmatrix} \\
  \Rightarrow
  \begin{bmatrix}
    e_2
  \end{bmatrix}
  &= 
  k_2
  \begin{bmatrix}
  1 \\
  -2
  \end{bmatrix} 
\end{align*}
with $k_1 = k_2 = \frac{1}{\sqrt{5}}$.
For a $Q$ such that it has columns 
of the eigenvectors of $A$.
\begin{align*}
Q &=
  \begin{bmatrix}
  | & |  \\
  e_1 & e_2 \\
  | & |
  \end{bmatrix} \\
  &=
  \begin{bmatrix}
  2 k_1 & k_2 \\
  k_1 & -2 k_2 
  \end{bmatrix}
\end{align*}

Transform $A$ with $Q$:
\begin{align*}
Q^T A Q &=
  \begin{bmatrix}
  2 k_1 & k_1 \\
  k_2 & -2 k_2
  \end{bmatrix}
  \begin{bmatrix}
  3 & 4 \\
  4 & -3 
  \end{bmatrix}
  \begin{bmatrix}
  2 k_1 & k_2 \\
  k_1 & -2 k_2 
  \end{bmatrix} \\
  &=
  \begin{bmatrix}
  5 & 0 \\
  0 & -5 
  \end{bmatrix}
\end{align*}

Taking the square root and transforming back:
\begin{align*}
Q \sqrt{Q^T A Q}Q^T &=
  \begin{bmatrix}
    2 k_1 & k_2 \\
    k_1 & -2 k_2 
  \end{bmatrix} 
  \begin{bmatrix}
    \sqrt{5} & 0 \\
    0 & \sqrt{-5}
  \end{bmatrix}
  \begin{bmatrix}
    2 k_1 & k_1 \\
    k_2 & -2 k_2
  \end{bmatrix} \\
  &=
  \frac{1}{\sqrt{5}}
  \begin{bmatrix}
    4+i  & 2-2i \\
    2-2i & 1+4i
  \end{bmatrix}
\end{align*}

\question
\emph{Find:}
\newline
Show by direct calculation that the matrix below 
satisfies its own characteristic equation.
\begin{equation*}
A = 
  \begin{bmatrix}
  2 & 4 & 6 \\
  0 & 3 & 0 \\
  4 & 0 & 5
  \end{bmatrix}
\end{equation*}

\question*{\S 19.1 page 658: 4, 8}
\emph{Find:}
\newline
In each of Problems 4 and 8, 
carry out the indicated calculation.

4. 
\begin{equation*}
  \frac{(2+i)-(3-4i)}{(5-i)(3+i)}
\end{equation*}

8. 
\begin{equation*}
(3+i)^3
\end{equation*}


\question*{\S 19.1 page 658: 12}
\emph{Find:}
\newline
In Problem 12, determine the magnitude and all of the arguments 
of $z$.

12.
\begin{equation*}
  -2+2i
\end{equation*}


\question*{\S 19.1 page 658: 18, 20, 22}
\emph{Find:}
\newline
In each of Problems 18, 20, and 22
write the number in polar form.

18.
\begin{equation*}
-7i
\end{equation*}

20. 
\begin{equation*}
  -4 -i
\end{equation*}

22.
\begin{equation*}
  -12 +3i
\end{equation*}

\question*{\S 19.1 page 658: 24}
\emph{Find:}
\newline
Let $z=a+ib$. 
Determine Re($z^2$) and Im($z^2$).

\end{document}
