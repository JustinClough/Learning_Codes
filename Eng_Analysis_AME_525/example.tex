\documentclass[a4paper, 12pt]{article}
\title{}
\usepackage{geometry}
\usepackage{float}
\usepackage{subfigure}
\usepackage[justification=centering]{caption}
\usepackage{enumerate}
\usepackage{multirow}
\usepackage{listings}
\lstset{
    escapechar=`,
    language=C++,
    numbers=left,
    tabsize=2,
    prebreak=\raisebox{0ex}[0ex][0ex]{\ensuremath{\hookleftarrow}},
    frame=single,
    breaklines=true,
}
\usepackage{graphicx}
\graphicspath{ {./} }
\usepackage{nameref}
\usepackage{amsmath}
\usepackage{amssymb}
\usepackage{amsfonts}
\usepackage[linesnumbered,ruled]{algorithm2e}
\usepackage{tikz}
\usetikzlibrary{calc,patterns,decorations.pathmorphing,decorations.markings,positioning,automata}
\usepackage{pgfplots}
\usepackage{pgfplotstable}
\usepackage{makecell}
\usepackage{ulem}
\usepackage{verbatim}

\begin{document}

\section*{Part II: Implementation (reattempted)} \label{sec:intro}
The purpose of this exercise was to use the Finite Element Method (FEM) 
to approximate, and study, the solution to the problem presented in 
Eq. \ref{eq:StrongForm}.

\begin{align}
-(p(x)u'(x))' + q(x) u(x) &= f(x)  \label{eq:StrongForm} \\
u(0) = \alpha , u(1) = \beta  &
\end{align}

\noindent
This equation was solved on the domain $x\in[0,1]$. 
The analytical solution was assumed to exist uniquely.
The variables in Eq. \ref{eq:StrongForm} are as denoted below in Eq. \ref{eq:GivenConds}.

\begin{align} \label{eq:GivenConds}
u(x)\in & C^2 [0,1]  & \\
f,q \in & C^0 [0,1] \quad & q \geq 0\\
p   \in & C^1 [0,1] \quad &p > 0
\end{align}

\noindent
The problem domain of one unit in one dimension was divided into a nonuniform but structured mesh.
The rule for determining element size is shown in the equation below.

\[
  h_{j} = \begin{cases} 
    0.9 \Delta x \quad \text{ for odd j}   \\
    1.1 \Delta x \quad \text{ for even j}
  \end{cases}
\]

\noindent
The $\Delta x$ referred to is the average size of any given element. 
Specifically, it is calculated by use of Eq. \ref{eq:DeltaX}.

\begin{equation} \label{eq:DeltaX}
\Delta x = \frac{1}{N+1}
\end{equation}

\noindent
Here, $N$ is the number of degrees of freedom; $N+1$ is the number of elements.
Tests were conducted for $N=10, 20, 40, 80, 160, 320$. 
These six mesh size tests were conducted for six different cases which are described in 
Table \ref{tab:Cases}.

\begin{table}
\caption{ Case numbers and descriptions.}
\vspace{0.1in}
\centering
\begin{tabular}{ |c|c|c|c|c|c|}
  \hline
  Case Number & $\alpha$ & $\beta$ & $p$ & $q$ & $u$ \\
  \hline
  1           &   0      &  0      &  3  &  2  &  $x(x-1)(sin(5x)+3e^x)$ \\
  \hline
  2           &   0      &  0      &  $1+x$  &  0  &  $x(x-1)(sin(5x)+3e^x)$ \\
  \hline
  3           &   4      &  4      &  3  &  2  &  $4$ \\
  \hline
  4           &   -2     &  -1     &  3  &  2  &  $x-2$ \\
  \hline
  5           &   -3     &  -2     &  3  &  2  &  $x^2-3$ \\
  \hline
\end{tabular}
\label{tab:Cases}
\end{table}

A computer code was written to evaluate the posed problems. 
The results follow this section and the code itself 
\sout{is appended to this report} can be viewed online at 
\texttt{www.github.com/JustinClough/Learning\_Codes/tree/master/FEA}; 
it was not printed to conserve paper.
Tabulated data, in the form of error norms and convergence order 
are presented in Tables \ref{tab:C1} through \ref{tab:C5}.


\begin{table}[!ht]
\caption{Errors and convergence orders for Case 1.}
\vspace{0.1in}
\centering
\begin{tabular}{|c|c|c| c| c| c| c|}
\hline
 $N+1$&  $||e_h(\cdot)||_{L^2([0,1])}$ & Order  & $||e_h(\cdot)||_{L^\infty([0,1])}$ & Order& $||e_h(\cdot)||_h$& Order \\
 \hline
     10  &  3.528607e-03 & 0.00 & 2.291486e-02 & 0.00 & 3.317002e-01 & 0.00\\
     20  &  8.963598e-04 & 1.97 & 6.683362e-03 & 1.77 & 1.792580e-01 & 0.88\\
     40  &  2.250262e-04 & 1.99 & 2.211256e-03 & 1.59 & 9.377064e-02 & 0.93\\
     80  &  5.631579e-05 & 1.99 & 6.268937e-04 & 1.81 & 4.803148e-02 & 0.96\\
     160 &  1.408267e-05 & 1.99 & 1.663275e-04 & 1.91 & 2.431665e-02 & 0.98\\
     320 &  3.520901e-06 & 1.99 & 4.280166e-05 & 1.95 & 1.223537e-02 & 0.99\\
\hline
\end{tabular}
\label{tab:C1}
\end{table}

\begin{table}[!ht]
\caption{Errors and convergence orders for Case 2.}
\vspace{0.1in}
\centering
\begin{tabular}{|c|c|c| c| c| c| c|}
\hline
 $N+1$&  $||e_h(\cdot)||_{L^2([0,1])}$ & Order  & $||e_h(\cdot)||_{L^\infty([0,1])}$ & Order& $||e_h(\cdot)||_h$& Order \\
 \hline
     10  & 5.291277e-03 & 0.00 & 2.157976e-02 & 0.00 & 3.260128e-01 & 0.00\\
     20  & 1.310432e-03 & 2.01 & 6.406604e-03 & 1.75 & 1.774688e-01 & 0.87\\
     40  & 3.273072e-04 & 2.00 & 2.134625e-03 & 1.58 & 9.327537e-02 & 0.92\\
     80  & 8.184745e-05 & 1.99 & 6.168818e-04 & 1.79 & 4.790149e-02 & 0.96\\
     160 & 2.046769e-05 & 1.99 & 1.650486e-04 & 1.90 & 2.428336e-02 & 0.98\\
     320 & 5.117846e-06 & 1.99 & 4.264007e-05 & 1.95 & 1.222695e-02 & 0.98\\
\hline
\end{tabular}
\label{tab:C2}
\end{table}

\begin{table}[!ht]
\caption{Errors and convergence orders for Case 3.}
\vspace{0.1in}
\centering
\begin{tabular}{|c|c|c| c| c| c| c|}
\hline
 $N+1$&  $||e_h(\cdot)||_{L^2([0,1])}$ & Order  & $||e_h(\cdot)||_{L^\infty([0,1])}$ & Order& $||e_h(\cdot)||_h$& Order \\
 \hline
     10  & 5.058005e-15 &  0.00 & 1.243450e-14 &  0.00 & 1.539017e-14 &  0.00\\
     20  & 2.019833e-15 &  1.32 & 7.105427e-15 &  0.80 & 2.019843e-14 & -0.39\\
     40  & 3.587761e-14 & -4.15 & 1.021405e-13 & -3.84 & 1.571914e-13 & -2.96\\
     80  & 7.490578e-14 & -1.06 & 2.904343e-13 & -1.50 & 5.474745e-13 & -1.80\\
     160 & 9.489067e-14 & -0.34 & 2.859935e-13 &  0.02 & 8.401636e-13 & -0.61\\
     320 & 4.464905e-12 & -5.55 & 1.042100e-11 & -5.18 & 1.436984e-11 & -4.09\\
\hline
\end{tabular}
\label{tab:C3}
\end{table}

\begin{table}[!ht]
\caption{Errors and convergence orders for Case 4.}
\vspace{0.1in}
\centering
\begin{tabular}{|c|c|c| c| c| c| c|}
\hline
 $N+1$&  $||e_h(\cdot)||_{L^2([0,1])}$ & Order  & $||e_h(\cdot)||_{L^\infty([0,1])}$ & Order& $||e_h(\cdot)||_h$& Order \\
 \hline
     10  & 2.170933e-15 &  0.00 & 5.329071e-15 &  0.00 & 6.018740e-15  &  0.00\\
     20  & 6.371945e-16 &  1.76 & 2.442491e-15 &  1.12 & 7.750267e-15  & -0.36\\
     40  & 1.067982e-14 & -4.06 & 3.175238e-14 & -3.70 & 5.341707e-14  & -2.78\\
     80  & 2.419360e-14 & -1.17 & 7.016610e-14 & -1.14 & 1.886608e-13  & -1.82\\
     160 & 3.201968e-14 & -0.40 & 1.316725e-13 & -0.90 & 3.214517e-13  & -0.76\\
     320 & 1.626372e-12 & -5.66 & 3.860690e-12 & -4.87 & 5.246423e-12  & -4.02\\
\hline
\end{tabular}
\label{tab:C4}
\end{table}

\begin{table}[!ht]
\caption{Errors and convergence orders for Case 5.}
\vspace{0.1in}
\centering
\begin{tabular}{|c|c|c| c| c| c| c|}
\hline
 $N+1$&  $||e_h(\cdot)||_{L^2([0,1])}$ & Order  & $||e_h(\cdot)||_{L^\infty([0,1])}$ & Order& $||e_h(\cdot)||_h$& Order \\
 \hline
     10  & 3.411114e-15 &  0.00 & 2.688889e-03 & 0.00 & 5.648304e-02 & 0.00\\
     20  & 1.557481e-15 &  1.13 & 6.722222e-04 & 2.00 & 2.877427e-02 & 0.97\\
     40  & 2.048400e-14 & -3.71 & 1.680556e-04 & 2.00 & 1.451849e-02 & 0.98\\
     80  & 4.361838e-14 & -1.09 & 4.201389e-05 & 2.00 & 7.291860e-03 & 0.99\\
     160 & 6.317536e-14 & -0.53 & 1.050347e-05 & 2.00 & 3.654057e-03 & 0.99\\
     320 & 2.883586e-12 & -5.51 & 2.625868e-06 & 2.00 & 1.829057e-03 & 0.99\\
\hline
\end{tabular}
\label{tab:C5}
\end{table}

\noindent
The following was done to incorporate non-homogeneous Dirichlet boundary conditions.
First, the only change to the weak form of the problem would be separating the 
trial and test spaces; the solution must have a non-zero value on the specified 
boundary whereas the trial function must be zero on the boundaries.
For the FEM formulation, the contributions of the non-zero boundary
conditions are taken account for in the forcing vector. 
No changes need to be made to represent $U_h$ on \emph{interior} elements.
However, on the boundary, additional hat functions are used to recover the
assigned boundary values.
The matrix \textbf{A} is not changed as the original PDE does not change. 
The forcing vector \textbf{F} includes the addition of the boundary terms
on the first and last entry.

The error measures behave as expected for Cases 1 and 2; 
the order of the $L^2$ and $L^\infty$ norms tend to 2.0 
and the order of the energy norm tends to 1.0.
The solution to Cases 3 and 4 was captured up to machine precision.
This is shown in the first row of Tables \ref{tab:C3} and \ref{tab:C4}.
The reason for the increasing norm values with increasing mesh density 
was due to round off error in calculating the norms.
For example, assuming machine precision is $\epsilon = 10^{-16}$ 
then $3*320*\epsilon = 9.6 \times 10^{-14}$; this represents sum taken
to compute the $L^2$ norm on the mesh with $N+1=320$ with
floating point arithmetic in the best case senario.
A possible solution to this would be to normalize the norms 
against the number of evaluation points. 
The result of doing this is shown in Table \ref{tab:C3n} for Case 3.
The $L^2$ norm for Case 5 behaves similar to that of Cases 3 and 4.
The $L^\infty$ and energy norms behave similar to that of Cases 1 and 2.

\begin{table}[!ht]
\caption{Errors and convergence orders for Case 3 with norms divided by number of evaluation points.}
\vspace{0.1in}
\centering
\begin{tabular}{|c|c|c| c| c| c| c|}
\hline
 $N+1$&  $||e_h(\cdot)||_{L^2([0,1])}$ & Order  & $||e_h(\cdot)||_{L^\infty([0,1])}$ & Order& $||e_h(\cdot)||_h$& Order \\
 \hline
     10  & 1.686002e-16 & 0.00 & 1.381611e-16 &  0.00 & 5.130058e-16 &  0.00\\
     20  & 3.366388e-17 & 2.32 & 3.947460e-17 &  1.80 & 3.366405e-16 &  0.60\\
     40  & 2.989801e-16 &-3.15 & 2.837237e-16 & -2.84 & 1.309928e-15 & -1.96\\
     80  & 3.121074e-16 &-0.06 & 4.033810e-16 & -0.50 & 2.281144e-15 & -0.80\\
     160 & 1.976889e-16 & 0.65 & 1.986066e-16 &  1.02 & 1.750341e-15 &  0.38\\
     320 & 4.650943e-15 &-4.55 & 3.618402e-15 & -4.18 & 1.496858e-14 & -3.09\\
\hline
\end{tabular}
\label{tab:C3n}
\end{table}

The \texttt{Eigen} software package
was used as the linear solver for this project. 
The matrix inversion method used was Householder-QR with pivoting.

\begin{comment}
\newpage
\appendix
\section{Source Code and Headers} \label{sec:code}

\subsection{fea\_hw1.cpp} \label{subsec:fea_hw1.cpp}
\lstinputlisting{/lore/clougj/Learning_Codes/FEA/HW1/src/fea_hw1.cpp}

\subsection{driver.cpp} \label{subsec:driver.cpp}
\lstinputlisting{/lore/clougj/Learning_Codes/FEA/HW1/src/driver.cpp}
\subsection{driver.hpp} \label{subsec:driver.hpp}
\lstinputlisting{/lore/clougj/Learning_Codes/FEA/HW1/src/driver.hpp}

\subsection{element1D.cpp} \label{subsec:element1D.cpp}
\lstinputlisting{/lore/clougj/Learning_Codes/FEA/HW1/src/element1D.cpp}
\subsection{element1D.hpp} \label{subsec:element1D.hpp}
\lstinputlisting{/lore/clougj/Learning_Codes/FEA/HW1/src/element1D.hpp}

\subsection{errorCalcs.cpp} \label{subsec:errorCalcs.cpp}
\lstinputlisting{/lore/clougj/Learning_Codes/FEA/HW1/src/errorCalcs.cpp}
\subsection{errorCalcs.hpp} \label{subsec:errorCalcs.hpp}
\lstinputlisting{/lore/clougj/Learning_Codes/FEA/HW1/src/errorCalcs.hpp}

\subsection{forcing.cpp} \label{subsec:forcing.cpp}
\lstinputlisting{/lore/clougj/Learning_Codes/FEA/HW1/src/forcing.cpp}
\subsection{forcing.hpp} \label{subsec:forcing.hpp}
\lstinputlisting{/lore/clougj/Learning_Codes/FEA/HW1/src/forcing.hpp}

\subsection{mesh1D.cpp} \label{subsec:mesh1D.cpp}
\lstinputlisting{/lore/clougj/Learning_Codes/FEA/HW1/src/mesh1D.cpp}
\subsection{mesh1D.hpp} \label{subsec:mesh1D.hpp}
\lstinputlisting{/lore/clougj/Learning_Codes/FEA/HW1/src/mesh1D.hpp}

\subsection{stiffness.cpp} \label{subsec:stiffness.cpp}
\lstinputlisting{/lore/clougj/Learning_Codes/FEA/HW1/src/stiffness.cpp}
\subsection{stiffness.hpp} \label{subsec:stiffness.hpp}
\lstinputlisting{/lore/clougj/Learning_Codes/FEA/HW1/src/stiffness.hpp}

\subsection{eig\_wrap.hpp} \label{subsec:eig_wrap.hpp}
\lstinputlisting{/lore/clougj/Learning_Codes/FEA/HW1/src/eig_wrap.hpp}
\end{comment}

\end{document}
