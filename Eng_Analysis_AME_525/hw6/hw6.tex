\documentclass[11pt]{homework}

\newcommand{\hwname}{Justin L. Clough}
\newcommand{\hwemail}{jlclough@usc.edu}
\newcommand{\hwtype}{Homework}
\newcommand{\hwnum}{6}
\newcommand{\hwclass}{AME: 525}

\usepackage{amsmath}
 \usepackage{delarray}

\begin{document}
\maketitle

\question
\emph{Find:}
\newline
Evaluate the determinant of:
\begin{equation*}
  \begin{bmatrix}
  6 & 1 & -1 & 2 & 2 & 1 \\
  0 & -4 & 2 & 2 & -3 & 1 \\
  0 & 0 & 5 & 10 & 1 & -7 \\
  0 & 0 & 0 & 14 & 0 & 0 \\
  0 & 0 & 0 & 0 & 13 & -4 \\
  0 & 0 & 0 & 0 & 0 & 3
  \end{bmatrix}
\end{equation*}
\emph{Solution:}
\newline
From lecture, if $T$ is a triangular matrix with entries $T = t_{ij} \quad ( i,j = 1 (1) n)$,
then $\det (T) = \prod_{i=1}^{n} t_{ii}$.

\begin{align*}
  \begin{vmatrix}
  6 & 1 & -1 & 2 & 2 & 1 \\
  0 & -4 & 2 & 2 & -3 & 1 \\
  0 & 0 & 5 & 10 & 1 & -7 \\
  0 & 0 & 0 & 14 & 0 & 0 \\
  0 & 0 & 0 & 0 & 13 & -4 \\
  0 & 0 & 0 & 0 & 0 & 3
  \end{vmatrix}
  &=
  6 \times (-4) \times 5 \times 14 \times 13 \times 3 \\
  &= -65520
\end{align*}

\newpage
\question*{\S 11.7 page 392: 2 and 4}
\emph{Find:}
\newline
In each of Problems 2 and 4, solve the system using 
Cramer's rule, 
or show that the rule does not apply because the matrix of coefficients is singular.
\newline
2. 
\begin{align*}
  x_1 + 4 x_2 &= 3 \\
  x_1 +  x_2 &= 0
\end{align*}
4. 
\begin{align*}
  5 x_1 - 6 x_2 +x_3 &= 4 \\
  -x_1 + 3 x_2 - 4 x_3 &= 5\\
  2 x_1 + 3 x_2 + x_3 &= -8 
\end{align*}

\emph{Solution:}
\newline
2. Let $A \vec x = \vec b$ such that
\begin{equation*}
  \begin{bmatrix}
  1 & 4 \\
  1 & 1 \\
  \end{bmatrix}
  \begin{bmatrix}
  \vec x
  \end{bmatrix}
  =
  \begin{bmatrix}
  3 \\
  0
  \end{bmatrix}
\end{equation*}

Following Cramer's Rule:
\begin{align*}
x_i &= \frac{\det (A_i)}{\det (A)} \\
\det(A)   &= 1 \times 1 - 1 \times 4 \\
          &= -3\\
\det(A_1) &= 3 \times 1 - 0 \times 4 \\
          &= 3 \\
\det(A_2) &= 1 \times 0 - 3 \times 1 \\
          &= -3 \\
\Rightarrow
\begin{bmatrix}
  \vec x
\end{bmatrix}
  &=
  \begin{bmatrix}
  \frac{ 3}{-3} \\
  \frac{ -3}{-3} 
  \end{bmatrix} \\
  &=
  \begin{bmatrix}
  -1 \\
  1
  \end{bmatrix}
\end{align*} 

4. Let $ A \vec x = \vec b$ such that
\begin{equation*}
  \begin{bmatrix}
  5 & -6 & 1 \\
  -1 & 3 & -4 \\
  2 & 3 & 1
  \end{bmatrix}
  \begin{bmatrix}
    \vec x
  \end{bmatrix}
  =
  \begin{bmatrix}
  4 \\
  5 \\
  -8 
  \end{bmatrix}
\end{equation*}

Following Cramer's Rule:
\begin{align*}
x_i &= \frac{\det (A_i)}{\det (A)} \\
\det (A)  &= 5 (3 \times 1 - 3 \times(-4)) 
           + 6 ((-1) \times (-4) - 2 \times 1)
           + 1 ((-1) \times 3 - 2 \times 3) \\
          &= 108 \\
\det (A_1)&= 
  \begin{vmatrix}
  4 & -6 & 1 \\
  5 & 3 & -4 \\
  -8 & 3 & 1
  \end{vmatrix} \\
  &= 4 ( 3 \times 1 - 3 \times (-4))
   + 5 ( (-6) \times 1 - 3 \times 1)
   + (-8) ( (-6) \times (-4) - 3 \times 1)\\
  &= -63  \\
\det (A_2) &=
  \begin{vmatrix}
  5 & 4 & 1 \\
  -1 & 5 & -4 \\
  2 & -8 & 1
  \end{vmatrix} \\
  &= 5 ( 5 \times 1 - (-8) \times (-4))
   + 4 ( (-1) \times 1 - 2  \times (-4))
   + 1 ( (-1) \times (-8) - 2 \times 5) \\
  &= -165 \\
\det (A_3) &=
  \begin{vmatrix}
  5 & -6 & 4 \\
  -1 & 3 & 5 \\
  2 & 3 & -8
  \end{vmatrix} \\
  &= 5 ( 3 \times (-8) - 5 \times 3)
   + (-6) ((-1) \times (-8) - 2 \times 5)
   + 4 ( (-1) \times 3 - 3 \times 2) \\
  &= -243 \\
\Rightarrow
\begin{bmatrix}
  \vec x
\end{bmatrix}
 &=
  \frac{1}{108}
  \begin{bmatrix}
  -63 \\
  -165 \\
  -243 
  \end{bmatrix} \\
  &=
  \begin{bmatrix}
  -\frac{7}{12} \\
  -\frac{55}{36} \\
  -\frac{9}{4}
  \end{bmatrix}
\end{align*}

\newpage
\question
Are the following sets of vectors linearly independent or linearly dependent.
Use the Gram method described in class.
Use that information to help determine the dimension of space spanned by the vectors.
\newline
4.
\begin{align*}
  \vec u_1 &= ( 1, 1, 0) \\
  \vec u_2 &= ( 1, 0, 1) \\
  \vec u_3 &= ( 0, 1, 1)
\end{align*}
5.
\begin{align*}
  \vec u_1 &= ( 1, 1, 1) \\
  \vec u_2 &= ( 1, 0, 1) \\
  \vec u_3 &= ( 1, 2, 1)
\end{align*}

\emph{Solution:}
\newline
4. Let $G$ have components $G_{ij} = (u_i,u_j)$ and $i,j = 1,2,3$.
\begin{equation*}
  G =
  \begin{bmatrix}
  2 & 1 & 1 \\
  1 & 2 & 1 \\
  1 & 1 & 2 
  \end{bmatrix}
\end{equation*}
The grammian is then $\det(G)$:
\begin{align*}
  \det{G} &= 
    2( 2 \times 2 - 1 \times 1) 
   -1( 1 \times 2 - 1 \times 1)
   +1( 1 \times 1 - 2 \times 2) \\
 &= 4 \neq 0
\end{align*}
Since the grammian is non-zero, then the vectors are linearly independent.
Three linearly independent vectors form a 3-space.

5. Let $G$ have components $G_{ij} = (u_i,u_j)$ and $i,j = 1,2,3$.
\begin{equation*}
  G =
  \begin{bmatrix}
  3 & 2 & 4 \\
  2 & 2 & 2 \\
  4 & 2 & 6
  \end{bmatrix}
\end{equation*}
The grammian is then $\det(G)$:
\begin{align*}
  \det (G) &= 
    3( 2 \times 6 - 2 \times 2)
   -2( 2 \times 6 - 2 \times 4)
   +6( 2 \times 2 - 2 \times 4) \\
  &= 0
\end{align*}
Since the grammian is zero, the three vectors are linearly dependent;
the dimension spanned by the space is less than 3.
Checking pairwise; let $G^{p}$ have components 
$G^{p}_{ij} = (u_i, u_j) \quad i,j=1,2,3$ and $i,j\neq p$.

\begin{align*}
G^1 &= 
  \begin{bmatrix}
  2 & 2 \\
  2 & 6
  \end{bmatrix} \\
\det (G^1) &= 8 \neq 0 \\
G^2 &= 
  \begin{bmatrix}
  3 & 4 \\
  4 & 6
  \end{bmatrix} \\
\det (G^2) &= 2 \neq 0 \\
G^3 &= 
  \begin{bmatrix}
  3 & 2 \\
  2 & 2
  \end{bmatrix} \\
\det (G^3) &= 2 \neq 0 \\
\end{align*}
Since the pairwise grammians are all non-zero, 
then the dimension of the space spanned by the vectors is 2.

\question*{\S 11.5 page 383: 2, 8, and 10}
\emph{Find:}
In each of the Problems 2, 8, and 10, 
find the inverse of the matrix or
show that the matrix is singular.
\newline
2.
\begin{equation*}
  \begin{bmatrix}
  12 & 3 \\
  4  & 1 
  \end{bmatrix}
\end{equation*}
8. 
\begin{equation*}
  \begin{bmatrix}
  -2 & 1 & -5 \\
   1 & 1 & 4 \\
   0 & 3 & 3
  \end{bmatrix}
\end{equation*}
10.
\begin{equation*}
  \begin{bmatrix}
  12 & 1 & 14 \\
  -3 & 2 & 0 \\
   0 & 9 & 14 
  \end{bmatrix}
\end{equation*}

\emph{Solution:}
\newline
2. First check that $A$ is non-singular.
\begin{align*}
  \det (A) &= 12 \times 1 - 3 \times 4 \\
           &= 0
\end{align*}
So $A$ is singular and does not have an inverse.

8. First check that $A$ is non-singular.
\begin{align*}
  \det (A) &= (-2) (1 \times 3 - 3 \times 4) 
            - 1    (1 \times 3 - 3 \times (-5))
            + 0 \\
           &= 0
\end{align*}
So $A$ is singular and does not have an inverse.

10. First check that $A$ is non-singular.
\begin{align*}
  \det (A) &= 12 (2 \times 14 - 0)
            -(-3)(1 \times 14 - 14 \times 9)
            + 0 \\
           &= 0
\end{align*}
So $A$ is singular and does not have an inverse.

\question
\emph{Find:}
\newline
Test the matrix for singularity by evaluating its determinant.
If the matrix is non-singular, compute the inverse.
\newline
9.
\begin{equation*}
  \begin{bmatrix}
  3 & 0 \\
  1 & 4 
  \end{bmatrix}
\end{equation*}

10.
\begin{equation*}
  \begin{bmatrix}
  -14 & 1 & -3 \\
  2  & -1 & 3 \\
  1 & 1 & 7 
  \end{bmatrix}
\end{equation*}

\emph{Solution:}
\newline
9. Since $A$ is triangular, $\det(A) = 3 \times 4 = 12 \neq 0$. 
So $\exists A^{-1}$ such that $AA^{-1} = I$.
\begin{equation*}
[A|I] =
  \begin{array}{cc|cc}
  3 & 0 & 1 & 0 \\
  1 & 4 & 0 & 1
  \end{array} 
\end{equation*}

\begin{align*}
\text{Multiply row 1 by $\frac{1}{3}$}
  &\Rightarrow
    \left[
    \begin{array}{cc|cc}
      1 & 0 & \frac{1}{3} & 0 \\
      1 & 4 & 0 & 1
    \end{array}  
    \right] \\
\text{Add -1 row 1 to row 2}
  &\Rightarrow
    \left[
    \begin{array}{cc|cc}
    1 & 0 & \frac{1}{3} & 0 \\
    0 & 4 & -\frac{1}{3} & 1
    \end{array} 
    \right] \\
\text{Multiply row 2 by $\frac{1}{4}$}
  &\Rightarrow 
    \left[
    \begin{array}{cc|cc}
    1 & 0 & \frac{1}{3} & 0 \\
    0 & 1 & -\frac{1}{12} & \frac{1}{4}
    \end{array} 
    \right] \\
\Rightarrow 
&A^{-1} = 
  \begin{bmatrix}
  \frac{1}{3} & 0 \\
  -\frac{1}{12} & \frac{1}{4}
  \end{bmatrix}
\end{align*}
\newpage
10. First find $\det(A)$.
\begin{align*}
  \det(A) &= (-14)((-1)\times 7 - 1 \times 3)
           -    1 ( 2  \times 7 - 1 \times 3)
           -    3 ( 2  \times 1 - (-1) \times 1) \\
          &= 120 \neq 0
\end{align*}
So $\exists A^{-1}$ such that $AA^{-1} = I$.
\begin{equation*}
[A|I] = 
  \left[
  \begin{array}{ccc|ccc}
  -14 & 1 & -3 & 1 & 0 & 0 \\
  2  & -1 & 3 & 0 & 1 & 0 \\
  1 & 1 & 7 & 0 & 0 & 1
  \end{array}
  \right]
\end{equation*}

\begin{align*}
\text{Add -2 row 3 to row 2, 14 row 3 to row 1}
  &\Rightarrow
  \left[
  \begin{array}{ccc|ccc}
  0 & 15  & 95  & 1 & 0 & 14 \\
  0  & -3 & -11 & 0 & 1 & -2 \\
  1 & 1   & 7   & 0 & 0 & 1
  \end{array}
  \right] \\
\text{Multiply row 1 by $\frac{1}{5}$}
  &\Rightarrow
  \left[
  \begin{array}{ccc|ccc}
  0 & 1  & \frac{19}{3}  & \frac{1}{15} & 0 & \frac{14}{15} \\
  0  & -3 & -11 & 0 & 1 & -2 \\
  1 & 1   & 7   & 0 & 0 & 1
  \end{array}
  \right] \\
\text{Add 3 row 1 to row 2, -1 row 1 to row 3}
  &\Rightarrow
  \left[
  \begin{array}{ccc|ccc}
  0 & 1 & \frac{19}{3}  & \frac{1}{15}  & 0 & \frac{14}{15} \\
  0 & 0 & 8             & \frac{1}{5}   & 1 & \frac{4}{5} \\
  1 & 0 & \frac{2}{3}   & -\frac{1}{15} & 0 & \frac{1}{15} 
  \end{array}
  \right] \\
\text{Multiply row 2 by $\frac{1}{8}$}
  &\Rightarrow
  \left[
  \begin{array}{ccc|ccc}
  0 & 1 & \frac{19}{3}  & \frac{1}{15}  & 0           & \frac{14}{15} \\
  0 & 0 & 1             & \frac{1}{40}  & \frac{1}{8} & \frac{1}{10} \\
  1 & 0 & \frac{2}{3}   & -\frac{1}{15} & 0           & \frac{1}{15} 
  \end{array}
  \right] \\
\text{Add $-\frac{19}{3}$ row 2 to row 1, $-\frac{2}{3}$ row 2 to row 3}
  &\Rightarrow
  \left[
  \begin{array}{ccc|ccc}
  0 & 1 & 0 & -\frac{11}{120} & -\frac{19}{24} & \frac{3}{10} \\
  0 & 0 & 1 &  \frac{1}{40}   & \frac{1}{8}    & \frac{1}{10} \\
  1 & 0 & 0 & -\frac{1}{12}   & -\frac{2}{3}   & 0
  \end{array}
  \right] \\
\text{Swap row 3 $\rightarrow$ row 1 $\rightarrow$ row 2 $\rightarrow$ row 3 }
  &\Rightarrow
  \left[
  \begin{array}{ccc|ccc}
  1 & 0 & 0 & -\frac{1}{12}   & -\frac{2}{3}   & 0 \\
  0 & 1 & 0 & -\frac{11}{120} & -\frac{19}{24} & \frac{3}{10} \\
  0 & 0 & 1 &  \frac{1}{40}   & \frac{1}{8}    & \frac{1}{10} 
  \end{array}
  \right] \\
\Rightarrow 
&A^{-1} = 
  \begin{bmatrix}
    -\frac{1}{12}   & -\frac{2}{3}   & 0 \\
    -\frac{11}{120} & -\frac{19}{24} & \frac{3}{10} \\
     \frac{1}{40}   & \frac{1}{8}    & \frac{1}{10} 
  \end{bmatrix}
\end{align*}
\newpage
\question*{\S 12.1 page 407: 4, 8, 10, and 14}
\emph{Find:}
\newline
In each of Problems 4, 8, 10, and 14, 
find the eigenvalues of the matrix.
For each eigenvalue, find an eigenvector.
\newline
4.
\begin{equation*}
  \begin{bmatrix}
  6 & -2 \\
  -3 & 4
  \end{bmatrix}
\end{equation*}
8. 
\begin{equation*}
  \begin{bmatrix}
  -2 & 1 & 0 \\
  1  & 3 & 0 \\
  0  & 0 & -1
  \end{bmatrix}
\end{equation*}
10.
\begin{equation*}
  \begin{bmatrix}
  0 & 0 & -1 \\
  0 & 0 & 1 \\
  2 & 0 & 0 
  \end{bmatrix}
\end{equation*}
14.
\begin{equation*}
  \begin{bmatrix}
  -2 & 1 & 0 & 0  \\
  1 & 0 & 0 & 1  \\
  0 & 0 & 0 & 0  \\
  0 & 0 & 0 & 0 
  \end{bmatrix}
\end{equation*}

\emph{Solution:}

4. First find solutions to $\det (\lambda I - A) = 0 $ :
\begin{align*}
  \begin{vmatrix}
  \lambda -6 & 2 \\
  3 & \lambda -4
  \end{vmatrix}
  &=  (\lambda - 6)(\lambda-4) - 2 \times 3 = 0 \\
  &= \lambda^2 - 10 \lambda +18 =0
\end{align*}
which has solutions $\lambda = 5 \pm \sqrt{7}$.
\begin{align*}
\text{When $\lambda = 5 + \sqrt{7}$ :}
  \begin{bmatrix}
  -1 + \sqrt{7} & 2 \\
  3 & 1+\sqrt{7}
  \end{bmatrix} 
  \begin{bmatrix}
  \vec e_1
  \end{bmatrix}
  &=
  \begin{bmatrix}
  \vec 0
  \end{bmatrix} \\
\Rightarrow
  \begin{bmatrix}
  \vec e_1
  \end{bmatrix}
  &=
\begin{bmatrix}
  2 \\
  - 1 + \sqrt{7}
\end{bmatrix} \\
\text{When $\lambda = 5 - \sqrt{7}$ :}
  \begin{bmatrix}
  -1 - \sqrt{7} & 2 \\
  3 & 1-\sqrt{7}
  \end{bmatrix} 
  \begin{bmatrix}
  \vec e_2
  \end{bmatrix}
  &=
  \begin{bmatrix}
  \vec 0
  \end{bmatrix} \\
\Rightarrow
  \begin{bmatrix}
  \vec e_2
  \end{bmatrix}
  &=
\begin{bmatrix}
  2 \\
  - 1 - \sqrt{7}
\end{bmatrix}
\end{align*}
\newpage
8. First find solutions to $\det (\lambda I - A) = 0 $ :
\begin{align*}
  \begin{vmatrix}
  \lambda+2 & -1 & 0 \\
  -1  & \lambda-3 & 0 \\
  0  & 0 & \lambda+1
  \end{vmatrix} 
  &= (\lambda+1)((\lambda +2)(\lambda-3) - 1) =0 \\
  &= (\lambda+1)(\lambda^2 - \lambda -7)
\end{align*}
which has solutions $\lambda = -1, \frac{1}{2} \pm \frac{\sqrt{29}}{2}$.
\begin{align*}
\text{When $\lambda = -1$:}
  \begin{bmatrix}
  1 & -1 & 0 \\
  -1  & -4 & 0 \\
  0  & 0 & 0
  \end{bmatrix} 
  \begin{bmatrix}
  \vec e_1
  \end{bmatrix}
  &= 
  \begin{bmatrix}
  \vec 0
  \end{bmatrix} \\
\Rightarrow
  \begin{bmatrix}
  \vec e_1
  \end{bmatrix}
  &= 
  \begin{bmatrix}
  0 \\
  0 \\
  1
  \end{bmatrix} \\
\text{When $\lambda = \frac{1}{2} + \frac{\sqrt{29}}{2}$:}
  \begin{bmatrix}
  \frac{3}{2}+\frac{\sqrt{29}}{2} & -1 & 0 \\
  -1  & -\frac{5}{2} + \frac{\sqrt{29}}{2} & 0 \\
  0  & 0 & \frac{3}{2}+\frac{\sqrt{29}}{2}
  \end{bmatrix} 
  \begin{bmatrix}
  \vec e_2
  \end{bmatrix}
  &= 
  \begin{bmatrix}
  \vec 0
  \end{bmatrix} \\
\Rightarrow
  \begin{bmatrix}
  \vec e_2
  \end{bmatrix}
  &= 
  \begin{bmatrix}
  2 \\
  -3 - \sqrt{29} \\
  0
  \end{bmatrix} \\
\text{When $\lambda = \frac{1}{2} - \frac{\sqrt{29}}{2}$:}
  \begin{bmatrix}
  \frac{3}{2}-\frac{\sqrt{29}}{2} & -1 & 0 \\
  -1  & -\frac{5}{2} - \frac{\sqrt{29}}{2} & 0 \\
  0  & 0 & \frac{3}{2}-\frac{\sqrt{29}}{2}
  \end{bmatrix} 
  \begin{bmatrix}
  \vec e_3
  \end{bmatrix}
  &= 
  \begin{bmatrix}
  \vec 0
  \end{bmatrix} \\
\Rightarrow
  \begin{bmatrix}
  \vec e_3
  \end{bmatrix}
  &= 
  \begin{bmatrix}
  2 \\
  -3 + \sqrt{29} \\
  0
  \end{bmatrix}
\end{align*}
\newpage
10. First find solutions to $\det (\lambda I - A) = 0 $ :
\begin{align*}
  \begin{vmatrix}
  \lambda & 0 & 1 \\
  0 & \lambda & -1 \\
  -2 & 0 & \lambda 
  \end{vmatrix}
  &= \lambda ( \lambda \lambda - 1 (-2)) =0 \\
  &= \lambda ( \lambda^2 +2)
\end{align*}
which has solutions $\lambda = 0, \pm i \sqrt{2}$.
\begin{align*}
\text{When $\lambda = 0$:}
  \begin{bmatrix}
  0 & 0 & 1 \\
  0 & 0 & -1 \\
  -2 & 0 & 0 
  \end{bmatrix} 
  \begin{bmatrix}
  \vec e_1
  \end{bmatrix}
  &=
  \begin{bmatrix}
  \vec 0
  \end{bmatrix} \\
\Rightarrow
  \begin{bmatrix}
  \vec e_1
  \end{bmatrix}
  &=
  \begin{bmatrix}
  0 \\
  1 \\  
  0
  \end{bmatrix} \\
\text{When $\lambda = i\sqrt{2}$:}
  \begin{bmatrix}
  i\sqrt{2} & 0 & 1 \\
  0 & i\sqrt{2} & -1 \\
  -2 & 0 & i\sqrt{2} 
  \end{bmatrix} 
  \begin{bmatrix}
  \vec e_2
  \end{bmatrix}
  &=
  \begin{bmatrix}
  \vec 0
  \end{bmatrix} \\
\Rightarrow
  \begin{bmatrix}
  \vec e_2
  \end{bmatrix}
  &=
  \begin{bmatrix}
  1 \\
  -1 \\
  i\sqrt{2} 
  \end{bmatrix} \\
\text{When $\lambda = -i\sqrt{2}$:}
  \begin{bmatrix}
  -i\sqrt{2} & 0 & 1 \\
  0 & -i\sqrt{2} & -1 \\
  -2 & 0 & -i\sqrt{2} 
  \end{bmatrix} 
  \begin{bmatrix}
  \vec e_3
  \end{bmatrix}
  &=
  \begin{bmatrix}
  \vec 0
  \end{bmatrix} \\
\Rightarrow
  \begin{bmatrix}
  \vec e_3
  \end{bmatrix}
  &=
  \begin{bmatrix}
  1 \\
  -1 \\
  -i\sqrt{2} 
  \end{bmatrix} 
\end{align*}
\newpage
14. First find solutions to $\det (\lambda I - A) = 0 $ :
\begin{align*}
  \begin{vmatrix}
  \lambda+2 & -1 & 0 & 0  \\
  -1 & \lambda & 0 & -1  \\
  0 & 0 & \lambda & 0  \\
  0 & 0 & 0 & \lambda 
  \end{vmatrix}
  &= \lambda^2 ( \lambda (\lambda +2) - (-1)(-1)) = 0\\
  &= \lambda^2 ( \lambda^2 + 2 \lambda -1)
\end{align*}
which has solutions $\lambda = 0, -1 \pm \sqrt{2}$.
The zero solution has multiplicity of 2.
\begin{align*}
\text{When $\lambda = 0$:}
  \begin{bmatrix}
  2 & -1 & 0 & 0  \\
  -1 & 0 & 0 & -1  \\
  0 & 0 & 0 & 0  \\
  0 & 0 & 0 & 0
  \end{bmatrix} 
  \begin{bmatrix}
  \vec e_1
  \end{bmatrix}
  &=
  \begin{bmatrix}
  \vec 0
  \end{bmatrix} \\
\Rightarrow
  \begin{bmatrix}
  \vec e_1
  \end{bmatrix}
  &=
  \begin{bmatrix}
  \beta \\
  2 \beta \\
  \alpha \\
  -\beta
  \end{bmatrix} 
\end{align*}
where $\alpha$ and $\beta$ are unknown constants. 
To satisfy orthogonality and multiplicity, choose instances of
$\alpha=1$ with $ \beta =0$ 
and
$\alpha=0$ with $ \beta =1$ 
\begin{align*}
  \begin{bmatrix}
  \vec e_1
  \end{bmatrix}
  &= 
  \begin{bmatrix}
  0 \\
  0 \\
  1 \\
  0
  \end{bmatrix}
  &
  \begin{bmatrix}
  \vec e_2
  \end{bmatrix}
  = 
  \begin{bmatrix}
  1 \\
  2 \\
  0 \\
  -1
  \end{bmatrix}
\end{align*}
\begin{align*}
\text{When $\lambda = -1 + \sqrt{2}$:}
  \begin{bmatrix}
  1+\sqrt{2} & -1 & 0 & 0  \\
  -1 & -1+\sqrt{2} & 0 & -1  \\
  0 & 0 & -1+\sqrt{2} & 0  \\
  0 & 0 & 0 &-1+\sqrt{2}
  \end{bmatrix} 
  \begin{bmatrix}
  \vec e_3
  \end{bmatrix}
  &=
  \begin{bmatrix}
  \vec 0
  \end{bmatrix} \\
\Rightarrow
  \begin{bmatrix}
  \vec e_3
  \end{bmatrix}
  &=
  \begin{bmatrix}
  1 \\
  -1 - \sqrt{2} \\
  0 \\
  0
  \end{bmatrix}  \\
\text{When $\lambda = -1 - \sqrt{2}$:}
  \begin{bmatrix}
  1-\sqrt{2} & -1 & 0 & 0  \\
  -1 & -1-\sqrt{2} & 0 & -1  \\
  0 & 0 & -1-\sqrt{2} & 0  \\
  0 & 0 & 0 &-1-\sqrt{2}
  \end{bmatrix} 
  \begin{bmatrix}
  \vec e_4
  \end{bmatrix}
  &=
  \begin{bmatrix}
  \vec 0
  \end{bmatrix} \\
\Rightarrow
  \begin{bmatrix}
  \vec e_4
  \end{bmatrix}
  &=
  \begin{bmatrix}
  1 \\
  -1 + \sqrt{2} \\
  0 \\
  0
  \end{bmatrix} 
\end{align*}



\end{document}
