\documentclass[11pt]{homework}

\newcommand{\hwname}{Justin L. Clough}
\newcommand{\hwemail}{jlclough@usc.edu}
\newcommand{\hwtype}{Homework}
\newcommand{\hwnum}{1}
\newcommand{\hwclass}{AME: 525}

\begin{document}
\maketitle

\question*{ \S 10.2 Page 327: 2, 4, 6}
\emph{Find:}
\newline
For each of the following compute the dot product and cosine of the 
angle between the vectors. 
Determine if the vectors are orthogonal.
\newline
\newline
2. $\vec u = 2i - 6j + k, \quad \vec v =i-j$
\newline
4. $\vec u = 8i - 3j +2k, \quad \vec v = -8i -3j +k$
\newline
6. $\vec u = i + j + 2k,  \quad \vec v = i -j +2k$

\emph{Solution}
\newline
2. Dot product:

\begin{align*}
(\vec u, \vec v) &= \sum_{i=1}^{3} u_i v_i \\
       &= 2 * 1 +(-6) * (-1) + 1 *0 \\
       &= 8
\end{align*}

Cosine of angle:

\begin{align*}
\cos(\theta) &= \frac{ (\vec u, \vec v)}{  ||\vec u|| \cdot ||\vec v||} \\
             &= \frac{ 8}{  (\sqrt{2^2 + 6^2 + 1^2}) (\sqrt{1^2 + 1^2})} \\
             &= \frac{ 8}{  (\sqrt{41}) (\sqrt{2}) } \\
             &\approx 0.883
\end{align*}

The vectors are \textbf{not} orthogonal.


4. Dot product:

\begin{align*}
(\vec u, \vec v) &= \sum_{i=1}^{3} u_i v_i \\
       &= 8 * (-8) + (-3) * (-3) + 2 *1 \\
       &= -53
\end{align*}

Cosine of angle:
 
\begin{align*}
\cos(\theta) &= \frac{ (\vec u, \vec v)}{  ||\vec u|| \cdot ||\vec v||} \\
             &= \frac{ -53}{  (\sqrt{8^2 + 3^2 + 2^2}) (\sqrt{8^2 + 3^2 + 1^2})} \\
             &= \frac{ -53}{  (\sqrt{77}) (\sqrt{74}) } \\
             &\approx  -0.702
\end{align*}

The vectors are \textbf{not} orthogonal.

6. Dot product:

\begin{align*}
(\vec u, \vec v) &= \sum_{i=1}^{3} u_i v_i \\
            &= 1 * 1 + 1 * (-1) + 2 *2 \\
            &= 4
\end{align*}

Cosine of angle:

\begin{align*}
\cos(\theta) &= \frac{ (\vec u, \vec v)}{  ||\vec u|| \cdot ||\vec v||} \\
             &= \frac{ 4}{  (\sqrt{1^2 + 1^2 + 2^2}) (\sqrt{1^2 + 1^2 + 2^2})} \\
             &= \frac{ 4}{  (\sqrt{6}) (\sqrt{6}) } \\
             &\approx 0.667
\end{align*}

The vectors are \textbf{not} orthogonal.

\newpage
\question*{ \S 10.4 Page 338: 2, 4, 6, 18 (no basis)}
\emph{Find:}
\newline
Determine whether $S$ is a subspace of $\mathcal{R}^n$.
\newline
\newline
2. $S$ consists of all vectors in $\mathcal{R}^6$ with zero third and fifth components.
\newline
4. $S$ consists of all vectors in $\mathcal{R}^8$ with length less than one.
\newline
6. $S$ consists of all vectors in $\mathcal{R}^5$ having equal first and fourth components.
\newline
18.$S$ consists of all vectors in $\mathcal{R}^4$ with $<x, y, 2x, 3y>$.

\emph{Solution:}
\newline

2. Let $\vec x \in S$ such that $\vec x=(a, b, 0, d, 0, e) | a,b,d,e\in\mathcal{R}$.
Check against requirements:
\begin{enumerate}
  \item $0 \in S$? 
        Choose $a=b=d=e=0$ such that $\vec x = ( 0, 0, 0,0,0,0) = \vec 0$
  \item Sum of two vectors in $S$ remains in $S$? 
        Let $\vec y \in S$ such that 
         $\vec y=(f, g, 0, h, 0, m)|f,g,h,m \in \mathcal{R}$ and $\vec z = \vec x + \vec y$. 
         Then, $\vec z = ( a+f, b+g, 0, d+h, 0, e+m) \in S$.
  \item Product of any vector in $S$ with any real scalar is real? 
        Let $k \in \mathcal{R}$ and
         let $\vec u = k\vec x$. Then, $\vec u = ( ka, kb, 0, kd, 0, ke)$ where $ka, kb, kd, ke \in \mathcal{R}$.
\end{enumerate}

\textbf{Yes}, $S \subset \mathcal{R}^n$.
\newline
\newline
4. Let $\vec x \in S$ such that $\vec x=(x_1, x_2, x_3, x_4, x_5, x_6, x_7, x_8)| \quad ||\vec x||<1, x_i \in \mathcal{R} \quad \forall i=1,2,...8$.
Choose $x_1 = \frac{2}{3}$ and all other components as zero. Let $\vec y = \vec x + \vec x$; since $||y|| = \frac{4}{3} > 1$ then $\vec y \not\in S$.
So \textbf{no},  $S \not \subset \mathcal{R}^n$.
\newline
\newline
6. Let $\vec x \in S$ such that $\vec x=(a, b, c, a, d) | \quad a,b,c,d \in \mathcal{R}$.
Check against requirements:
\begin{enumerate}
  \item $0 \in S$? 
        Choose $a=b=c=d=0$ such that $\vec x = ( 0, 0,0,0,0) = \vec 0$
  \item Sum of two vectors in $S$ remains in $S$? 
        Let $\vec y \in S$ such that 
         $\vec y=(f,g, h, f, m)|f,g,h,m \in \mathcal{R}$ and $\vec z = \vec x + \vec y$. 
         Then, $\vec z = ( a+f, b+g, c+h, a+f, d+m) \in S$.
  \item Product of any vector in $S$ with any real scalar is real? 
        Let $k \in \mathcal{R}$ and
         let $\vec u = k\vec x$. Then, $\vec u = ( ka, kb, kc, ka, kd)$ where $ka, kb, kc, kd \in \mathcal{R}$.
\end{enumerate}
\textbf{Yes}, $S \subset \mathcal{R}^n$.

\newpage
18. Let $u \in S$ such that $\vec u = (x, y, 2x, 3y) | \quad x,y\in\mathcal{R}$.
Check against requirements:
\begin{enumerate}
  \item $0 \in S$? 
        Choose $x=y=0$ such that $\vec u = ( 0,0,0,0) = \vec 0$
  \item Sum of two vectors in $S$ remains in $S$? 
        Let $v \in S$ such that 
         $\vec v=(\zeta, \eta, 2\zeta, 3\eta)| \quad \zeta, \eta\in\mathcal{R}$ and $\vec w=\vec u+\vec v$.
         Then $\vec w = (x+\zeta, y+\eta, 2(x+\zeta), 3(y+\eta)) \in S$.
  \item Product of any vector in $S$ with any real scalar is real? 
        Let $k \in \mathcal{R}$ and
         let $\vec z = k\vec u$. Then, $\vec z = ( kx, ky, 2kx, 3ky)$ where $kx, ky\in \mathcal{R}$.
\end{enumerate}
\textbf{Yes}, $S \subset \mathcal{R}^n$.

\question*{Question (8)}
\emph{Find:}
The Hermitian product and angle for:
\newline
$ u = (1+2i, 1-i) \quad v = (1-i, 2-i)$.

\emph{Solution:}
\newline

Hermitian product:

\begin{align*}
(\overline{u}, v) &= \sum_{i=1}^{2} \overline{u_i} v_i \\
     &= (1-2i)*(1-i) + (1+i)*(2-i) \\
     &= 2 - 2i
\end{align*}

\newpage
Hermitian angle:

\begin{equation*}
cos(\theta_H) = \frac{ (\overline{u},v) + (u, \overline{v})}{2 l_H(u) l_H(v)}
\end{equation*}

\begin{align*}
l_H(u) &= \sqrt{(\overline{u},u)}  \\
       &= \sqrt{ (1-2i)*(1+2i) + (1+i)*(1-i)} \\
       &= \sqrt{7 } 
\end{align*}

\begin{align*}
l_H(v) &= \sqrt{(\overline{v},v)}  \\
       &= \sqrt{ (1+i)*(1-i) + (2+i)*(2-i)} \\
       &= \sqrt{7 } 
\end{align*}

\begin{align*}
(u, \overline{v}) &= \overline{( \overline{u}, v)} \\
                  &= \overline{ 2 -2i} \\
                  &= 2+2i
\end{align*}

\begin{align*}
cos(\theta_H) &= \frac{ (\overline{u},v) + (u, \overline{v})}{2 l_H(u) l_H(v)} \\
              &= \frac{ (2 - 2i)+ (2+2i)}{2 \sqrt{7} \sqrt{7}} \\
              &= \frac{1}{7}\\
\Rightarrow \theta_H &\approx 73.4^{\circ}
\end{align*}

\end{document}

