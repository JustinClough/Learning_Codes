\documentclass[11pt]{homework}

\newcommand{\hwname}{Justin L. Clough}
\newcommand{\hwemail}{jlclough@usc.edu}
\newcommand{\hwtype}{Homework}
\newcommand{\hwnum}{6}
\newcommand{\hwclass}{AME: 525}

\usepackage{amsmath}
 \usepackage{delarray}

\begin{document}
\maketitle

\question*{\S 12.2 page 413: 2,6,8, and 10}
\emph{Find:}
\newline
In each of Problems 2, 6, 8, and 10, 
find a matrix $P$ that diagonalizes the given matrix,
or show that this matrix is not diagonalizable.

2.
\begin{equation*}
  A = 
  \begin{bmatrix}
  5 & 3 \\
  1 & 3 
  \end{bmatrix}
\end{equation*}

6.
\begin{equation*}
  A = 
  \begin{bmatrix}
  0 & 0 & 0 \\
  1 & 0 & 2 \\
  0 & 1 & 3 
  \end{bmatrix}
\end{equation*}

8.
\begin{equation*}
  A = 
  \begin{bmatrix}
  2 & 0 & 0 \\
  0 & 2 & 1 \\
  0 & -1 & 2 
  \end{bmatrix}
\end{equation*}

10.
\begin{equation*}
  A = 
  \begin{bmatrix}
  -2 & 0 & 0 &0 \\
  -4 & -2 & 0 &0 \\
  0 & 0 & -2 &0 \\
  0 & 0 & 0 &-2 
  \end{bmatrix}
\end{equation*}

\emph{Solution:}
\newline
2. First find eigenpairs, $\{\lambda,\vec e\}$ of $A$.
\begin{align*}
\det (\lambda I - A) = 
  \begin{vmatrix}
  \lambda -5 & -3 \\
  -1         & \lambda - 3
  \end{vmatrix}
  &= 0 \\
  &= (\lambda -5)(\lambda - 3) - (-1)(-3) \\
  &= \lambda^2 - 8 \lambda + 12 \\
  &= (\lambda-6)(\lambda-2) =0  \\
\end{align*}
which has solution of $\lambda = 6,2$.
The corresponding eigenvectors are:
\begin{align*}
\text{When $\lambda=6$:}
  \begin{bmatrix}
  1 & -3 \\
  -1         & 3
  \end{bmatrix}
  \begin{bmatrix}
    \vec e_1
  \end{bmatrix}
  &=
  \begin{bmatrix}
    \vec 0
  \end{bmatrix} \\
  \Rightarrow
  \begin{bmatrix}
    \vec e_1
  \end{bmatrix}
  &=
  \begin{bmatrix}
  1 \\
  \frac{1}{3}
  \end{bmatrix} \\
\text{When $\lambda=2$:}
  \begin{bmatrix}
  -3 & -3 \\
  -1         & -1
  \end{bmatrix}
  \begin{bmatrix}
    \vec e_2
  \end{bmatrix}
  &=
  \begin{bmatrix}
    \vec 0
  \end{bmatrix} \\
  \Rightarrow
  \begin{bmatrix}
    \vec e_2
  \end{bmatrix}
  &=
  \begin{bmatrix}
  1 \\
  -1
  \end{bmatrix}
\end{align*}

Form $P$ such that $P=\{\vec e_1, \vec e_2\}$:
\begin{equation*}
  P = 
  \begin{bmatrix}
  1 & 1 \\
  \frac{1}{3} & -1
  \end{bmatrix}
\end{equation*}

The inverse of $P$ is then:
\begin{align*}
[P|I] &= 
  \left[
  \begin{array}{cc|cc}
  1           &  1 & 1 & 0 \\
  \frac{1}{3} & -1 & 0 & 1
  \end{array} 
  \right] \\
\text{Add $-\frac{1}{3}$ row 1 to row 2} 
  & \Rightarrow
  \left[
  \begin{array}{cc|cc}
  1  &  1           & 1 & 0 \\
  0  & -\frac{4}{3} & -\frac{1}{3} & 1
  \end{array}
  \right] \\
\text{Multiply row 2 by $-\frac{3}{4}$}
  & \Rightarrow
  \left[
  \begin{array}{cc|cc}
  1  & 1           & 1 & 0 \\
  0  & 1 & \frac{1}{4} & -\frac{3}{4}
  \end{array} 
  \right] \\
\text{Add -1 row 2 to row 1}
  & \Rightarrow
  \left[
  \begin{array}{cc|cc}
  1  & 0 & \frac{3}{4} & \frac{3}{4} \\
  0  & 1 & \frac{1}{4} & -\frac{3}{4}
  \end{array} 
  \right]
  = [I|P^{-1}]
\end{align*}

Evaluating $P^{-1}AP$:
\begin{align*}
P^{-1}AP  &=
  \begin{bmatrix}
    \frac{3}{4} & \frac{3}{4} \\
    \frac{1}{4} & -\frac{3}{4}
  \end{bmatrix} 
  \begin{bmatrix}
    5 & 3 \\
    1 & 3 
  \end{bmatrix}
  \begin{bmatrix}
    1 & 1 \\
    \frac{1}{3} & -1
  \end{bmatrix} \\
  &=
  \begin{bmatrix}
    6 & 0 \\
    0 & 2
  \end{bmatrix}
\end{align*}

6. First find eigenpairs, $\{\lambda, \vec e\}$ of A.
\begin{align*}
\det (\lambda I - A) =
  \begin{vmatrix}
    \lambda & 0 & 0 \\
    -1      & \lambda & -2 \\
    0 & -1  & \lambda - 3 
  \end{vmatrix}
  &= 0 \\
  &= \lambda ( \lambda (\lambda-3) - (-1)(-2)) \\
  &= \lambda ( \lambda^2 - 3 \lambda - 2)
\end{align*}
which has solutions $\lambda = 0, \frac{3}{2}\pm\frac{\sqrt{17}}{2}$.
The corresponding eigenvectors are:
\begin{align*}
\text{When $\lambda = 0$ :}
  \begin{bmatrix}
    0 & 0 & 0 \\
    -1 & 0 & -2 \\
    0 & -1 & -3
  \end{bmatrix}
  \begin{bmatrix}
    \vec e_1
  \end{bmatrix}
  &=
  \begin{bmatrix}
    \vec 0
  \end{bmatrix} \\
\Rightarrow
  \begin{bmatrix}
    \vec e_1
  \end{bmatrix}
  &=
  \begin{bmatrix}
    1 \\
    \frac{3}{2} \\
    -\frac{1}{2}
  \end{bmatrix} \\
\text{When $\lambda = \frac{3}{2} + \frac{\sqrt{17}}{2}$ :}
  \begin{bmatrix}
    \frac{3}{2} + \frac{\sqrt{17}}{2} & 0 & 0 \\
    -1 &  \frac{3}{2} + \frac{\sqrt{17}}{2} & -2 \\
    0  & -1 & -\frac{3}{2} + \frac{\sqrt{17}}{2}
  \end{bmatrix}
  \begin{bmatrix}
    \vec e_2
  \end{bmatrix}
  &=
  \begin{bmatrix}
    \vec 0
  \end{bmatrix} \\
\Rightarrow
  \begin{bmatrix}
    \vec e_2
  \end{bmatrix}
  &=
  \begin{bmatrix}
    0 \\
    4 \\
    3 + \sqrt{17}
  \end{bmatrix} \\
\text{When $\lambda = \frac{3}{2} - \frac{\sqrt{17}}{2}$ :}
  \begin{bmatrix}
    \frac{3}{2} - \frac{\sqrt{17}}{2} & 0 & 0 \\
    -1 &  \frac{3}{2} - \frac{\sqrt{17}}{2} & -2 \\
    0  & -1 & -\frac{3}{2} - \frac{\sqrt{17}}{2}
  \end{bmatrix}
  \begin{bmatrix}
    \vec e_3
  \end{bmatrix}
  &=
  \begin{bmatrix}
    \vec 0
  \end{bmatrix} \\
\Rightarrow
  \begin{bmatrix}
    \vec e_3
  \end{bmatrix}
  &=
  \begin{bmatrix}
    0 \\
    4 \\
    3 - \sqrt{17}
  \end{bmatrix}
\end{align*}

Form $P$ such that $P=\{\vec e_1, \vec e_2, \vec e_3\}$:
\begin{equation*}
  P =
  \begin{bmatrix}
    1            & 0             & 0             \\
    \frac{3}{2}  & 4             & 4             \\
    -\frac{1}{2} & 3 + \sqrt{17} & 3 - \sqrt{17}
  \end{bmatrix}
\end{equation*}

The inverse of $P$ is then:
\begin{align*}
[P|I]& = 
  \left[
  \begin{array}{ccc|ccc}
    1            & 0             & 0             & 1 & 0 & 0 \\
    \frac{3}{2}  & 4             & 4             & 0 & 1 & 0 \\
    -\frac{1}{2} & 3 + \sqrt{17} & 3 - \sqrt{17} & 0 & 0 & 1 
  \end{array}
  \right] \\
\text{Add $-\frac{3}{2}$ row 1 to row 2, $\frac{1}{2}$ row 1 to row 3}
  & \Rightarrow
  \left[
  \begin{array}{ccc|ccc}
    1 & 0             & 0             & 1 & 0 & 0 \\
    0 & 4             & 4             & -\frac{3}{2} & 1 & 0 \\
    0 & 3 + \sqrt{17} & 3 - \sqrt{17} & \frac{1}{2} & 0 & 1 
  \end{array}
  \right] \\
\text{Multiply row 2 by $\frac{1}{4}$}
  & \Rightarrow
  \left[
  \begin{array}{ccc|ccc}
    1 & 0             & 0             & 1 & 0 & 0 \\
    0 & 1             & 1             & -\frac{3}{8} & \frac{1}{4} & 0 \\
    0 & 3 + \sqrt{17} & 3 - \sqrt{17} & \frac{1}{2} & 0 & 1 
  \end{array}
  \right] \\
\text{Add $-(3+\sqrt{17})$ row 2 to row 3}
  & \Rightarrow
  \left[
  \begin{array}{ccc|ccc}
    1 & 0 & 0             & 1                       & 0 & 0 \\
    0 & 1 & 1             & -\frac{3}{8}            & \frac{1}{4} & 0 \\
    0 & 0 & -2\sqrt{17} & \frac{13+3\sqrt{17}}{8} & -\frac{3+\sqrt{17}}{4} & 1 
  \end{array}
  \right] \\
\text{Multiply row 3 by $-\frac{1}{2\sqrt{17}}$ }
  & \Rightarrow
  \left[
  \begin{array}{ccc|ccc}
    1 & 0 & 0 & 1                       & 0 & 0 \\
    0 & 1 & 1 & -\frac{3}{8}            & \frac{1}{4} & 0 \\
    0 & 0 & 1 & -\frac{13+3\sqrt{17}}{16\sqrt{17}} & \frac{3+\sqrt{17}}{8\sqrt{17}} & -\frac{1}{2\sqrt{17}}
  \end{array}
  \right] \\
\text{Add -1 row 3 to row 2}
  & \Rightarrow
  \left[
  \begin{array}{ccc|ccc}
    1 & 0 & 0 & 1                         & 0 & 0 \\
    0 & 1 & 0 & -\frac{13-3\sqrt{17}}{16} & \frac{-3 + \sqrt{17}}{8\sqrt{17}} & \frac{1}{2\sqrt{17}} \\
    0 & 0 & 1 & -\frac{13+3\sqrt{17}}{16\sqrt{17}} & \frac{3+\sqrt{17}}{8\sqrt{17}} & -\frac{1}{2\sqrt{17}}
  \end{array}
  \right] = [I|P^{-1}]
\end{align*}

Evaluaing $P^{-1}AP$:
\begin{align*}
P^{-1}AP &= 
  \begin{bmatrix}
    1                         & 0 & 0 \\
    -\frac{13-3\sqrt{17}}{16} & \frac{-3 + \sqrt{17}}{8\sqrt{17}} & \frac{1}{2\sqrt{17}} \\
    -\frac{13+3\sqrt{17}}{16\sqrt{17}} & \frac{3+\sqrt{17}}{8\sqrt{17}} & -\frac{1}{2\sqrt{17}}
  \end{bmatrix}
  \begin{bmatrix}
    0 & 0 & 0 \\
    1 & 0 & 2 \\
    0 & 1 & 3 
  \end{bmatrix}
  \begin{bmatrix}
    1            & 0             & 0             \\
    \frac{3}{2}  & 4             & 4              \\
    -\frac{1}{2} & 3 + \sqrt{17} & 3 - \sqrt{17} 
  \end{bmatrix} \\
  &=
  \begin{bmatrix}
    0 & 0 & 0 \\
    0 & \frac{3}{2} - \frac{\sqrt{17}}{2} & 0 \\
    0 & 0 & \frac{3}{2} + \frac{\sqrt{17}}{2}
  \end{bmatrix}
\end{align*}

8. First find eigenpairs,$ \{\lambda, \vec e\} $, of $A$.
\begin{align*}
\det (\lambda I - A) = 
  \begin{vmatrix}
    \lambda -2 & 0 & 0 \\
    0 & \lambda -2 & -1 \\
    0 & 1 & \lambda -2 
  \end{vmatrix} 
  &= 0 \\
  &= (\lambda - 2) ( (\lambda-2)^2 - (-1)(1)) \\
  &= (\lambda - 2) (\lambda^2 - 4 \lambda + 5) =0
\end{align*}
which has solutions $\lambda=2, 2 \pm i$. 
The corresponding eigenvectors are:
\begin{align*}
\text{When $\lambda = 2$:}
  \begin{bmatrix}
    0 & 0 & 0 \\
    0 & 0 & -1 \\
    0 & 1 & 0 
  \end{bmatrix}
  \begin{bmatrix}
    \vec e_1
  \end{bmatrix}
  &=
  \begin{bmatrix}
    \vec 0
  \end{bmatrix} \\
\Rightarrow
  \begin{bmatrix}
    \vec e_1
  \end{bmatrix}
  &=
  \begin{bmatrix}
  1 \\
  0 \\
  0
  \end{bmatrix} \\
\text{When $\lambda = 2+i$:}
  \begin{bmatrix}
    i & 0 & 0 \\
    0 & i & -1 \\
    0 & 1 & i
  \end{bmatrix}
  \begin{bmatrix}
    \vec e_2
  \end{bmatrix}
  &=
  \begin{bmatrix}
    \vec 0
  \end{bmatrix} \\
\Rightarrow
  \begin{bmatrix}
    \vec e_2
  \end{bmatrix}
  &=
  \begin{bmatrix}
    0 \\
    1 \\
    i
  \end{bmatrix} \\
\text{When $\lambda = 2-i$:}
  \begin{bmatrix}
    -i & 0 & 0 \\
    0 & -i & -1 \\
    0 & 1 & -i
  \end{bmatrix}
  \begin{bmatrix}
    \vec e_3
  \end{bmatrix}
  &=
  \begin{bmatrix}
    \vec 0
  \end{bmatrix} \\
\Rightarrow
  \begin{bmatrix}
    \vec e_3
  \end{bmatrix}
  &=
  \begin{bmatrix}
    0 \\
    1 \\
    -i
  \end{bmatrix} \\
\end{align*}

Form $P$ such that $P = \{\vec e_1, \vec e_2, \vec e_3\}$:
\begin{equation*}
P =
  \begin{bmatrix}
    1 & 0 & 0 \\
    0 & 1 & 1 \\
    0 & i & -i
  \end{bmatrix}
\end{equation*}

The inverse of $P$ is then:
\begin{align*}
[P|I] &= 
  \left[
  \begin{array}{ccc|ccc}
    1 & 0 & 0 & 1 & 0 & 0 \\
    0 & 1 & 1 & 0 & 1 & 0 \\
    0 & i & -i & 0 & 0 & 1 
  \end{array}
  \right] \\
\text{Add $-i$ row 2 to row 3}
  &\Rightarrow
  \left[
  \begin{array}{ccc|ccc}
    1 & 0 & 0   & 1 & 0  & 0 \\
    0 & 1 & 1   & 0 & 1  & 0 \\
    0 & 0 & -2i & 0 & -i & 1 
  \end{array}
  \right] \\
\text{Multiply row 3 by $-\frac{1}{2i}$}
  &\Rightarrow
  \left[
  \begin{array}{ccc|ccc}
    1 & 0 & 0 & 1 & 0  & 0 \\
    0 & 1 & 1 & 0 & 1  & 0 \\
    0 & 0 & 1 & 0 & \frac{1}{2} & -\frac{1}{2i}
  \end{array}
  \right] \\
\text{Add -1 row 3 to row 2}
  &\Rightarrow
  \left[
  \begin{array}{ccc|ccc}
    1 & 0 & 0 & 1 & 0  & 0 \\
    0 & 1 & 0 & 0 & \frac{1}{2}  & \frac{1}{2i} \\
    0 & 0 & 1 & 0 & \frac{1}{2} & -\frac{1}{2i}
  \end{array}
  \right]  = [I|P^{-1}]
\end{align*}

Evaluating $P^{-1}AP$:
\begin{align*}
P^{-1}AP = 
  \begin{bmatrix}
    0  & 0 \\
    \frac{1}{2}  & \frac{1}{2i} \\
    \frac{1}{2} & -\frac{1}{2i}
  \end{bmatrix}
  \begin{bmatrix}
    2 & 0 & 0 \\
    0 & 2 & 1 \\
    0 & -1 & 2 
  \end{bmatrix}
  \begin{bmatrix}
    1 & 0 & 0  \\
    0 & 1 & 1  \\
    0 & i & -i
  \end{bmatrix} \\
  &=
  \begin{bmatrix}
    2 & 0 & 0 \\
    0 & 2+i & 0 \\
    0 & 0 & 2-i
  \end{bmatrix}
\end{align*}

10. First find eigenpairs, $\{ \lambda, \vec e \} $, of A.
\begin{align*}
\det( \lambda I - A) = 
  \begin{vmatrix}
    \lambda+2 & 0 & 0 & 0 \\
    4  & \lambda+2 & 0 & 0\\
    0 & 0 & \lambda+2 & 0 \\
    0 & 0 & 0 & \lambda+2
  \end{vmatrix}
  & = 0 \\
  & = (\lambda + 2)^4 
\end{align*}
which has eigenvalues $\lambda = -2$ with multiplicity of 4.
The corresponding eigenvectors are:
\begin{align*}
\text{When $\lambda = 2$:}
  \begin{bmatrix}
  0 & 0 & 0 & 0 \\
  4 & 0 & 0 & 0 \\
  0 & 0 & 0 & 0 \\
  0 & 0 & 0 & 0 
  \end{bmatrix}
  \begin{bmatrix}
    \vec e_1 
  \end{bmatrix}
  &= 
  \begin{bmatrix}
    \vec 0
  \end{bmatrix} \\ 
  \begin{bmatrix}
    \vec e_1 
  \end{bmatrix}
  &= 
  \begin{bmatrix}
    0 \\
    \alpha \\
    \beta \\
    \gamma 
  \end{bmatrix} 
\end{align*}
where $\alpha $, $\beta$, and $\gamma$ are unknown constants.
To satisfy orthogonality between eigenvectors, 
choose instances of:
$\alpha = 1$, $\beta = \gamma = 0$;
$\alpha = \beta = 0$, $ \gamma = 1$;
and
$\alpha = \gamma = 0$, $\beta = 1$.
This yeilds only three eigenvectors but four 
are needed to diagonailze $A$.
$\therefore$ $A$ is not diagonalizable.

\newpage
\question*{ \S 12.2 page 413: 11}
\emph{Find:}
\newline
Let $A$ have eigenvalues $\lambda_1,...,\lambda_n$
and suppose that $P$ diagonalizes $A$.
Show that for any positive integer $k$, 
$P$ diagonalizes $A^k$ and determine $P^{-1}A^kP$.

\emph{Solution:}
\newline
Let $P^{-1}AP = D$ where $D$ is the diagonal matrix 
with components $\lambda_i$ where $i=1,2,...,n$.
Let $P^{-1} A^k P = M$ and want to show 
$M$ is diagonal.

\begin{align*}
P^{-1} A A A ... A P &= M \\
P^{-1} A I A I A ...  A I A P &= M \\
  I &= P P^{-1} \\
P^{-1} A P P^{-1}  A P P^{-1}  A P  ...  P^{-1} A P P^{-1}  A P &= M \\
(P^{-1} A P)( P^{-1}  A P)( P^{-1}  A P)...  (P^{-1} A P)( P^{-1}  A P) &= M \\
D D D ...  D &= M
\end{align*}

So $M = D^k$ which is diagonal. Its $i$th element is $\lambda_i^k$.

\newpage
\question*{ \S 12.2 page 413: 16}
\emph{Find:}
\newline
Show that, if $A^2$ is diagonalizable, so is $A$.

\emph{Solution:}
\newline
Assume $\exists P$ which diagonalizes $A^2$ such that $P^{-1}A^2P=D$
and $D$ is the diagonal matrix with the squared eigenvalues of $A$ as
elements.
Let $P^{-1}AP=G$; want to show G is diagonal.

\begin{align*}
P^{-1} A^2 P &= D \\
P^{-1} A A P &= D \\
P^{-1} A I A P &= D \\
  I &= P P^{-1} \\
P^{-1} A P P^{-1} A P &= D \\
(P^{-1} A P) ( P^{-1} A P) &= D \\
  G G &= D
\end{align*}

With indicial notation:
let $G$ have rows $\vec r_i$ and columns $\vec c_k$ where $i,k=1,2,... \text{rank}(A)$;
let $D$ have components $D_{ij}$.
Note $D^T=D \Rightarrow G^T=G$, so $\vec r_i = \vec c_i$.

\begin{align*}
GG=D \Rightarrow ( \vec r_i, \vec r_j) &= D_{ij} \\
  &= 
  \left\{
  \begin{array}{cc}
    0 & \text{if } i \neq j \\
    \lambda_i^2 & \text{if } i = j
  \end{array}
  \right.
\end{align*}

where $\lambda_i$ is the $i$th eigenvalue of $A$.
For $G$ to satisfy the above conditions it must be diagonal 
with elements $\lambda_i$.

\end{document}
