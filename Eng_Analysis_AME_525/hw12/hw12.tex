\documentclass[11pt]{homework}

\newcommand{\hwname}{Justin L. Clough}
\newcommand{\hwemail}{jlclough@usc.edu}
\newcommand{\hwtype}{Homework}
\newcommand{\hwnum}{12}
\newcommand{\hwclass}{AME: 525}

\usepackage{amsmath}
 \usepackage{delarray}
\usepackage{physics}

\begin{document}
\maketitle

\question*{\S 22.1 page 717: 2, 8, 12}
\emph{Find:}
\newline
In each of problems 2, 8, and 12,
determine all singularities of the function
and classify each singularity. 
In the case of a pole, give the order of the pole.

2.
\begin{equation*}
\frac{4 \sin( z+2)}{(z+i)^2 (z-i)}
\end{equation*}

8.
\begin{equation*}
\frac{\sin(z)}{\sinh(z)}
\end{equation*}

12.
\begin{equation*}
e^{\frac{1}{z(z+1)}}
\end{equation*}

\emph{Solution:}
\newline
2. Let $f(z) =\frac{g(z)}{h(z)}$, 
$h(z)=(z+i)^2 (z-i)$ has 
a zero of order 2 at $-i$ and 
order 1 at $+i$;
$g(z)$ has a zero at $z = -2 + n\pi (1+i)$ $\forall n=0,\pm 1 \pm 2,...$.
So $f(z)$ has an order 2 pole at $z=-i$ and order 1 at $z=+i$.

8. Let $f(z) =\frac{g(z)}{h(z)}$, 
$g(z) = \sin(z)$ has zeros at $z=m\pi$ $\forall m=0,\pm 1, \pm 2, ...$;
$h(z) = \sinh(z)$ has zeros at $z=i n\pi$ $\forall n=0,\pm 1, \pm 2, ...$.
So $f(z)$ has order 1 poles at $z=i n\pi$ $\forall n=0,\pm 1, \pm 2, ...$.

12.  The Laurent expansion about 0 is:
\begin{equation*}
e^{\frac{1}{z(z+1)}} = \sum_{n=0}^{\infty} \frac{1}{n!} \left( \frac{1}{z(z+1)}\right)^n
\end{equation*}
So there is an essential singularity at $z=0,-1$.

\newpage
\question*{\S 22.2 page 723: 2, 4, 8, 10, 14}
\emph{Find:}
\newline
In each of problems 2, 4, 8, 10, and 14,
use the residue theorem to evaluate the integral.

2. $\gamma = \{z | z=1+3e^{it} \forall t\in[0,2\pi]\}$ and
\begin{equation*}
J = \int_\gamma \frac{2z}{(z-i)^2} dz
\end{equation*}

4. $\gamma = \{z |  \left\Vert z+2i \right\Vert_{L_\infty}=\frac{3}{2} \}$ and
\begin{equation*}
J = \int_\gamma \frac{\cos(z)}{4+z^2} dz
\end{equation*}

8. $\gamma = \{z | z=\frac{i}{8}+\frac{1}{2}e^{it} \forall t\in[0,2\pi]\}$ and
\begin{equation*}
J = \int_\gamma \frac{\cos(z)}{ze^z} dz
\end{equation*}

10.  $\gamma = \{z |  \left\Vert z+i \right\Vert_{L_\infty}=\frac{3}{2} \}$ and
\begin{equation*}
J = \int_\gamma e^{\frac{2}{z^2}} dz
\end{equation*}

14.  $\gamma = \{z | z=2+2e^{it} \forall t\in[0,2\pi]\}$ and
\begin{equation*}
J = \int_\gamma \frac{(1-z)^2}{z^3-8} dz
\end{equation*}

\emph{Solution:}
\newline
2. By residue theory for order 2 pole at $z=i$:
\begin{align*}
J &=  \int_\gamma \frac{2z}{(z-i)^2} dz \\
  &= 2i\pi \Res(f(z), i) \\
  &= 2i\pi \left( \frac{1}{1}\lim_{z\to i} \frac{(z-i)^2 2z}{(z-i)^2}  \right) \\
  &= 2i\pi \left( \lim_{z\to i} 2z \right) \\
  &= 2i\pi \frac{2i}{2} \\
  &= -4\pi
\end{align*}

4. The function $f(z)$ has 2 order 1 poles at $z=\pm2i$ but 
only $z=-2i \in \bar{\gamma}$. 
So, by the residue theorem for order 1 poles:
\begin{align*}
J &= \int_\gamma \frac{\cos(z)}{4+z^2} dz \\
  &= 2i\pi \left( \lim_{z\to -2i} \frac{(z+2i)\cos(z)}{4+z^2} \right) \\
  &= 2i\pi \left( \lim_{z\to -2i} \frac{(z+2i)\cos(z)}{(z-2i)(z+2i)} \right) \\
  &= 2i\pi \left( \lim_{z\to -2i} \frac{\cos(z)}{(z-2i)} \right) \\
  &= 2i\pi \left( \frac{\cos(-2i)}{(-2i-2i)} \right) \\
  &= 2i\pi \left( \frac{\cos(-2i)}{(-4i)} \right) \\
  &= \frac{-\pi \cos(2i)}{2}
\end{align*}

8. By residue theory for order 1 pole at $z=0$:
\begin{align*}
J &= \int_\gamma \frac{\cos(z)}{ze^z} dz \\
  &= 2 i \pi \left(\lim_{z\to 0} \frac{z\cos(z)}{ze^z} \right) \\
  &= 2 i \pi \left(\lim_{z\to 0} \frac{\cos(z)}{e^z} \right) \\
  &= 2 i \pi \left(\frac{\cos(0)}{e^0} \right) \\
  &= 2 i \pi 
\end{align*}

10. The function has an essential singularity at $z=0$.
Evaluating the Taylor expansion:
\begin{align*}
e^{\frac{2}{z^2}} &= \sum_{n=0}^{\infty} \frac{1}{n!} \left(\frac{2}{z^2}\right)^n \\
 &= 0 + \frac{1}{z^2} + ...
\end{align*}

So the coefficient of $\frac{1}{z}$ is 0:
\begin{align*}
J &= \int_\gamma e^{\frac{2}{z^2}} dz \\
  &= 2 i \pi \left( \Res(e^\frac{2}{z^2}, 0) \right) \\
  &= 2 i \pi (0) \\
  &= 0
\end{align*}

14. The function has an order 1 pole at $z_0 = 2e^{it}, t = 0,\frac{2\pi}{3}, \frac{4\pi}{3}$
but only $z_0 = 2 \in \bar{\gamma}$.
From the residue theory:
\begin{align*}
J &= \int_\gamma \frac{(1-z)^2}{z^3-8} dz \\
  &= 2i\pi \left( \lim_{z\to 2} \frac{(z-2)(1-z)^2}{z^3-8} \right) \\
  &= 2i\pi \left( \lim_{z\to 2} \frac{(z-2)(1-z)^2}{(z-2e^{i\frac{4\pi}{3}})(z-2e^{i\frac{2\pi}{3}})(z-2)} \right) \\
  &= 2i\pi \left( \lim_{z\to 2} \frac{(1-z)^2}{(z-2e^{i\frac{4\pi}{3}})(z-2e^{i\frac{2\pi}{3}})} \right) \\
  &= 2i\pi \left( \frac{(1-2)^2}{(2-2e^{i\frac{4\pi}{3}})(2-2e^{i\frac{2\pi}{3}})} \right) \\
  &= 2i\pi \left( \frac{1}{8+4i} \right) \\
  &= i\pi \left( \frac{1}{4+2i} \right) \\
  &= \frac{i\pi}{4+2i}
\end{align*}

\question*{\S 22.3 page 730: 2,6}
\emph{Find:}
\newline
In each of problems 2 and 6,
evaluate the integral. 
Where ever they appear $\alpha,\beta\in\mathcal{R}>0$.

2. 
\begin{equation*}
\int_{-\infty}^{\infty} \frac{1}{x^4+1} dx
\end{equation*}

6.
\begin{equation*}
\int_{-\infty}^{\infty} \frac{1}{x^2-2x+6} dx
\end{equation*}

\emph{Solution:}
\newline
2. Let $f(z) = \frac{p}{q}$ such that $q(z) = z^4+1$ and $p = 1$.
The roots of $q$ are then $z_0 = e^{\frac{i\pi k}{4}}, k = 1, 3, 5, 7$;
only roots with positive imaginary components are when $k=1,3$.
Let $\gamma = \{z |\quad |z|=R; z=a+bi, a,b\in\mathcal{R}, b>0\}$ as $R \to \infty$.
Applying the residue theory:
\begin{align*}
J &= \int_{-\infty}^{\infty} \frac{1}{x^4+1} dx \\
  &= \int_{\gamma} \frac{1}{z^4+1} dz \\
  &= 2i\pi \left( \Res( f, e^{i\frac{\pi}{4}}) + \Res( f, e^{i\frac{3\pi}{4}}) \right) \\
  &= 2i\pi \left( \frac{1}{4(e^{i\frac{\pi}{4}})^3} + \frac{1}{4 (e^{i\frac{3\pi}{4}})^3} \right) \\
  &= \frac{i\pi }{2} ( e^{\frac{-i3\pi}{4}} + e^\frac{-i9\pi}{4}  ) \\
  &= \frac{i\pi}{2} \left( \frac{-2i}{\sqrt{2}} \right) \\
  &= \frac{\pi}{\sqrt{2}}
\end{align*}

6. Let $f(z) = \frac{p}{q}$ such that $q(z) = z^2-2z+6$ and $p = 1$.
The roots of $q$ are then $z_0 = 1\pm i \sqrt{5}$;
only roots with positive imaginary components are $z=1+i\sqrt{5}$.
Let $\gamma = \{z |\quad |z|=R; z=a+bi, a,b\in\mathcal{R}, b>0\}$ as $R \to \infty$.
Applying the residue theory:
\begin{align*}
J &= \int_{-\infty}^{\infty} \frac{1}{z^2-2z+6} dx \\
  &= \int_{\gamma} \frac{1}{z^2-2z+6} dz  \\
  &= 2i\pi \left( \Res( f, 1+i\sqrt{5}) \right) \\
  &= 2i\pi \left( \frac{1}{2(1+i\sqrt{5})-2} \right) \\
  &= 2i\pi \left( \frac{1}{i\sqrt{5}} \right) \\
  &= \frac{2\pi}{\sqrt{5}}
\end{align*}

\newpage
\question
\emph{Find:}
\newline
Invert these Laplace Transforms:

13.
\begin{equation*}
\frac{1}{(z+3)^2}
\end{equation*}

14.
\begin{equation*}
\frac{1}{(z^2+9)(z-2)^2}
\end{equation*}

\emph{Solution:}
\newline

13. The function has a order 2 pole at $z_0=-3$.
\begin{align*}
\Res( e^{zt} F, -3) 
  &= \lim_{z\to -3} \left( \pdv{e^{zt}}{z} \right) \\
  &= \lim_{z\to -3} \left( te^{zt} \right) \\
  &= te^{-3t} \\
\\
\Rightarrow
f(t) &= te^{-3t} 
\end{align*}

14. The function has an order 2 pole at $z_0=2$ and order 1 poles at $z_0=\pm3i$.
\begin{align*}
\Res( e^{zt}F, 3i) 
  &= \lim_{z\to 3i} \left( \frac{e^{zt}}{(z+3i)(z-2)^2} \right) \\
  &= \frac{e^{i3t}}{6i(3i-2)^2} \\
\Res( e^{zt}F, -3i) 
  &= \lim_{z\to 3i} \left( \frac{e^{zt}}{(z-3i)(z-2)^2} \right) \\
  &= \frac{e^{-i3t}}{-6i(-3i-2)^2} \\
\Res( e^{zt}F, 2) 
  &= \lim_{z\to 3i} \left( \frac{\partial}{\partial z}\left( \frac{e^{zt}}{z^2+9} \right) \right)  \\ 
  &= \lim_{z\to 3i} \left( \frac{-2ze^{zt}}{(z^2+9)^2} + \frac{te^{zt}}{z^2+9}  \right)  \\ 
  &= (13t-4) \frac{e^{2t}}{196} \\
\\
\Rightarrow
f(t) 
  &= \frac{e^{i3t}}{6i(3i-2)^2} + \frac{e^{-i3t}}{-6i(-3i-2)^2} + (13t-4) \frac{e^{2t}}{196} \\
  &= \frac{5i+12}{1014} e^{i3t} + \frac{-5i+12}{1014} e^{-i3t} +\frac{13t-4}{196} e^{2t}
\end{align*}


\end{document}
