\documentclass[11pt]{homework}

\newcommand{\hwname}{Justin L. Clough}
\newcommand{\hwemail}{jlclough@usc.edu}
\newcommand{\hwtype}{Homework}
\newcommand{\hwnum}{2}
\newcommand{\hwclass}{AME: 525}

\begin{document}
\maketitle

\question*{ \S 10.4: Page 338: 8, 10, 12, 14, 16, 19,20}
\emph{Find:}
\newline
For questions 8 through 16, determine whether the vectors are 
linearly independent or dependent in the appropriate $\mathcal{R}^n$.
\newline
\newline
8.  $\vec u_1 = 3i +2j, \quad \vec u_2 = i - j$ in $\mathcal{R}^3$.
\newline
10. $\vec u_1 = <1,0,0,0>, \quad \vec u_2 = <0,1,1,0>, \quad \vec u_3 = <-4,6,6,0>$ in $\mathcal{R}^4$.
\newline
12. $\vec u_1 = <0,1,1,1>, \quad \vec u_2 = <-3,2,4,4>, \quad \vec u_3 = <-2,2,34,2>, \quad \vec u_4 = <1,1,-6,-2>$ in $\mathcal{R}^4$.
\newline
14. $\vec u_1 = <-1,1,0,0,0>, \quad \vec u_2 = <0,-1,1,0,0>, \quad \vec u_3 = <0,1,1,1,0>$ in $\mathcal{R}^5$.
\newline
16. $\vec u_1 = <3,0,0,4>, \quad \vec u_2 = <2,0,0,8>$ in $\mathcal{R}^4$.
\newline

\emph{Solution:}
\newline
For any set of vectors, $\{\vec u_i\}_{i=1}^{n}$, if a set of constants $a_i$ can be chosen
with $a_i \neq 0 \forall i=1,2,...,n$ such that $\sum_{i=1}^{n} a_i \vec u_i = \vec 0$ then 
the set of vectors are linearly dependent.

8. 
\begin{align*}
a_1 ( 3i + 2j) + a_2 (i - j) &= \vec 0 \\
  \Rightarrow \quad i: \quad 3 a_1 &= -a_2 \\
  \Rightarrow \quad j: \quad 2 a_1 &= a_2 \\
  \Rightarrow \quad 3 a_1 &= -2 a_1 \\
  \Rightarrow \quad 3 &= -2
\end{align*}
The vectors are \textbf{linearly independent}.

10. 
\begin{equation*}
a_1(1,0,0,0) + a_2 (0,1,1,0) +a_3 (-4,6,6,0) = \vec 0 
\end{equation*}

Choose $a_3 = 1$, $a_1 = 4$, and $a_2 = -6$.

\begin{equation*}
(4)(1,0,0,0) + (-6) (0,1,1,0) + (1) (-4,6,6,0) = \vec 0 
\end{equation*}
The vectors are \textbf{linearly dependent}.

\newpage
12. 
\begin{equation*}
a_1 (0,1,1,1) + a_2  (-3,2,4,4) + a_3  (-2,2,34,2) +  a_4  (1,1,-6,-2) = \vec 0
\end{equation*}

\begin{align*}
\begin{bmatrix}
  0 & -3  & -2 & 1 \\
  1 &  2  &  2 & 1 \\
  1 &  4  & 34 & -6 \\
  1 &  4  &  2 & -2 
\end{bmatrix}
\begin{bmatrix}
 a_1 \\
 a_2 \\
 a_3 \\
 a_4 
\end{bmatrix}
&=
\begin{bmatrix}
 0 \\
 0 \\
 0 \\
 0 
\end{bmatrix} 
\\
\begin{bmatrix}
U
\end{bmatrix}
\vec a 
&= 
\vec 0
\end{align*}

Since $\det ([U]) = -240 \neq 0$ then $\vec a = \vec 0$;
the set of vectors are \textbf{linearly independent}.

14. 
\begin{equation*}
a_1  (-1,1,0,0,0) + a_2  (0,-1,1,0,0) + a_3  (0,1,1,1,0) = \vec 0 
\end{equation*}
\begin{align*}
& \Rightarrow i: -a_1 = 0  \\
& \Rightarrow j: a_1 - a_2 + a_3 = 0  \\
& \Rightarrow k: a_2 + a_3 = 0  \\
& \Rightarrow l: a_3 = 0  
\end{align*}

The set of vectors are \textbf{linearly independent}.

16. 
\begin{equation*}
a_1  (3,0,0,4) + a_2  (2,0,0,8) = \vec 0
\end{equation*}
\begin{align*}
& \Rightarrow i: 3 a_1 + 2 a_2 = 0  \\
& \Rightarrow j:  0 = 0 \\
& \Rightarrow k:  0 = 0 \\
& \Rightarrow l: 4 a_1 + 8 a_2 = 0
\end{align*}
\begin{align*}
\Rightarrow a_2 &= -\frac{3}{2} a_1 \\
 4 a_1 + 8 ( -\frac{3}{2} a_1) &= 0 \\
 4 - 3 &= 0
\end{align*}

The set of vectors are \textbf{linearly independent}.

\newpage
\emph{Find:}
\newline
For questions 19 and 20, show that the set $\mathcal{S}$ is 
a subspace of the appropriate $\mathcal{R}^n$ and find the 
basis for this subspace and its dimension.
\newline
\newline
19. $\mathcal{S}$ consists of all vectors in $\mathcal{R}^n$ 
with zero second component.
\newline
20. $\mathcal{S}$ consists of all vectors in $\mathcal{R}^6$
of the form $(x, x, y, y, 0, z)$.

\emph{Solution:}

19. Consider $\vec x \in \mathcal{S} \subset \mathcal{R}^n$ 
such that $\vec x = (a, 0, b_1, b_2, ..., b_{n-2})$ where
$ a, b_i \in \mathcal{R} \quad \forall i=1,2,...,n-2$.
Check first that $\mathcal{S}$ is a subspace of $\mathcal{R}^n$:
\begin{enumerate}
  \item $\vec 0 \in \mathcal{S}$?
        Choose $a = b_i = 0 \forall i$ such that $\vec x = \vec 0$.
  \item Sum of two vectors in $\mathcal{S}$ remains in $\mathcal{S}$?
        Let $\vec y = (c, 0, d_1, d_2, ..., d_{n-2}) \in \mathcal{S}$ and 
        $\vec z = \vec y + \vec x$, then
        $\vec z = (a + c, 0, b_1+d_1, b_2+d_2, ..., b_{n-2}+d_{n-2}) \in \mathcal{S}$.
  \item Product of any vector in $\mathcal{S}$ with any real scalar is real?
        Let $k\in \mathcal{R}$ and $ \vec w = k \vec x$, then
        $\vec w = (ka, 0, kb_1, kb_2,...,kb_{n-2})$ with $ ka, kb_i \in \mathcal{R} \forall i$.
\end{enumerate}
So $\mathcal{S} \subset \mathcal{R}^n$.
\newline
The set of basis vectors are:
\begin{align*}
 \hat{e}_1 &= (1, 0, 0, 0, ..., 0, 0) \\
 \hat{e}_2 &= (0, 0, 1, 0, ..., 0, 0) \\
 \hat{e}_3 &= (0, 0, 0, 1, ..., 0, 0) \\
           &\vdots     \\
 \hat{e}_{n-2} &= (0, 0, 0, 0, ..., 1, 0) \\
 \hat{e}_{n-1} &= (0, 0, 0, 0, ..., 0, 1) \\
\end{align*}
There are $n-1$ basis vectors so the dimension of $\mathcal{S}$ is $n-1$.
\newline
\newline
20. Consider $\vec u \in \mathcal{S} \subset \mathcal{R}^6$
such that $\vec u = ( x,x, y,y, 0, z)$ where $x, y, z \in \mathcal{R}$.
Check first that $\mathcal{S}$ is a subspace of $\mathcal{R}^6$:
\begin{enumerate}
  \item $\vec 0 \in \mathcal{S}$? 
        Choose $x,y,z = 0$ such that $\vec u = (0,0,0,0,0,0) = \vec 0$.
  \item Sum of two vectors in $\mathcal{S}$ remains in $\mathcal{S}$?
        Let $\vec v = ( a,a, b,b, 0, c) \in \mathcal{S}$ and
        $ \vec w = \vec v + \vec u$, then
        $\vec w = (a+x, a+x, b+y, b+y, 0, c+z) \in \mathcal{S}$.
  \item Product of any vector in $\mathcal{S}$ with any real scalar is real?
        Let $k \in \mathcal{R}$ and $\vec \eta = k \vec u$, then
        $\vec \eta = (kx, kx, ky,  ky, 0, kz)$ with $kx, ky, kz \in \mathcal{R}$.
\end{enumerate}
So $\mathcal{S} \subset \mathcal{R}^6$.
\newline
The set of basis vectors are:
\begin{align*}
  \hat{e}_1 &= (1, 1, 0, 0, 0, 0) \\
  \hat{e}_2 &= (0, 0, 1, 1, 0, 0) \\
  \hat{e}_3 &= (0, 0, 0, 0, 0, 1)
\end{align*}
There are 3 basis vectors so the dimension of $\mathcal{S}$ is 3.

\question*{ \S 10.5: Page 342: 5, 7}
\emph{Find:}
\newline
For questions 5 and 7, use the Gram-Schmidt process to find an
orthogonal basis spanning the same subspace of $\mathcal{R}^n$ 
as the given set of basis vectors.
\newline
5. $\vec u_1 = (0, -1, 2, 0)$ and $\vec u_2 = (0, 3, -4, 0)$
\newline
7. $\vec u_1 = (-1, 0, 3, 0, 4)$, $\vec u_2 = (4, 0, -1, 0, 3)$, and $\vec u_3 = (0,0,-1, 0, 5)$.

\emph{Solution:}

5. The first basis vector $\hat{e}_1$:

\begin{align*}
||\vec u_1|| &= \sqrt{1 + 4} \\
             &= \sqrt{5}
\end{align*}
\begin{align*}
\hat{e}_1 &= \frac{\vec u_1}{||\vec u_1||} \\
          &= (0, -\frac{1}{\sqrt{5}}, \frac{2}{\sqrt{5}}, 0) 
\end{align*}
The second basis vector $\hat{e}_2$:
\begin{align*}
\vec v_2 &= \vec u_2 - (u_2, \hat{e}_1) \hat{e}_1 \\
         &= \begin{bmatrix}
             0 \\   
             3 \\   
            -4 \\  
             0
            \end{bmatrix}
            - (\frac{-3}{\sqrt{5}} + \frac{-8}{\sqrt{5}})
            \begin{bmatrix}
            0 \\ 
            -\frac{1}{\sqrt{5}} \\ 
            \frac{2}{\sqrt{5}} \\
            0
            \end{bmatrix} \\
         &= (0, \frac{4}{5}, \frac{2}{5}, 0) \\
||\vec v_2|| &= \sqrt{\frac{16}{25} + \frac{4}{25}} = \frac{2}{\sqrt{5}}
\end{align*}
\begin{align*}
\hat{e}_2 &= \frac{\vec v_2}{||\vec v_2||} \\
          &= (0, \frac{2}{\sqrt{5}}, \frac{1}{\sqrt{5}}, 0)
\end{align*}
\newpage
7. The first basis vector $\hat{e}_1$:
\begin{align*}
||\vec u_1|| &= \sqrt{1 + 9 + 16} \\
             &= \sqrt{26}
\end{align*}
\begin{align*}
\hat{e}_1 &= \frac{\vec u_1}{||\vec u_1||} \\
          &= (-\frac{1}{\sqrt{26}}, 0, \frac{3}{\sqrt{26}}, 0, \frac{4}{\sqrt{26}} ) 
\end{align*}
The second basis vector $\hat{e}_2$:
\begin{align*}
\vec v_2 &= \vec u_2 - (u_2, \hat{e}_1) \hat{e}_1 \\
         &= \begin{bmatrix}
             4 \\   
             0 \\   
            -1 \\  
             0 \\
             3
            \end{bmatrix}
            - (\frac{-4 -3 +12}{\sqrt{26}})
            \begin{bmatrix}
            -\frac{1}{\sqrt{26}} \\
            0 \\
            \frac{3}{\sqrt{26}} \\
            0 \\
             \frac{4}{\sqrt{26}} 
            \end{bmatrix} \\
         &= (\frac{109}{26}, 0,  \frac{-41}{26}, 0, \frac{29}{13}) \\
||\vec v_2|| &= \frac{1301}{10 \sqrt{26}}
\end{align*}
\begin{align*}
\hat{e}_2 &= \frac{\vec v_2}{||\vec v_2||} \\
          &= (\frac{1}{1301 \sqrt{26}})(1090, 0, -410, 0, 580)
\end{align*}
The third basis vector $\hat{e}_3$:
\begin{align*}
\vec v_3 &= \vec u_3 - (u_3, \hat{e}_1) \hat{e}_1  - (u_3, \hat{e}_2)\hat{e}_2 \\
         &= \begin{bmatrix}
             0 \\   
             0 \\   
            -1 \\  
             0 \\
             5
            \end{bmatrix}
            - (\frac{-3+20}{\sqrt{26}})
            \begin{bmatrix}
              -\frac{1}{\sqrt{26}} \\
              0 \\
              \frac{3}{\sqrt{26}} \\
              0 \\
               \frac{4}{\sqrt{26}} 
            \end{bmatrix}
            - (\frac{410+2900}{1301\sqrt{26}})
            \begin{bmatrix}
              \frac{1090}{1301\sqrt{26}} \\
              0 \\
              \frac{-410}{1301\sqrt{26}} \\
              0 \\
               \frac{580}{1301\sqrt{26}} 
            \end{bmatrix} \\
         &\approx (-1.478, 0, -2.160, 0, 1.250) \\
||\vec v_3|| &\approx 2.900
\end{align*}
\begin{align*}
\hat{e}_3 &= \frac{\vec v_3}{||\vec v_3||} \\
          &\approx (-0.510, 0, -0.745, 0, 0.431)
\end{align*}

\newpage
\question*{ \S 11.1: Page 357: 2, 6, 10, 12, 16, 18, 20}
\emph{Find:}
\newline
For questions 2 and 6, perform the indicated operations.
\newline
2. $-5A + 3B$ where
\begin{align*}
A &= 
 \begin{bmatrix}
  -2 & 2 \\
  0 & 1 \\
  14 & 2 \\
  6 & 8 
 \end{bmatrix}
&B =
 \begin{bmatrix}
  4 & 4 \\
  2 & 1 \\
  14 & 16 \\
  1 & 25
 \end{bmatrix}
\end{align*}

6. $A^3 - B^2$ where
\begin{align*}
A &= 
 \begin{bmatrix}
  -2 & 3 \\
  1 & 1 
 \end{bmatrix}
&B = 
 \begin{bmatrix}
  0 & 8 \\
  -5 & 1 
 \end{bmatrix}
\end{align*}

\emph{Solution:}
\newline
2. 
\begin{align*}
-5A + 3B &=
-5
 \begin{bmatrix}
  -2 & 2 \\
  0 & 1 \\
  14 & 2 \\
  6 & 8 
 \end{bmatrix}
+ 3
 \begin{bmatrix}
  4 & 4 \\
  2 & 1 \\
  14 & 16 \\
  1 & 25
 \end{bmatrix} \\
&=
 \begin{bmatrix}
  10 & -10 \\
  0 & -5 \\
  -70 & -10 \\
  -30 & -40
 \end{bmatrix}
+ 
 \begin{bmatrix}
  12 & 12 \\
  6 & 3 \\
  42 & 48 \\
  3 & 75
 \end{bmatrix}\\
&= 
 \begin{bmatrix}
  22 & 2 \\
  6 & -2 \\
  -28 & 38 \\
  -27 & 35
 \end{bmatrix}
\end{align*}

6. 
\begin{align*}
A^3 - B^2 &=
 \begin{bmatrix}
  -2 & 3 \\
  1 & 1 
 \end{bmatrix}^3
-
 \begin{bmatrix}
  0 & 8 \\
  -5 & 1 
 \end{bmatrix}^2 \\
&=
 \begin{bmatrix}
  -2 & 3 \\
  1 & 1 
 \end{bmatrix}
 \begin{bmatrix}
  -2 & 3 \\
  1 & 1 
 \end{bmatrix}
 \begin{bmatrix}
  -2 & 3 \\
  1 & 1 
 \end{bmatrix}
-
 \begin{bmatrix}
  0 & 8 \\
  -5 & 1 
 \end{bmatrix}
 \begin{bmatrix}
  0 & 8 \\
  -5 & 1 
 \end{bmatrix} \\
&=
 \begin{bmatrix}
  -17 & 18 \\
  6 & 1 
 \end{bmatrix}
-
 \begin{bmatrix}
  -40 & 8 \\
  -5 & -39
 \end{bmatrix} \\
&=
 \begin{bmatrix}
  23 & 10 \\
  11 & 40
 \end{bmatrix}
\end{align*}

\newpage
\emph{Find:}
\newline
For questions 10, 12, and 16, determine whether $AB$, $BA$,
or both products are defined; carry out the defined products.
\newline
10. 
\begin{align*}
A&=
  \begin{bmatrix}
    -3 & 1 \\
    6 & 2 \\
    18 & -22 \\
    1 & 6
  \end{bmatrix}
&B=
  \begin{bmatrix}
    -16 & 0 & 0 & 28 \\
    0   & 1 & 1 & 26 
  \end{bmatrix}
\end{align*}
12. 
\begin{align*}
A&=
  \begin{bmatrix}
    -2 & 4 \\
    3 & 9 
  \end{bmatrix}
&B=
  \begin{bmatrix}
    1 & -3 & 7 & 2 \\
    5 & 9 & 1 & 0 
  \end{bmatrix}
\end{align*}  
16. 
\begin{align*}
A&=
  \begin{bmatrix}
    -3 & 2 \\
    0 & -2 \\
    1 & 8 \\
    3 & -3 
  \end{bmatrix}
&B=
  \begin{bmatrix}
    -5 & 5 & 7 & 2
  \end{bmatrix}
\end{align*}  

\emph{Solution:}
\newline
10. Both $AB$ and $BA$ are possible:
\begin{align*}
AB &= 
  \begin{bmatrix}
    -3 & 1 \\
    6 & 2 \\
    18 & -22 \\
    1 & 6
  \end{bmatrix}
  \begin{bmatrix}
    -16 & 0 & 0 & 28 \\
    0   & 1 & 1 & 26 
  \end{bmatrix} \\
&= 
  \begin{bmatrix}
    48 & 1 & 1 & -51 \\
    -96 & 2 & 2 & 220 \\
    -288 & -22 & -22 & -68 \\
    -16 & 6 & 6 & 184 \\
  \end{bmatrix} \\
BA &= 
  \begin{bmatrix}
    -16 & 0 & 0 & 28 \\
    0   & 1 & 1 & 26 
  \end{bmatrix} 
  \begin{bmatrix}
    -3 & 1 \\
    6 & 2 \\
    18 & -22 \\
    1 & 6
  \end{bmatrix} \\
&= 
  \begin{bmatrix}
    76 & 152 \\
    50 & 136
  \end{bmatrix} 
\end{align*}

12. Only $AB$ is possible:
\begin{align*}
AB &= 
  \begin{bmatrix}
    -2 & 4 \\
    3 & 9 
  \end{bmatrix}
  \begin{bmatrix}
    1 & -3 & 7 & 2 \\
    5 & 9 & 1 & 0 
  \end{bmatrix} \\
& = 
  \begin{bmatrix}
    18 & 42 & -10 & -4 \\
    48 & 72 & 30 & 6
  \end{bmatrix}
\end{align*}

16. Only $BA$ is possible:
\begin{align*}
BA &=
  \begin{bmatrix}
    -5 & 5 & 7 & 2
  \end{bmatrix}
  \begin{bmatrix}
    -3 & 2 \\
    0 & -2 \\
    1 & 8 \\
    3 & -3 
  \end{bmatrix}\\
& = 
  \begin{bmatrix}
    28 & 30
  \end{bmatrix}
\end{align*}
\newline
\emph{Find:}
\newline
For questions 18 and 20, determine if the product $AB$ and/or $BA$ is defined;
determine the dimensions of the product matrix.
\newline
18. $A$ is $14 \times 21$ and $B$ is $ 21 \times 14$.
\newline
20. $A$ is $1 \times 3$ and $B$ is $3\times 3$.
\newline
\newline
\emph{Solution:}
\newline
18. Both products are defined; $AB$ is $ 14 \times 14$ 
and $BA$ is $21 \times 21$.
\newline
20. Only $AB$ is defined; it is $1 \times 3$.



\end{document}

