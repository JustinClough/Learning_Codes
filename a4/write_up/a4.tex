\documentclass[a4paper, 12pt]{article}
\author{Justin Clough, RIN:661682899}
\title{FEP Assignment 4}
\usepackage{geometry}
\usepackage{float}
\usepackage{subfigure}
\usepackage[justification=centering]{caption}
\usepackage{enumerate}
\usepackage{multirow}
\usepackage{listings}
\lstset{
    language=C++,
    numbers=left,
    tabsize=2,
    prebreak=\raisebox{0ex}[0ex][0ex]{\ensuremath{\hookleftarrow}},
    frame=single,
    breaklines=true,
}
\usepackage{graphicx}
\graphicspath{ {images/} }
\usepackage{nameref}
\usepackage{amsmath}
\usepackage{amssymb}

\begin{document}
\maketitle

\newpage
\section{Introduction} \label{sec:intro}
The purpose of this project was to develop a Finite
Element Analysis (FEA) code that solves linear
elliptic problems. This project does so for
three dimensional solid mechanic problems. The user
provides the code with the following information
(an example input script is shown in
Appendix \ref{subsec:ExIn}):

\begin{itemize}
  \item A geometric model of the domain in the
        form of a \emph{.dmg} file.
  \item The corresponding mesh of the domain in the
        from of a \emph{.smb} file.
  \item An associations file which define geometric
        node sets in the form of a \emph{.txt} file. The
        details of this file are explained in detail in
        section \ref{sec:techDes}.
  \item A control file which define material properties,
        boundary conditions, and linear algebra details in
        the form of a \emph{.yaml} file.
  \item A file name to write the solution to as an array of
        characters.
  \item The order of numerical integration as an integer.
  \item The body load in the form of three floating point
        number.
\end{itemize}

Based on these input parameters, the code assembles and
solves the relevant finite element problem to retrieve
the displacement at each mesh Degree of Freedom (DOF).
It then calculates the Cauchy stress tensor for each
integration point of each element. It finally writes
the displacement, traction, and Cauchy stress fields to
file in \emph{.vtk} format.

The rest of this report is separated into the following
section. The technical description of the code, which
includes an overview of the Finite Element Method (FEM),
the numerical integration techniques used, and the
linear algebra assembly and solution methods is shown in
section \ref{sec:techDes}. A description of the code, which
includes an outline of classes created and a pseudo code
is shown in section \ref{sec:codeDes}. The tests performed
are described in section \ref{sec:testing}. Finally,
conclusions and closing comments are made
in \ref{sec:conclusion}. The source code and headers
are appended to this report in Appendix \ref{sec:code}.

\section{Technical Description} \label{sec:techDes}
The purpose of the code is to approximate the displacement
$u$ subjected to the conditions shown in Equations
\ref{eq:StrongForm} through \ref{eq:Compliance}.

\begin{align}
\sigma_{ij},_{j} + f_i &= 0  &  &\text{on} \quad  \Omega
  \label{eq:StrongForm}                                          \\
u_i &= g_i                   &  &\text{on} \quad \Gamma^{g}_{i}
  \label{eq:DBC}                                                 \\
\sigma_{ij} \cdot n_j &= h_i &  &\text{on} \quad \Gamma^{h}_{i}
  \label{eq:NBC}                                                 \\
\varepsilon_{ij} &= u_{i},_{j}&  &\quad
  \label{eq:strainDef}                                           \\
\sigma_{ij} &= c_{ijkm} \varepsilon_{km} & &\quad
  \label{eq:Compliance}
\end{align}

\noindent
where $\sigma$ is the Cauchy stress tensor in
the domain $\Omega$, $f$ is the
vector-valued traction, and $\varepsilon$ is the strain.
The subscripts indicate the spatial dimension of the vector or
tensor they follow and range from 1 to $n_{sd}$, the number
of spatial dimensions; summation is implied for repeated
indices.  The domain $\Omega$ is bounded by boundaries
$\Gamma^{g}$ and $\Gamma^{h}$ as shown
in Equations \ref{eq:OmegaUnion} through \ref{eq:GammaUnion}.

\begin{align}
\Omega &= \hat{\Omega} \cup \Gamma
  \label{eq:OmegaUnion}               \\
\Gamma &= \bigcup_{i=1}^{n_{sd}} \Gamma^{g}_{i} \cup \Gamma^{h}_{i}
  \label{eq:GammaUnion}
\end{align}

\noindent
where $\Gamma$ is the total boundary of domain $\Omega$
and $\hat{\Omega}$ is the internal portion of $\Omega$.
The super scripts, $g$ and $h$ correspond
to the Dirichlet and Neumann boundary conditions shown in Equations
\ref{eq:DBC} and \ref{eq:NBC}, respectively. The Dirichlet and
Neumann boundary conditions are not defined on the same location
for the same spatial dimension, as shown in Equation \ref{eq:DNconf}.

\begin{equation} \label{eq:DNconf}
\Gamma^{g}_{i} \cup \Gamma^{h}_{j} = \varnothing \quad \text{if} \quad i=j
\end{equation}

The compliance tensor, $c$ relates the stain and stress by
material properties.

\subsection{Finite Element Method} \label{subsec:fem}
FE text here.

\subsection{Numerical Integration} \label{subsec:numInt}
Numerical integration text here.

\subsection{Linear Algebra} \label{subsec:LinAlg}
Linear algebra text here

\section{Code Description} \label{sec:codeDes}
Pseudo code intro text here.

\subsection{Class Description} \label{subsec:class}
Class description text here.

\subsection{Pseudo Code} \label{subsec:pseudo}
Pseudo code text here.

\section{Testing} \label{sec:testing}
Testing text here.

\subsection{Linear Tetrahedron Elements} \label{subsec:linTet}
Linear Tri element text here.

Convergence of linear tet elements here.

\subsection{Linear Hexahedron Elements} \label{subsec:linHex}
Linear quad text here.

\subsection{Quadratic Tetrahedron Elements} \label{subsec:quadTet}
Quadratic Tri element text here.

Convergence of quad tets here.

\subsection{Quadratic Hexahedron Elements} \label{subsec:quadHex}
Quadratic quad element text here.

\section{Conclusion} \label{sec:conclusion}
conclusion text here.

\subsection{Closing Comments} \label{sec:comments}
closing comments here.

\newpage
\appendix
\section{Source Code and Headers} \label{sec:code}

\subsection{a4.cc} \label{subsec:a4.cc}
\lstinputlisting{/lore/clougj/FiniteElementProgramming/a4/src/a4.cc}

\subsection{A4\_BodyLoads.hpp} \label{subsec:BLhpp}
\lstinputlisting{/lore/clougj/FiniteElementProgramming/a4/src/A4_BodyLoads.hpp}

\subsection{A4\_BodyLoads.cpp} \label{subsec:BLcpp}
\lstinputlisting{/lore/clougj/FiniteElementProgramming/a4/src/A4_BodyLoads.cpp}

\subsection{A4\_Control.hpp} \label{subsec:Cont.hpp}
\lstinputlisting{/lore/clougj/FiniteElementProgramming/a4/src/A4_Control.hpp}

\subsection{A4\_Control.cpp} \label{subsec:Cont.cpp}
\lstinputlisting{/lore/clougj/FiniteElementProgramming/a4/src/A4_Control.cpp}

\subsection{A4\_DBCs.hpp} \label{subsec:DBCs.hpp}
\lstinputlisting{/lore/clougj/FiniteElementProgramming/a4/src/A4_DBCs.hpp}

\subsection{A4\_DBCs.cpp} \label{subsec:DBCs.cpp}
\lstinputlisting{/lore/clougj/FiniteElementProgramming/a4/src/A4_DBCs.cpp}

\subsection{A4\_Disc.hpp} \label{subsec:Disc.hpp}
\lstinputlisting{/lore/clougj/FiniteElementProgramming/a4/src/A4_Disc.hpp}

\subsection{A4\_Disc.cpp} \label{subsec:Disc.cpp}
\lstinputlisting{/lore/clougj/FiniteElementProgramming/a4/src/A4_Disc.cpp}

\subsection{A4\_ElasticStiffness.hpp} \label{subsec:ElasticStiffness.hpp}
\lstinputlisting{/lore/clougj/FiniteElementProgramming/a4/src/A4_ElasticStiffness.hpp}

\subsection{A4\_ElasticStiffness.cpp} \label{subsec:ElasticStiffness.cpp}
\lstinputlisting{/lore/clougj/FiniteElementProgramming/a4/src/A4_ElasticStiffness.cpp}

\subsection{A4\_FESolver.hpp} \label{subsec:FESolver.hpp}
\lstinputlisting{/lore/clougj/FiniteElementProgramming/a4/src/A4_FESolver.hpp}

\subsection{A4\_FESolver.cpp} \label{subsec:FESolver.cpp}
\lstinputlisting{/lore/clougj/FiniteElementProgramming/a4/src/A4_FESolver.cpp}

\subsection{A4\_LinAlg.hpp} \label{subsec:LinAlg.hpp}
\lstinputlisting{/lore/clougj/FiniteElementProgramming/a4/src/A4_LinAlg.hpp}

\subsection{A4\_LinAlg.cpp} \label{subsec:LinAlg.cpp}
\lstinputlisting{/lore/clougj/FiniteElementProgramming/a4/src/A4_LinAlg.cpp}

\subsection{A4\_LinSolve.hpp} \label{subsec:LinSolve.hpp}
\lstinputlisting{/lore/clougj/FiniteElementProgramming/a4/src/A4_LinSolve.hpp}

\subsection{A4\_LinSolve.cpp} \label{subsec:LinSolve.cpp}
\lstinputlisting{/lore/clougj/FiniteElementProgramming/a4/src/A4_LinSolve.cpp}

\subsection{A4\_NBCs.hpp} \label{subsec:NBCs.hpp}
\lstinputlisting{/lore/clougj/FiniteElementProgramming/a4/src/A4_NBCs.hpp}

\subsection{A4\_NBCs.cpp} \label{subsec:NBCs.cpp}
\lstinputlisting{/lore/clougj/FiniteElementProgramming/a4/src/A4_NBCs.cpp}

\subsection{A4\_PostProc.hpp} \label{subsec:PostProc.hpp}
\lstinputlisting{/lore/clougj/FiniteElementProgramming/a4/src/A4_PostProc.hpp}

\subsection{A4\_PostProc.cpp} \label{subsec:PostProc.cpp}
\lstinputlisting{/lore/clougj/FiniteElementProgramming/a4/src/A4_PostProc.cpp}

\newpage
\section{Example Input Files}\label{sec:ExInput}

\subsection{Associations File} \label{subsec:ExAssoc}
\lstinputlisting{/lore/clougj/FiniteElementProgramming/a4/test/box3D.txt}

\subsection{Example .yaml File} \label{subsec:ExYaml}
\lstinputlisting{/lore/clougj/FiniteElementProgramming/a4/test/fem.yaml}

\subsection{Example Input Script} \label{subsec:ExIn}
\lstinputlisting{/lore/clougj/FiniteElementProgramming/a4/test/run_test.sh}


\end{document}
