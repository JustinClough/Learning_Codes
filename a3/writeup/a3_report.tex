\documentclass[a4paper, 12pt]{article}
\author{Justin Clough, RIN:661682899}
\title{FEP: Assignment 3}
\usepackage[margin=1.0in]{geometry}
\usepackage{float}
\usepackage{enumerate}
\usepackage{listings}
\lstset{
    frame=single,
    breaklines=true,
}

\begin{document}
\maketitle

\section*{Algorithm Description}

\subsection*{Operation 1} \label{sec:sub:op1}

The number of mesh regions for each part was determined by 
first iterating over all mesh vertices. For each mesh vertex, 
its third dimensional adjacencies, surrounding mesh regions, were 
collected into a temporary vector. The contents of this temporary 
vector were then added to another data structure if it did not 
already exist in the other structure. The size of this data 
structure was then the number of regions on each part. The mesh
vertices were then iterated over again. For each mesh vertex, 
it was first checked if it was owned by the operating process. 
If it was owned, then the local and global identification numbers 
were recorded. The number of adjacent regions was recorded as well 
for each vertex. The results for this operation are shown in the
results section. Printed out content from process zero is followed
by process one.

\subsection*{Operation 2} \label{sec:sub:op2}

A similar process as shown in Operation 1 was used. The number of
mesh faces for each part was determined by first iterating over all mesh
vertices. For each mesh vertex, 
its first dimensional adjacencies, surrounding mesh edges, were 
collected into a temporary vector. The contents of this temporary 
vector were then added to another data structure if it did not 
already exist in the other structure. The size of this data 
structure was then the number of edges on each part. The mesh
vertices were then iterated over again. For each mesh vertex, 
it was first checked if it was owned by the operating process. 
If it was owned, then the local and global identification numbers 
were recorded. The number of adjacent faces was recorded as well 
for each vertex. The results for this operation are shown in the
results section.
Printed out content from process zero is followed
by process one.


\subsection*{Operation 3} \label{sec:sub:op3}

The number of mesh faces on the partition model face before migration
was determined by iterating over all mesh faces. An integer counter
was started at zero and was incremented if a mesh face was on the 
part boundary. After all mesh faces were iterated on, the value of 
the counter was recorded. Next, a migration plan was constructed. 
Process zero then iterated over all mesh regions. For each mesh 
region, it was first checked if it was owned by process zero. If it
was, then its second dimension adjacencies were collected in a 
temporary vector. The faces in this temporary vector were then 
iterated over and each checked if they were on the part boundary.
If any of each region's faces were on the part boundary, then the 
region was added to another data structure. The region was only added
to this other data structure if it had not already been added. The 
regions in this data structure were then iterated over. Each region was 
then added to the migration plan and its local identification number 
was recorded. Mesh migration was then performed and the resulting mesh
was verified. The mesh faces were then iterated over again. A counter 
was stared at zero and was incremented each time a face was found on 
the boundary. The results for this operation are shown in the
results section.
Printed out content from process zero is followed
by process one. The code to execute the above three process begins
on page \pageref{sec:code}.

\section*{Operation 1 Results} \label{sec:results1}
\lstinputlisting{
  /lore/clougj/FiniteElementProgramming/a3/Op1Results_Proc0.txt}
\lstinputlisting{
  /lore/clougj/FiniteElementProgramming/a3/Op1Results_Proc1.txt}
\section*{Operation 2 Results} \label{sec:results2}
\lstinputlisting{
  /lore/clougj/FiniteElementProgramming/a3/Op2Results_Proc0.txt}
\lstinputlisting{
  /lore/clougj/FiniteElementProgramming/a3/Op2Results_Proc1.txt}
\section*{Operation 3 Results} \label{sec:results3}
\lstinputlisting{
  /lore/clougj/FiniteElementProgramming/a3/Op3Results_Proc0.txt}
\lstinputlisting{
  /lore/clougj/FiniteElementProgramming/a3/Op3Results_Proc1.txt}

\newpage
\section*{Code} \label{sec:code}
\lstinputlisting{/lore/clougj/FiniteElementProgramming/a3/a3.cc}

\end{document}
